\chapter{Graeme Segal -- A Perspective on Quantum Field Theory}

\section{6/10 -- Overview of QFT}

\subsection{From classical to quantum}

In classical mechanics, one focuses one's attention on a fixed space $M_0$ and obtains a configuration space $X$ from that.
Dynamics are specified by giving a Lagrangian $L: TX \to \RR$, which we assume is positive, inhomogeneous, and quadratic on tangent vectors.
Under these hypotheses, the Lagrangian can be viewed as a generalized Riemannian metric, so that time evolution is given by a (generalized) ``geodesic flow.''
This gives rise to a symplectic form $\omega$ and Poisson bracket $\{-,-\}$ on $T^* X$.
Furthermore, from the Lagrangian $L$, we may extract a Hamiltonian $H: T^*X \to \RR$ such that
\[
	\frac{d}{dt} f = \{ H, f \} \hspace{1em} \forall f \in \Cinfty(X).
\]
Thus we may view a classical dynamical system as a triple $(Y, \omega, H)$ with $(Y, \omega)$ a symplectic manifold and $H: Y \to \RR$ a function.

Quantum mechanics can be viewed as a complex noncommutative analogue of this.
We replace $(Y, \omega, H)$ by a triple $(\Ac, \star, H)$ with $\Ac$ a (topological) $\CC$-algebra, $\star$ an antilinear involution on $\Ac$, and $H \in \Ac$.
For an observable $f \in \Ac$, we require
\[
	\frac{d}{dt} f = i h [H, f].
\]
If the quantum system $(\Ac, \star, H)$ is a ``quantization'' of a classical system $(Y, \omega, H)$, then there is a deep relationship between the two.
For example, we should view $\Ac$ as a completion / extension of $\Cinfty(Y)$, and we can relate the values of an observable to the eigenvalues of the corresponding operator.

The above picture privileges the time dimension, so it is fundamentally non-relativistic.
Working relativistically requires us to move from finite-dimensional spaces to infinite-dimensional spaces.
Heuristically, we take $Y$ to be our space of fields on spacetime (typically $M = M_0 \times \RR$), and we fix a Lagrangian $L: Y \to \RR$.
However, $\Ac$ should now be viewed as a space of functions on an ``almost finite-dimensional manifold.''
This requires many corrections! 

\begin{ex}
	Consider scalar field theory, with $M = M_0 \times \RR$ and $Y = \Cinfty(M)$.
	Then $\Ac$ looks like functions on a stratified configuration space
	\[
		\coprod_{n \geq 0} \Conf_n(M_0),
	\]
	where $\Conf_n(M_0)$ is the space of sets of $n$ distinct unordered points on $M_0$.
\end{ex}

\subsection{Approaching quantum field theory}

Heuristically, we should ``spread things out over spacetime'' by attaching an algebra $\Oc_x$ to every point $x \in M$ and letting
\[
	\Ac \approx \bigotimes_{x \in M} \Oc_x.
\]
This has been formalized in the approach of algebraic quantum field theory -- see e.g.\ the definition of Haag, which assigns an algebra $\Ac_U$ to each open set $U \subset M$.

An alternative approach, which we will pursue, privileges the time dimension.
In this approach, we define a $d$-dimensional quantum field theory as a symmetric monoidal functor
\[
	E: \Cob_{d-1} \to \Vect.
\]
Here $\Cob_{d-1}$ is the symmetric monoidal category with:
\begin{itemize}
	\item Objects: $(d - 1)$-manifolds, often assumed compact or compact with boundary (and thought of as time slices)
	\item Morphisms $M_0 \to M_1$: $d$-dimensional cobordisms from $M_0$ to $M_1$ (thought of as controlling time evolution)
	\item Monoidal structure: disjoint union of manifolds / cobordisms
\end{itemize}
and $\Vect$ is the symmetric monoidal category of (possibly infinite-dimensional) topological vector spaces (with $\otimes$ as tensor product).

There are many other ways to view quantum field theory.
For example, one could think of QFT as a continuum limit of lattice models.

QFT is traditionally seen as describing ``the world except for gravity.''
That is, we think of spacetime as fixed, without allowing for the effects of gravity.
However, QFT can tell us about what sorts of gravitational effects are possible.

The perspective on QFT that we will follow has the advantage that it allows for direct comparisons between QFTs.
Thus, in principle, we may consider a ``moduli space'' of QFTs (of a given type).
We can think of gravitational effects as acting upon this moduli space.

\subsection{The perspective of Connes}

We'd like to use geometry to understand the noncommutative world of QFT.
This can be accessed by studying the spectrum of our Hamiltonian $H$.
Operators which evolve slowly in time ``nearly commute'' with $H$ and thus must be ``nearly diagonal.''
We're mostly interested in studying such operators (as the time scale of humans is much slower than that of the universe).
Thus, when studying QFT, we're interested in algebras which are ``nearly commutative.''

Connes was interested in finding a mildly noncommutative generalization of algebras of functions on a manifold $M$.
Recall that we can define the Clifford algebra $\Cliff(T_m M)$ as the algebra with generators $\{ \gamma_\xi \}_{\xi \in T_m M}$ and relations $\gamma_\xi^2 = -\norm{\xi}^2$.
Let $\Bc$ be a bundle of finite-dimensional algebras on $M$ containing $\Cliff(T_m M)$, and take a connection on $\Bc$.
Let $\Dsl$ be the Dirac operator
\[
	\Dsl = \sum_i \gamma_{\xi_i} \otimes \frac{\partial}{\partial x_i},
\]
and let $H = \Dsl^2$.
Connes gave a formula for $\tr \Dsl^2$ and extracted interesting physical objects, e.g.\ the Higgs field, from $\Dsl$.
From Connes' perspective, the Dirac operator is ``as good as'' the metric.

\subsection{Algebra from functorial QFT}

Suppose we view spacetime $M$ as a cobordism from $M_0$ to $M_1$.
Fix a $d$-dimensional quantum field theory $E$, where $E = \dim M$.
For $x \in M$, we can define a vector space of operators $\Oc_x$ by taking a small disk $D_x$ about $x$ and letting $\Oc_x = E_{\partial D_x}$.
We can view a punctured copy of $M$ as giving a cobordism
\[
	M \sqcup \left(\sqcup_i \partial D_{x_i} \right) \to M_1,
\]
so that applying $E$ gives $E_{M_0} \otimes (\otimes_i \Oc_{x_i}) \to E_{M_1}$.
This gives a sort of ``higher multiplication'' on the spaces of operators $\Oc_{x_i}$.

We can make this much more precise in special cases.
For example, if $E$ is topological, then $\Oc_{x_i} = \Oc$ is independent of $x_i$.
Taking $M_0 = \emptyset$ and $M_1 = S^{d-1}$, we obtain a family of multiplications $\Oc^{\otimes n} \to \Oc$.

\begin{ex}
	If $d = 2$, the cap / cup / pair-of-pants cobordisms equip $A = \Oc$ with the structure of a finite-dimensional commutative Frobenius algebra.
	This means that $A$ is a unital algebra with a nondegenerate trace $\theta: A \to \CC$.
\end{ex}

It is interesting to extend this picture to non-topological $2$-dimensional QFTs.
Here, one can set up a moduli space of QFTs and obtain a gravitational flow.
This was a major historical motivation for string theory.

The second lecture will discuss the importance of positive energy to this theory.
The third lecture will discuss scaling, and the fourth lecture will focus on finding a suitable definition for $\Cob$.

\section{6/10 -- Positive Energy}

\subsection{Warmup}

Consider a quantum mechanical system with Hilbert space $\Hc$.
The Hamiltonian is a self-adjoint operator $H: \Hc \to \Hc$, and we can integrate $iH$ to get unitary operators $U_t = e^{i H t}: \Hc \to \Hc$ describing the time evolution of the system.

Assuming $H \geq 0$ has strong consequences for the theory.
In this case, $U_{-}: \RR \to \End(\Hc)$ is the boundary value of the function on the complex upper half-plane also given by $t \mapsto e^{i H t}$.

To fit this into QFT, consider $\CC$ as a bundle over $\RR$, where the inner product on the base is Lorentzian and the inner product on the fiber is Riemannian.
Think of $\RR$ as our spacetime $M$ and the fiber as $V = T_x M$.

We will introduce the notion of a \emph{complex metric} on a real vector space $V$.
This is a quadratic form $V \to \CC$ satisfying a certain condition to be determined later.

\subsection{Historical digression}

The notion of a complex metric was introduced by Kontsevich and Segal.
Segal was originally interested in loop groups $\Lc G = \Map(S^1, G)$.
There is a nice class of ``positive energy'' representations of loop groups.
These extend to representations of $\Diff^+(S^1) \ltimes \Lc G$ which are also ``positive energy'' in a certain sense.
This notion effectively means that the actions respect the grading $\Hc = \oplus_{k \geq 0} \Hc_k$ coming from an $S^1$-action.

An even simpler illustrative case is that of ``discrete series'' representations of $\PSL_2(\RR)$.
Embed $\PSL_2(\RR) \hookrightarrow \PSL_2(\CC)$.
Let $\PSL_2^<(\CC)$ be the subsemigroup of M\"obius transformations sending the unit disk $D$ to a proper subdisk of itself.
Then $\PSL_2^<(\CC)$ acts by contraction on the discrete series representations.

There is a similar story for loop groups, if we consider the semigroup of holomorphic $f: D \to D$ with $f(D) \subset \mathring{D}$.
The Kontsevich-Segal definition connects to this somehow (I didn't quite catch how).

\subsection{Complex metrics}

\begin{dfn}
	A \emph{complex metric} on a real vector space $V$ is a quadratic form $g: V \to \CC$ such that, if $\lambda_k$ are the eigenvalues of the matrix corresponding to $g$ and $\lambda_k = e^{i \theta_k} \abs{\lambda_k}$, then $\sum_i \abs{\theta_i} < \pi$.
\end{dfn}

We can equivalently state this as
\[
	-\pi \leq \sum_i \pm \theta_i \leq \pi.
\]
One may view the $\theta_i$ as weights of the corresponding spin representation.

For $v \in V$, we have $g(v) \in \CC \setminus (-\infty, 0)$, so $g(v)$ has a canonical square root.
Thus we may assign a ``complex length'' in the positive half-plane to every $v \in V$.

Given an inner product $V \times V \to \CC$, we get an inner product on $\wedge^k V$ with squared norm given by $\alpha \wedge (\star \alpha)$.
Really, this is better thought of as a map $\wedge^k V \times \wedge^k V \to \wedge^d V$, for $d = \dim V$.

We'll change topics now.

\subsection{Application to QFT}

If we want to view a QFT as a functor $E$ defined on a category of cobordisms, we should require:
\begin{itemize}
	\item Cobordisms should be equipped with complex metrics.
	\item The induced maps $U_M : E_{M_0} \to E_{M_1}$ should be holomorphic in $M$ (viewed as living in some complex analytic moduli space of complex metrics).
	\item The operators $U_M$ should be \emph{trace class}: there should be bases $\xi_i$ for $E_{M_0}$ and $\eta_i$ for $E_{M_1}$ with $\xi_i \mapsto \lambda_i \eta_i$ and $\sum_i \abs{\lambda_i} < \infty$.
\end{itemize}

The case of Lorentzian metrics occurs ``on the boundary'' of the moduli space of complex metrics.
Thus usual Lorentzian QFTs can be viewed as holomorphic degenerations of nice QFTs.

Some simpler examples are similar in principle:
\begin{itemize}
	\item $\Diff^+(S^1)$ can be viewed as the boundary of the semigroup of contraction mappings.
	\item $\U(n) \subset \GL_n(\CC)$ can be viewed as the boundary of the semigroup of contraction operators $\CC^n \to \CC^n$.
\end{itemize}

However, $\PSL_2(\RR)$ is not compact, so we can't view it as a boundary.

Viewing our spacetime $M$ as Lorentzian, we can ask for $M$ to be ``globally hyperbolic,'' so every point in $M_0$ can be connected to a point in $M_1$ via a timelike curve.
This is similar to requiring that there are no black holes.
The results of Kontsevich-Segal are best in the case of globally hyperbolic metrics.

Heuristically, we are requiring the existence of a Dirac operator and requiring that this Dirac operator is a contraction operator.

\section{6/11 -- Desired Properties of QFT; Scaling}

In his first lecture, Ibou Bah proposed five properties that a (nice) QFT should satisfy: \emph{unitarity}, \emph{locality}, \emph{causality}, \emph{cluster decomposition}, and \emph{renormalization}.
We will first try to understand how these relate to Segal's perspective.

\subsection{Renormalization}

Historically, understanding renormalization was crucial to the development of QFT.
This is somewhat arcane from a mathematical point of view.
However, one should note that renormalizability of a Lagrangian does have a natural relationship to the calculus of variations.

Consider a $\sigma$-model where our fields are maps $M \to N$.
If $M$ and $N$ have Riemannian metrics, we can define an energy functional.
For $M = \RR$ or $S^1$, this recovers the usual energy functional on paths.
This is well-behaved and has a good Morse theory.
When $\dim M = 2$, this is almost well-behaved, except there are problems with bubbling.
For $\dim M > 2$, things are terrible.

The upshot is that (at least in this case?) renormalizability happens exactly when we expect things to be mathematically well-behaved.

\subsection{Unitarity}

If $M$ is a cobordism from $M_0$ to $M_1$, we can reverse orientations on the boundary and view $\ol{M}$ as a cobordism from $\ol{M}_1$ to $\ol{M}_0$.
When $M$ is Riemannian, this doesn't significantly affect things.
However, when $M$ is Lorentzian, this changes the direction of time.

If $U_M: \Hc_{M_0} \to \Hc_{M_1}$ is the induced map, then we can think of $U_{\ol{M}}: \Hc_{\ol{M}_1} \to \Hc_{\ol{M}_0}$ as suitably ``adjoint'' to $U_M$.
More precisely, we can write $\Hc_{M_i}$ and $\Hc_{\ol{M}_i}$ as duals, and this connects $U_M$ to $U_{\ol{M}}$.
We will assume that $U_{\ol{M}} = \ol{U_M}$.
Reflection-positivity turns into the assumption that the pairing $\Hc_{M_0} \otimes \Hc_{\ol{M}_0} \to \CC$ is positive (at least in nice cases?).

\subsection{Locality}

Given a point $x \in M$, we can view point operators as being defined on sufficiently small disks $D$ about $x$.
This gives algebras $\Oc_x = \lim_{x \in D} E_D$, where $E$ is the QFT.
In the first lecture here, we discussed how this relates to operadic multiplication.
This is quite similar to the language of operator product expansion used in physics.

\subsection{Cluster decomposition}

Saying that two events are a long time apart is essentially saying that there is a long cylinder between them in spacetime.
This can also be viewed as factorizing the cobordism defining spacetime into a product involving said long cylinder.
Recall that the long cylinder in spacetime corresponds to time evolution $e^{-i H t}$ in the QFT.

\subsection{Causality}

Consider a Lorentzian spacetime $M$ with boundaries $M_0$ and $M_1$.
Placing operations at points $x$ in the interior of $M$ gives operations $\Oc_x \otimes E_{M_0} \to E_{M_1}$.

We assume that our spacetime is globally hyperbolic, so every point in $M_0$ can be connected to a point in $M_1$ via a timelike path.
This splits $M$ into a family of time slices (all diffeomorphic to each other).
Given distinct points $x_i$ on the same time slice (so the $x_i$ are spacelike separated), causality ensures that we can define
\[
	\Oc_{x_1, \dots, x_n} \otimes \check{E}_{M_0} \to \hat{E}_{M_1},
\]
where $\check{-}$ denotes a dense subspace and $\hat{-}$ denotes a completion.
This product is independent of the ordering.
Using global hyperbolicity, we can extend this to any collection of spacelike separated points.

The notion of ``small disk'' makes complete sense in a Riemannian metric, but it seems a bit odd in a Lorentzian metric.
To make sense of this, we recall our notion of complex metric $g: V \to \CC$.
If we pass to the complexification, we get a standard quadratic form $V_\CC \to \CC$.
Thus we may consider $V$ as a subspace of $\CC^d \cong V_\CC$.
The space of allowable complex metrics can be viewed as a contractible subspace of the Grassmannian of real subspaces of $V$.
One can make sense of the algebras $\Oc_x$ from this perspective.

\subsection{Scaling}

Let's return to the originally proposed topic of this lecture.
Many things we'd like to prove about QFT can't be proved from the axioms we've given.

Viewing our spacetime $M$ as a cobordism from $M_0$ to $M_1$, we should view a scalar field as a compactly-supported density on $M$.
Such a scalar field should give an operator $E_{M_0} \to E_{M_1}$.
This transformation $\Dens_c(M) \to \Hom(E_{M_0}, E_{M_1})$ can be called a \emph{Wightman field}.
It's not clear how to relate this to the bundle of algebras $\Oc$ we've been discussing earlier.

We also don't really know what to do about Lagrangians.
From our definition, we can make sense of a moduli space of theories, and we get a renormalization group flow on this moduli space (by rescaling metrics).
Reconciling our notion of fields (using this space) should help us make sense of Lagrangians.
(It's not clear how global symmetries fit into this description.)

We should be able to fix these with a scaling axiom (defined in Riemannian signature).
This should be something saying that our spaces of local operators $E_{(r)}$ (defined on disks of radius $r$) tend to a limit $E_{(0)}$.
This would provide a filtration on the spaces $\Oc_x$ with associated graded $E_{(0)}$.
Using this should allow us to relate $\Oc_x$ with densities.

Extending our theories to codimension 2 should also help us prove things.

\section{6/12 -- More Desired Properties of QFT; Connection to Homotopy Theory}

\subsection{More on scaling}

The notion of scaling seems quite common in physical treatments of QFT but is largely absent from mathematical treatments of the same.

Segal began by studying 2d CFT to understand T-duality.
One version of this identifies a sigma-model with torus target $T$ and another sigma-model with dual torus target $T^*$.
This isomorphism isn't compatible with scaling: it interchanges short-distance and long-distance aspects of the theories.

\subsection{Locality}

Suppose that our theory sends a $(d-1)$-manifold $M$ to a (topological) vector space $E_M$.
To say that our theory is local, we'd like to say that $E_M$ can be constructed ``locally on $M$.''
More precisely, if $M = M_1 \cup_{\Sigma^{d-2}} M_2$, we'd like to construct $E_M$ from $E_{M_1}$, $E_{M_2}$, and ``$E_{\Sigma^{d-2}}$.''
A na\"ive guess, ignoring the unknown $E_{\Sigma}$, fails: we typically have $E_{M_1} \otimes E_{M_2} \not\cong E_M$.
To understand this further, let's consider an example.

\begin{ex}
	Let $G$ be a compact Lie group and $\Lc G$ the loop group of $G$.
	Assume $E_M$ is an irrep of $\Lc G$.
	If we think of $M$ as an interval with glued endpoints, then considering paths from any subinterval to $G$ gives rise to a dense subspace of $E_M$.
	It's not clear how to combine such subspaces to get $E_M$.
\end{ex}

In our case $M = M_1 \cup_{\Sigma^{d-2}} M_2$, we can construct $E_M$ as ``the space of operators on fields on all of $M$.''
The algebras of local operators $\Ac_{M_1}$ and $\Ac_{M_2}$ acts on $E_M$.
If we enlarge $M_1$ and $M_2$ slightly past $\Sigma$ (to get $M_1^+, M_2^+ \subset M$), we can get good results.
Specifically, if $\Ac_{M_1}^{\mathrm{comm}}$ is the commutant of $\Ac_{M_1}$ acting on $E_M$, then $\Ac_{M_1}^{\mathrm{comm}}$ contains $\Ac_{M_2}$ and $\Ac_{M_1^+}$.
The Tomita-Takesaki theorem tells us that $\Ac_{M_1}^{\mathrm{comm}}$ is the von Neumann algebra closure of $\Ac_{M_1}$ (?).
Combining this with the action of $\Ac_{M_2}$ allows us to recover $E_M$ from the behavior of the field theory $M_1$ and $M_2$.
Here $\Sigma$ gives us a bimodule relating $\Ac_{M_1}$ and $\Ac_{M_2}$.

\begin{ex}
	Returning to loop groups, suppose we decompose the interval $I = I_1 \cup_{\pt} I_2$.
	Consider the algebras of loops supported on each subinterval.
	To glue these together, we need to be able to extend these slightly across $\pt$.
	This corresponds to a bimodule relating the algebras.
	In functional analysis, this corresponds to the ``Tomita-Takesaki flow.''
	This is related to the action of Poincar\'e boosts in special relativity.
\end{ex}

\subsection{Non-commutative geometry}

We can reconstruct a manifold $M$ from the behavior of modules over $\Cinfty(M)$.
More generally, in non-commutative geometry, we study the category of modules over an algebra $A$.
We say that Morita-equivalent algebras have the same geometry.

For example, we can study the $K$-theory of $A$.
When $A = \Cinfty(M)$, the Chern character isomorphism lets us write
\[
	K(A) \otimes \QQ \cong \oplus_i H^{2i}(M; \QQ).
\]
The grading here relies on the existence of Adams operations (based on $E \mapsto \wedge^n E$, at least for line bundles).

In the non-commutative context, we don't have Adams operations.
Instead, there's a formal group law, and we get lots of strangeness governed by chromatic homotopy theory.
From a QFT perspective, the grading relies on some sort of scaling.

Rings of $\Cinfty$ functions also experience a sort of Bott periodicity.
Consider the Schwartz space $\Cinfty_S(\RR^2)$ of functions decaying sufficiently rapidly at infinity.
We can construct a non-commutative deformation $\Cinfty_S(\RR^2)_{(h)}$, which is isomorphic (via Fourier transformation) to the algebra of ``smoothing operators'' from $\Cinfty(\RR)$ to itself.
By ``smoothing operator,'' we mean a smooth function $K(x, y)$, viewed as an integral transform
\[
	(Kf)(x) = \int f(y) K(x, y) dy.
\]
This algebra is Morita equivalent to $\CC$ (?), and deforming again recovers $\Cinfty_S(\RR^2)$ (?).

More generally, we can relate the geometry of $X$ with a compactified geometry of $X \times \RR^2$.
This ends up producing Floer homotopy types.

\subsection{The smooth homotopy category}

Long-distance behavior in QFT looks somewhat like homotopy theory.
Short-distance behavior in QFT is a bit more open to debate.
Many people like to think about lattices.
However, Segal prefers to think about the short-distance behavior in terms of (rings of functions of) manifolds.\footnote{This may be more a matter of human taste in describing the universe than a fundamental fact of nature.}

We can ask how to represent QFTs geometrically.
Classically, Brown's representability theorem lets us view homotopy theory as captured by contravariant functors from the homotopy category of finite CW complexes to sets.
The ``smooth homotopy category'' replaces sets here by manifolds (and interprets everything in terms of suitable model categories).

\begin{ex}
	Given a Lie algebra $\gfr$, we can view the smooth homotopy type of $\gfr$ as given by sending $M$ to the moduli space of flat $\gfr$-connections on the trivial bundle on $M$.
	Similarly, given a Lie group $G$, we can view the smooth homotopy type of $G$ as given by sending $M$ to the moduli space of flat $G$-bundles on $M$.
\end{ex}

There's a paper of Kapranov which interprets the Lie algebra of a based loop group $\Omega X$ in this context.
Curvature is captured by maps $\Omega X \to G$.

Classical homotopy theory is ``perturbative:'' it is built one step at a time from sets (by adding homotopies, higher homotopies, etc.).
Analysis is also ``perturbative:'' we take higher derivatives one at a time.
However, the theories are ``perturbative'' in different dirrections: homotopy theory gives the ``long-distance perturbative theory'' while analysis gives ``short-distance perturbative theory.''
