\chapter{Graeme Segal -- A Perspective on Quantum Field Theory}

\section{6/10 -- Overview of QFT}

\subsection{From classical to quantum}

In classical mechanics, one focuses one's attention on a fixed space $M_0$ and obtains a configuration space $X$ from that.
Dynamics are specified by giving a Lagrangian $L: TX \to \RR$, which we assume is positive, inhomogeneous, and quadratic on tangent vectors.
Under these hypotheses, the Lagrangian can be viewed as a generalized Riemannian metric, so that time evolution is given by a (generalized) ``geodesic flow.''
This gives rise to a symplectic form $\omega$ and Poisson bracket $\{-,-\}$ on $T^* X$.
Furthermore, from the Lagrangian $L$, we may extract a Hamiltonian $H: T^*X \to \RR$ such that
\[
	\frac{d}{dt} f = \{ H, f \} \hspace{1em} \forall f \in \Cinfty(X).
\]
Thus we may view a classical dynamical system as a triple $(Y, \omega, H)$ with $(Y, \omega)$ a symplectic manifold and $H: Y \to \RR$ a function.

Quantum mechanics can be viewed as a complex noncommutative analogue of this.
We replace $(Y, \omega, H)$ by a triple $(\Ac, \star, H)$ with $\Ac$ a (topological) $\CC$-algebra, $\star$ an antilinear involution on $\Ac$, and $H \in \Ac$.
For an observable $f \in \Ac$, we require
\[
	\frac{d}{dt} f = i h [H, f].
\]
If the quantum system $(\Ac, \star, H)$ is a ``quantization'' of a classical system $(Y, \omega, H)$, then there is a deep relationship between the two.
For example, we should view $\Ac$ as a completion / extension of $\Cinfty(Y)$, and we can relate the values of an observable to the eigenvalues of the corresponding operator.

The above picture privileges the time dimension, so it is fundamentally non-relativistic.
Working relativistically requires us to move from finite-dimensional spaces to infinite-dimensional spaces.
Heuristically, we take $Y$ to be our space of fields on spacetime (typically $M = M_0 \times \RR$), and we fix a Lagrangian $L: Y \to \RR$.
However, $\Ac$ should now be viewed as a space of functions on an ``almost finite-dimensional manifold.''
This requires many corrections! 

\begin{ex}
	Consider scalar field theory, with $M = M_0 \times \RR$ and $Y = \Cinfty(M)$.
	Then $\Ac$ looks like functions on a stratified configuration space
	\[
		\coprod_{n \geq 0} \Conf_n(M_0),
	\]
	where $\Conf_n(M_0)$ is the space of sets of $n$ distinct unordered points on $M_0$.
\end{ex}

\subsection{Approaching quantum field theory}

Heuristically, we should ``spread things out over spacetime'' by attaching an algebra $\Oc_x$ to every point $x \in M$ and letting
\[
	\Ac \approx \bigotimes_{x \in M} \Oc_x.
\]
This has been formalized in the approach of algebraic quantum field theory -- see e.g.\ the definition of Haag, which assigns an algebra $\Ac_U$ to each open set $U \subset M$.

An alternative approach, which we will pursue, privileges the time dimension.
In this approach, we define a $d$-dimensional quantum field theory as a symmetric monoidal functor
\[
	E: \Cob_{d-1} \to \Vect.
\]
Here $\Cob_{d-1}$ is the symmetric monoidal category with:
\begin{itemize}
	\item Objects: $(d - 1)$-manifolds, often assumed compact or compact with boundary (and thought of as time slices)
	\item Morphisms $M_0 \to M_1$: $d$-dimensional cobordisms from $M_0$ to $M_1$ (thought of as controlling time evolution)
	\item Monoidal structure: disjoint union of manifolds / cobordisms
\end{itemize}
and $\Vect$ is the symmetric monoidal category of (possibly infinite-dimensional) topological vector spaces (with $\otimes$ as tensor product).

There are many other ways to view quantum field theory.
For example, one could think of QFT as a continuum limit of lattice models.

QFT is traditionally seen as describing ``the world except for gravity.''
That is, we think of spacetime as fixed, without allowing for the effects of gravity.
However, QFT can tell us about what sorts of gravitational effects are possible.

The perspective on QFT that we will follow has the advantage that it allows for direct comparisons between QFTs.
Thus, in principle, we may consider a ``moduli space'' of QFTs (of a given type).
We can think of gravitational effects as acting upon this moduli space.

\subsection{The perspective of Connes}

We'd like to use geometry to understand the noncommutative world of QFT.
This can be accessed by studying the spectrum of our Hamiltonian $H$.
Operators which evolve slowly in time ``nearly commute'' with $H$ and thus must be ``nearly diagonal.''
We're mostly interested in studying such operators (as the time scale of humans is much slower than that of the universe).
Thus, when studying QFT, we're interested in algebras which are ``nearly commutative.''

Connes was interested in finding a mildly noncommutative generalization of algebras of functions on a manifold $M$.
Recall that we can define the Clifford algebra $\Cliff(T_m M)$ as the algebra with generators $\{ \gamma_\xi \}_{\xi \in T_m M}$ and relations $\gamma_\xi^2 = -\norm{\xi}^2$.
Let $\Bc$ be a bundle of finite-dimensional algebras on $M$ containing $\Cliff(T_m M)$, and take a connection on $\Bc$.
Let $\Dsl$ be the Dirac operator
\[
	\Dsl = \sum_i \gamma_{\xi_i} \otimes \frac{\partial}{\partial x_i},
\]
and let $H = \Dsl^2$.
Connes gave a formula for $\tr \Dsl^2$ and extracted interesting physical objects, e.g.\ the Higgs field, from $\Dsl$.
From Connes' perspective, the Dirac operator is ``as good as'' the metric.

\subsection{Algebra from functorial QFT}

Suppose we view spacetime $M$ as a cobordism from $M_0$ to $M_1$.
Fix a $d$-dimensional quantum field theory $E$, where $E = \dim M$.
For $x \in M$, we can define a vector space of operators $\Oc_x$ by taking a small disk $D_x$ about $x$ and letting $\Oc_x = E_{\partial D_x}$.
We can view a punctured copy of $M$ as giving a cobordism
\[
	M \sqcup \left(\sqcup_i \partial D_{x_i} \right) \to M_1,
\]
so that applying $E$ gives $E_{M_0} \otimes (\otimes_i \Oc_{x_i}) \to E_{M_1}$.
This gives a sort of ``higher multiplication'' on the spaces of operators $\Oc_{x_i}$.

We can make this much more precise in special cases.
For example, if $E$ is topological, then $\Oc_{x_i} = \Oc$ is independent of $x_i$.
Taking $M_0 = \emptyset$ and $M_1 = S^{d-1}$, we obtain a family of multiplications $\Oc^{\otimes n} \to \Oc$.

\begin{ex}
	If $d = 2$, the cap / cup / pair-of-pants cobordisms equip $A = \Oc$ with the structure of a finite-dimensional commutative Frobenius algebra.
	This means that $A$ is a unital algebra with a nondegenerate trace $\theta: A \to \CC$.
\end{ex}

It is interesting to extend this picture to non-topological $2$-dimensional QFTs.
Here, one can set up a moduli space of QFTs and obtain a gravitational flow.
This was a major historical motivation for string theory.

The second lecture will discuss the importance of positive energy to this theory.
The third lecture will discuss scaling, and the fourth lecture will focus on finding a suitable definition for $\Cob$.

\section{6/10 -- Positive Energy}

\subsection{Warmup}

Consider a quantum mechanical system with Hilbert space $\Hc$.
The Hamiltonian is a self-adjoint operator $H: \Hc \to \Hc$, and we can integrate $iH$ to get unitary operators $U_t = e^{i H t}: \Hc \to \Hc$ describing the time evolution of the system.

Assuming $H \geq 0$ has strong consequences for the theory.
In this case, $U_{-}: \RR \to \End(\Hc)$ is the boundary value of the function on the complex upper half-plane also given by $t \mapsto e^{i H t}$.

To fit this into QFT, consider $\CC$ as a bundle over $\RR$, where the inner product on the base is Lorentzian and the inner product on the fiber is Riemannian.
Think of $\RR$ as our spacetime $M$ and the fiber as $V = T_x M$.

We will introduce the notion of a \emph{complex metric} on a real vector space $V$.
This is a quadratic form $V \to \CC$ satisfying a certain condition to be determined later.

\subsection{Historical digression}

The notion of a complex metric was introduced by Kontsevich and Segal.
Segal was originally interested in loop groups $\Lc G = \Map(S^1, G)$.
There is a nice class of ``positive energy'' representations of loop groups.
These extend to representations of $\Diff^+(S^1) \ltimes \Lc G$ which are also ``positive energy'' in a certain sense.
This notion effectively means that the actions respect the grading $\Hc = \oplus_{k \geq 0} \Hc_k$ coming from an $S^1$-action.

An even simpler illustrative case is that of ``discrete series'' representations of $\PSL_2(\RR)$.
Embed $\PSL_2(\RR) \hookrightarrow \PSL_2(\CC)$.
Let $\PSL_2^<(\CC)$ be the subsemigroup of M\"obius transformations sending the unit disk $D$ to a proper subdisk of itself.
Then $\PSL_2^<(\CC)$ acts by contraction on the discrete series representations.

There is a similar story for loop groups, if we consider the semigroup of holomorphic $f: D \to D$ with $f(D) \subset \mathring{D}$.
The Kontsevich-Segal definition connects to this somehow (I didn't quite catch how).

\subsection{Complex metrics}

\begin{dfn}
	A \emph{complex metric} on a real vector space $V$ is a quadratic form $g: V \to \CC$ such that, if $\lambda_k$ are the eigenvalues of the matrix corresponding to $g$ and $\lambda_k = e^{i \theta_k} \abs{\lambda_k}$, then $\sum_i \abs{\theta_i} < \pi$.
\end{dfn}

We can equivalently state this as
\[
	-\pi \leq \sum_i \pm \theta_i \leq \pi.
\]
One may view the $\theta_i$ as weights of the corresponding spin representation.

For $v \in V$, we have $g(v) \in \CC \setminus (-\infty, 0)$, so $g(v)$ has a canonical square root.
Thus we may assign a ``complex length'' in the positive half-plane to every $v \in V$.

Given an inner product $V \times V \to \CC$, we get an inner product on $\wedge^k V$ with squared norm given by $\alpha \wedge (\star \alpha)$.
Really, this is better thought of as a map $\wedge^k V \times \wedge^k V \to \wedge^d V$, for $d = \dim V$.

We'll change topics now.

\subsection{Application to QFT}

If we want to view a QFT as a functor $E$ defined on a category of cobordisms, we should require:
\begin{itemize}
	\item Cobordisms should be equipped with complex metrics.
	\item The induced maps $U_M : E_{M_0} \to E_{M_1}$ should be holomorphic in $M$ (viewed as living in some complex analytic moduli space of complex metrics).
	\item The operators $U_M$ should be \emph{trace class}: there should be bases $\xi_i$ for $E_{M_0}$ and $\eta_i$ for $E_{M_1}$ with $\xi_i \mapsto \lambda_i \eta_i$ and $\sum_i \abs{\lambda_i} < \infty$.
\end{itemize}

The case of Lorentzian metrics occurs ``on the boundary'' of the moduli space of complex metrics.
Thus usual Lorentzian QFTs can be viewed as holomorphic degenerations of nice QFTs.

Some simpler examples are similar in principle:
\begin{itemize}
	\item $\Diff^+(S^1)$ can be viewed as the boundary of the semigroup of contraction mappings.
	\item $\U(n) \subset \GL_n(\CC)$ can be viewed as the boundary of the semigroup of contraction operators $\CC^n \to \CC^n$.
\end{itemize}

However, $\PSL_2(\RR)$ is not compact, so we can't view it as a boundary.

Viewing our spacetime $M$ as Lorentzian, we can ask for $M$ to be ``globally hyperbolic,'' so every point in $M_0$ can be connected to a point in $M_1$ via a timelike curve.
This is similar to requiring that there are no black holes.
The results of Kontsevich-Segal are best in the case of globally hyperbolic metrics.

Heuristically, we are requiring the existence of a Dirac operator and requiring that this Dirac operator is a contraction operator.
