\chapter{Graeme Segal -- A Perspective on Quantum Field Theory}

\section{6/10 -- Overview of QFT}

\subsection{From classical to quantum}

In classical physics, one starts assuming that one knows what space and time are:
There is a fixed $3$-dimensional spatial manifold $M_0$, and time is a copy of the real
line. 
From this, one then obtains a configuration space $X$, which depends on the details of the
system one is studying.

Dynamics are specified by giving a Lagrangian
\begin{equation*}
L \colon TX \to \RR
\end{equation*}
which we assume is positive, inhomogeneous, and quadratic on tangent vectors.
The Lagrangian endows $T^*X$ with the structure of a symplectic manifold, which defines
a Poisson bracket $\left\{-,-\right\}$ on $C^\infty\left(T^* X\right)$. 

The Lagrangian can be viewed as a generalized Riemannian metric so that 
time evolution is given by a (generalized) ``geodesic flow.''
Another way of saying this is that $L$ 
determines a closely related function $H\colon T^*X \to \RR$, called the Hamiltonian, which
governs the time evolution of the system according to
\begin{equation*}
\frac{d}{dt} f = \{ H, f \} \hspace{3em} \forall f \in \Cinfty \left(T^* X\right) \ .
\end{equation*}
Thus we may view a classical dynamical system as a triple $(Y, \omega, H)$ with $(Y,
\omega)$ a symplectic manifold and $H\in C^\infty\left(Y\right)$ a function.

Quantum mechanics can be viewed as a complex noncommutative analogue of this:
We replace $(Y, \omega, H)$ by a triple $(\Ac, \star, H)$ with $\Ac$ a (topological)
$\CC$-algebra, $\star$ an antilinear involution on $\Ac$, and $H \in \Ac$.
For an observable $f \in \Ac$, an analogous equation governs the time evolution:
\begin{equation*}
\frac{d}{dt} f = i \hslash \left[H , f\right] \ .
\end{equation*}

One should think that when we pass from classical to quantum theory, 
the commutative algebra $C^\infty\left(Y\right)$ is
replaced by the associative algebra $\Ac$.
If the algebra $\Ac$ turns out to in fact be commutative itself, then of course the
bracket would always vanish, and an antilinear involution is just a real structure on
$\Ac$. So if $\Ac$ is commutative, the data of the associated quantum system is just a
real commutative algebra. Real commutative algebras are equivalent to certain spaces, e.g. by
taking the spectrum, and in some cases even manifolds.

Conversely, one should imagine that introducing the $*$-structure is passing from real
algebras to complex algebras. 
But at the same time, we introduce some noncommutativity.
If the quantum system $\left(\Ac , \star , H\right)$ is a ``quantization'' of a classical
system $(Y, \omega, H)$ in this sense, then there is a deep relationship between the two:
$\Ac$ is meant to be a deformation of functions on $Y$, and the relationship between the
two was meant to have all kinds of properties which can be described geometrically.
For example, given a function $f$ on $Y$ (classical observable), the volume\footnote{The
exterior power of the symplectic form defines a volume form on $Y$.}
of various subspaces of $Y$ (defined in terms of $f$) are meant to relate to eigenvalues
of the corresponding quantum observable in $\Ac$.

\subsection{Relativity and infinite-dimensional space}

The systems people had in mind, when quantum theory was first studied, were
systems with finitely many degrees of freedom. 
At the same time, everything so far is intrinsically nonrelativistic, because it singles
out the role of time, and working relativistically requires us to move from finite-dimensional spaces to
infinite-dimensional spaces:
Even the system consisting of two particles moving about somehow trying to attract one
another has infinitely many degrees of freedom. 
The point being that, because forces are not exchanged instantaneously but rather at the
speed of light, it is not good enough to simply know where the particle is at the relevant
time. One also needs to know all that the particle has been doing for a little bit of time
in the past. 
So there are no finite-dimensional relativistic systems. 
From a more modern point of view, this is the fact that the Poincar\'e group has no
finite-dimensional representations. 

Some of the field-theoretic framework discussed above breaks down in the
infinite-dimensional setting. 
For example the study of vector fields, differential
equations, etc. in classical mechanics does not carry over naively to infinite-dimensional
manifolds.

The original point of view on quantum field theory was that we should still start with a
classical system, for instance described by some kind of functions on spacetime, and then
quantize this to produce examples of QFTs.
The first issue is, that if we took this picture seriously, then the mathematics of
quantum field theory would be totally  different than what it is:
If one thinks of the field as a water wave or something of the sort, then one would have
an infinite-dimensional manifold of configurations with much more difficult dynamical
questions concerning analysis and PDEs than one actually finds in quantum systems. 

What one actually finds when one quantizes quantum fields is the following. 
Heuristically, we take $Y$ to be our space of fields on spacetime (typically $M = M_0
\times \RR$), and we fix a Lagrangian $L\colon T Y \to \RR$.
One might imagine that $\Ac$ should be functions on $Y$, but in fact it is functions on
something very different, which one should think of as an  ``almost finite-dimensional
manifold''  since it can be described as consisting of infinitely expanding finite pieces
as in the following example.

\begin{ex}
Consider scalar field theory: spacetime is $M = M_0 \times \RR$, and the space of fields is
$Y = C^\infty\left(M\right)$. 
Then $\Ac$ looks like functions on a stratified configuration space
\begin{equation*}
\coprod_{n \geq 0} \Conf_n(M_0),
\end{equation*}
where $\Conf_n(M_0)$ is the space of sets of $n$ distinct unordered points on $M_0$.
\end{ex}

% Lecture 1 - 24 mins

\subsection{Approaching quantum field theory}

Heuristically, one should ``spread things out over spacetime''. 
For instance, at the level of the algebra, one might have a decomposition,
\begin{equation*}
	\Ac \sim \bigotimes_{x \in M} \Oc_x \ ,
\end{equation*}
where $\Oc_x$ is an algebra assigned to a point $x\in M$.
One instance where this has been formalized is the
approach of algebraic quantum field theory -- see e.g. work of Haag \cite{HK} -- which
assigns an algebra $\Ac_U$ to each open set $U \subset M$. 
A related framework is that of ``factorization algebras'' as in 
\cite{CG1,CG2}.

An alternative approach, which we will pursue, privileges the time dimension.
In this approach, we define a $d$-dimensional quantum field theory as a symmetric monoidal
functor
\begin{equation*}
	E \colon \Cob_{d,d-1} \to \Vect \ .
\end{equation*}
Here $\Cob_{d,d-1}$ is the symmetric monoidal category with:
\begin{itemize}
	\item Objects are $\left(d-1\right)$-manifolds\footnote{All manifolds are assumed to be
    oriented. Sometimes objects of the cobordism category
    are assumed to be compact or compact with boundary.}
	\item Morphisms $M_0 \to M_1$ are cobordisms from $M_0$ to $M_1$ (i.e. $d$-dimensional
    manifolds with boundary given by the disjoint union of $M_0$ and $M_1$)
	\item Monoidal structure is disjoint union of manifolds / cobordisms
\end{itemize}
and $\Vect$ is the symmetric monoidal category of (possibly infinite-dimensional)
topological vector spaces (with $\otimes$ as tensor product).

\begin{rmk}
Heuristically, one should think of the image of a $\left(d-1\right)$-manifold (thought of
as a time slice) under $E$ as the Hilbert space at that moment in time. 
The image of a cobordism between two time slices should be thought of as the linear map
given by time evolution of a state in the Hilbert space. 
\end{rmk}

Quantum field theory is a very diverse subject. 
One can be interested in it from the point of view of fundamental physics, which is the
perspective generally taken in these notes. 
But there are many other areas where quantum field theory intervenes. 
For instance the study of some continuum limit of a lattice model in statistical
mechanics.

In classical physics, space and time are fixed notions.
QFT is traditionally seen as describing ``the world except for gravity,''
and gravity concerns the dynamics of spacetime itself.
This might lead one to believe that, as in classical physics, the notion of spacetime is
fixed in advance.
One of the points of these notes is that this is not quite true:
The constructions are not blind to the type of spacetime you feed in.

In fact, QFT can actually inform us about the nature of spacetime as follows. 
If one were trying to include gravity into this framework, one would
need to incorporate the moduli space of all possible spacetimes. 
The perspective on QFT that we will follow has the advantage that it allows one to
identify when two QFTs are the `same'.
Thus, in principle, one can consider a ``moduli space'' of QFTs of a given type.
However, the nature of these moduli spaces of theories depends on the specific
definition of the category $\Cob_{d,d-1}$, i.e. on the notion of spacetime that we are
working with. 
The upshot is that now there is a well-defined question:
What is a notion of `spacetime' for which these two moduli spaces (of spacetimes vs. of
theories) are of the same type, or at least can be incorporated into the same framework?

\subsection{The perspective of Connes}

% Lecture 1 - 40 mins

The states of a quantum system with algebra $\Ac$ are (positive, normalized) linear forms
$\Ac \to \CC$.
If a state is actually a ring homomorphism, then it induces a map:
\begin{equation*}
\Spec \CC = \pt \to \Spec \Ac \ ,
\end{equation*}
i.e. classical states are points of the spectrum.
The geometric properties of a commutative algebra are well-known: it essentially is the
algebra of functions on its spectrum, i.e. space of classical states. 
But what is geometric about noncommutative algebras?

Before discussing this, we first note that we will not be considering arbitrary
noncommutative algebras but rather ones which are ``nearly commutative''.
As human beings, we can only see things which move very slowly:
We cannot even see a bullet, which itself is moving quite slowly with respect to the
scale of quantum physics. 
No matter how one interprets the elements of our algebra, what is certain is that their
rate of change is governed by whether or not they commute with the
distinguished positive self-adjoint element $H\in \Ac$, the Hamiltonian.
In other words, things which are sufficiently slow can be thought of as ``nearly
stationary'', which means that they ``nearly commute'' with $H$, and therefore must be 
``nearly diagonal.''
It is important to hang onto the fact that, when studying QFT, one is interested in
algebras which are ``nearly commutative''.

Connes was interested in finding a mildly noncommutative generalization of algebras of
functions on a manifold $M$.
Recall that we can define the Clifford algebra $\Cliff\left(T_m M\right)$ to be generated
by 
\begin{equation*}
\left\{\gamma_\xi\right\}_{\xi\in T_m M} 
\quad\text{with relations}\quad
\gamma_\xi^2 = -\norm{\xi}^2 \ .
\end{equation*}
Let $\Bc$ be a bundle of finite-dimensional algebras on $M$ containing 
$\Cliff\left(T_m M\right)$ and fix a connection on $\Bc$.
Let $\Dsl$ be the Dirac operator
\begin{equation*}
\Dsl = \sum_i \gamma_{\xi_i} \otimes \frac{\partial}{\partial x_i} \ ,
\end{equation*}
and set the Hamiltonian to be $H = \Dsl^2$.

Connes produced a version of the standard model in this framework.
He gave a formula for $\tr \Dsl^2$ which, for suitable choices of $\Bc$, 
produced similar terms to the standard model Lagrangian.
For instance one finds the ordinary Yang-Mills term for a connection and 
the natural Einstein gravitational action.
The really impressive thing was that when you allow the algebra to be a direct sum of
multiple finite-dimensional algebras (so it wasn't simple, but had different components)
then one discovers a Higgs field.
From Connes' perspective, the Dirac operator is ``as good as'' the metric.

\subsection{Algebra from functorial QFT}

Suppose we view spacetime $M$ as a cobordism from $M_0$ to $M_1$.
Fix a $d$-dimensional quantum field theory $E$, where $E = \dim M$.
For $x \in M$, we can define a vector space of operators $\Oc_x$ by taking a small disk
$D_x$ about $x$ and letting $\Oc_x = E_{\partial D_x}$.
We can view a punctured copy of $M$ as giving a cobordism 
\begin{equation*}
M \sqcup \left(\sqcup_i \partial D_{x_i} \right) \to M_1 \ ,
\end{equation*}
so that applying $E$ gives a linear map:
\begin{equation*}
E_{M_0} \otimes \left(\otimes_i \Oc_{x_i}\right) \to E_{M_1} \ .
\end{equation*}
This gives a sort of ``higher multiplication'' on the spaces of operators $\Oc_{x_i}$.

We can make this much more precise in special cases.
For example, if $E$ is topological, then $\Oc_{x_i} = \Oc$ is independent of $x_i$.
Taking $M_0 = \emptyset$ and $M_1 = S^{d-1}$, we obtain a family of multiplications
$\Oc^{\otimes n} \to \Oc$.

\begin{ex}
If $d = 2$, the cap / cup / pair-of-pants cobordisms equip $A = \Oc$ with the structure of
a finite-dimensional commutative Frobenius algebra.
This means that $A$ is a unital algebra with a nondegenerate trace $\theta \colon A \to \CC$.
The pair of pants gives rise to the multiplication, and then 
cap and cup give rise to the inclusion of the unit and trace respectively.
\end{ex}

It is interesting to extend this picture to non-topological $2$-dimensional QFTs.
In this setting one can set up a moduli space of $2$-dimensional QFTs, which looks very
much like the moduli space of Riemannian manifolds.
If the bordisms come with a Riemannian metric, then one obtains a flow on this moduli
space, given by rescaling the metric. 
It was then discovered that this scaling in fact gives the gravitational flow.
This was a major historical motivation for string theory.

% The second lecture will discuss the importance of positive energy to this theory.
% The third lecture will discuss scaling, and the fourth lecture will focus on finding a
% suitable definition for $\Cob$.

\section{6/10 -- Positivity of Energy}

\subsection{Finite-dimensional quantum mechanics}

The behavior of classical systems and quantum systems is of course very different.
If one tracks down why this is, it has everything to do with positivity of energy.

Consider a quantum mechanical system with Hilbert space $\Hc$.
The Hamiltonian is a self-adjoint operator $H \colon \Hc \to \Hc$, and we can integrate $iH$ to
get a $1$-parameter unitary group of 
operators $U_t = e^{i H t} \colon \Hc \to \Hc$ describing the time evolution of the system.

Assuming that the Hamiltonian is positive
(i.e. that $\Spec H \subset [0,\infty ) \subset \RR$)
seems quite innocuous, but it has strong consequences for the theory.
This essentially implies that ``nothing can ever happen for the first time,''
which was first noticed by Fermi as early as 1932 \cite{F}.

In more detail:
the $1$-parameter unitary group $U_{-} \colon \RR \to \End(\Hc)$ is the boundary value of the
function on the complex upper half-plane given by $t \mapsto e^{i H t}$.
The boundary value of a bounded holomorphic function in the upper half-plane must vanish
identically if it vanishes on an open interval of the real axis.
Therefore, taking positivity of energy literally, 
a state $\zeta \in \Hc$ for which $U_t\left(\zeta\right)$ remains in some closed subspace
$\Hc_0 \subset \Hc$ for all $t < 0$ must remain in this subspace for all $t$.
Note that on the boundary (the real axis) we have unitary operators, 
and within the upper half space on the positive imaginary axis we have 
contraction operators.

From a physical perspective, one might imagine that the cobordisms involved in functorial
QFT are equipped with Lorentzian metrics. 
One should imagine that we have Lorentzian metrics 
(represented by unitary operators on some Hilbert space)
living along the real axis. Along the positive imaginary axis we have Riemannian metrics
(represented by contraction operators), and we want to ``fill in'' the upper half space
with what are called \emph{complex metrics}, originally introduced by Kontsevich-Segal
\cite{KS}. Just as an ordinary metric on a smooth manifold is a positive definite 
inner-product on each tangent space, a complex metric is a $\CC$-valued quadratic form on
each (real) tangent space, with a certain positivity condition stated in 
\Cref{sec:C_met}.

\subsection{Representations of loop groups}
\label{sec:loop_groups}

One source of motivation for the introduction of the notion of an allowable complex metric 
comes from the theory of loop groups \cite{PS}.
Given some compact Lie group $G$, the \emph{loop group} is the group of smooth maps 
\begin{equation*}
\Lc G = \Map\left(S^1 , G\right)
\end{equation*}
under pointwise multiplication. The general theory of representations of loop groups is
quite complicated, but the theory of so-called ``positive energy'' representations is
analogous to the theory of finite-dimensional Lie groups.
One property of the positive energy representations is a certain functoriality with
respect to $S^1$: 
The group $\Diff^+\left(S^1\right)$ of oriented-preserving diffeomorphisms of the circle
act on $\Lc G$ by automorphisms. One characterization of positive energy representations
is that they extend to representations of $\Diff^+(S^1) \ltimes \Lc G$
which have ``positive energy'' in the following sense:
Given a representation by unitary operators on a Hilbert space $\Hc$, there is an action
of the circle which gives a grading 
\begin{equation*}
\Hc = \oplus_{k \geq 0} \Hc_k
\end{equation*}
The representation has \emph{positive energy} if the action of
$\Diff^+\left(S^1\right)$ (via $\Aut\left(\Lc G\right)$) is compatible with this grading.

This discussion generalizes the simpler discussion of the representation theory of
$G = \PSL_2\left(\RR\right)$. (Note that $\PSL_2\left(\RR\right)$ contains
$\Diff^+\left(S^1\right)$.)
Irreducible unitary representations of this group are essentially all of two kinds: the 
\emph{discrete series} and the \emph{principal series}.
The group $G$ contains a copy of the circle, as rotations of the projective line. 
In the case of a \emph{principal} series representation, rotation of the circle acts with a
spectrum which extends from $-\infty$ to $\infty$.
In the case of \emph{discrete} series representations, rotation of the circle acts 
so that the weights are bounded below, i.e. the discrete series representations have
a positive energy property. 

This means that the discrete series representations extend to bounded holomorphic
representations of a certain semigroup acting by contraction operators.
In particular, $\PSL_2\left(\RR\right)$ lies on the boundary of a complex sub-semigroup 
$\PSL_2^{<}\left(\CC\right) \subset \PSL_2\left(\CC\right)$ 
given by M\"obius transformations of the Riemann sphere sending the unit disk to a proper
subdisk of itself. Then $\PSL_2^<(\CC)$ acts on the discrete series
representations, and it acts via contractions by the maximum modulus principal. 

It was (independently) realized by Kontsevich, Segal, and Neretin in the 80s
\cite{Ner,S:CFT} that there is an infinite-dimensional complex semigroup which has
precisely the same relationship to $\Diff^+\left(S^1\right)$ as 
$\PSL_2^{<}\CC$ has to $\PSL_2 \RR$.
In particular, consider the complex semigroup of holomorphic maps from the disk to itself
which embed the disk in its own interior.
The group $\Diff^+\left(S^1\right)$ lies on the boundary of this semigroup.
On can imagine that these are the maps which don't really shrink the disk but rather just
act a diffeomorphism on the boundary. 

\subsection{Complex metrics}
\label{sec:C_met}

Recall we are interested in imposing a positivity condition on
complex-valued quadratic forms on a finite-dimensional real vector space 
$g \colon V \to \CC$ in order to introduce an appropriate notion of a complex-valued
metric on a real manifold. 
Write $V_C = V\otimes \CC$ for the complexification of $V$.

A first guess might have been to study points in the Siegel `generalized upper half
space'. This is the space of $\CC$-valued quadratic forms with positive-definite real
part. For $\dim_\RR V = n$, this is the space of $n\times n$ complex symmetric
matrices with $\mathop{Re} A$ positive definite.

One way the Siegel upper half space appears is the following.
Suppose we are interested in an improper Gaussian integral of the form:
\begin{equation*}
\int_{\RR^n} \exp\left(
i x^T A x 
\right)
\end{equation*}
for a real $n\times n$ symmetric matrix $A$.
Even in the $1$-dimensional case, the usual strategy is to move the real number into the
upper half plane to make this a Gaussian integral.
One can also do this with a matrix as follows. Begin with a complex symmetric matrix whose imaginary
part is positive definite, i.e. a point of the Siegel upper half space. 
The integral is a proper Gaussian integral for such matrices and converges.
The value of the original integral can then be defined as the limit as the imaginary part
goes to zero. 

\begin{dfn}
A \emph{complex metric} on a real vector space $V$ is a quadratic form
$g \colon V \to \CC$ such that, if $\lambda_k$ are the eigenvalues of the 
matrix corresponding to $g$ and 
$\lambda_k = e^{i \theta_k} \abs{\lambda_k}$, then
\begin{equation}
\sum_i \abs{\theta_i} < \pi \ .
\label{eqn:C_metric}
\end{equation}
\label{dfn:C_metric}
\end{dfn}

\begin{rmk}
We can equivalently state this as
\begin{equation*}
-\pi \leq \sum_i \pm \theta_i \leq \pi \ .
\end{equation*}
One may view the $\theta_i$ as weights of the corresponding spin representation.
\end{rmk}

\begin{ex}
If $g$ happens to be real then each $\abs{\theta_i}$ is either $0$ or $\pi$, and the
condition \eqref{eqn:C_metric} implies that in fact $g$ is positive definite.
Therefore there is a subspace of real positive definite metrics in the space of allowable
complex metrics. Similarly, \eqref{eqn:C_metric} implies that real $g$ on the so-called
\emph{Shilov boundary} of the space of allowable complex metrics can have at most one
$\abs{\theta_i} = \pi$, the rest must vanish. Therefore the real Lorentzian metrics, and
no other nondegenerate metrics, lie on the Shilov boundary of the space of allowable complex metrics. 
\label[ex]{allowable_real}
\end{ex}

This definition turns out to be equivalent to the following characterization. 
One consequence of \eqref{eqn:C_metric}, is that $\max \theta_i - \min \theta_i < \pi$. 
This means that, for all $v \in V$,
we have $g(v) \in \CC \setminus (-\infty, 0)$.
Therefore $g(v)$ has a canonical square root with positive real part. 
Thus we may assign a ``complex length'' in the positive half-plane to every $v \in V$.

Recall that, given the inner product on $V$, we obtain a Hodge star operator $\star$ which
sends $k$-forms, i.e. elements of $\wedge^k V^*$ to $\left(n-k\right)$-forms
(twisted by the orientation bundle)
where $n = \dim_\RR V$.
This allows us to, starting from the inner product $V \times V \to \CC$, define a
quadratic form on $\wedge^k V^*$ for all $k \geq 0$:
\begin{equation*}
\begin{tikzcd}[row sep=tiny]
\wedge^k V^* \ar{r}&
\CC \simeq 
\wedge^n\left(V^*\right)_\CC 
\\
\alpha \ar[mapsto]{r}&
\alpha \wedge \star \alpha
\end{tikzcd}
\end{equation*}

The Hodge star sends the constant function to the volume element:
\begin{equation*}
\star 1 = \vol_g = \left(\det g\right)^{1/2} \abs{ dx^1 \cdots dx^n }
\end{equation*}
We think of $\vol_g$ as an element of the complex line 
$\wedge^n \left(V^*\right)_\CC$, the complexification of the real line of real volume
forms (tensored with the real orientation line).
We say that an element of the real part of the line $\wedge^nV^*_\CC$
is positive if it is a positive volume-element. 

\begin{thm}[\cite{KS}]
A quadratic form $g \colon V \to \CC$ 
satisfies \Cref{dfn:C_metric} if and only if for all $k\geq 0$ the real part of
the quadratic form
\begin{equation*}
\wedge^k\left(V^*\right)  \to \wedge^n\left(V^*\right)_\CC
\end{equation*}
sending $\alpha$ to $\alpha \wedge \star \alpha$ is positive-definite.
\end{thm}

% Lecture 2 - 40 minutes 

\subsection{Representations of bordism categories}

If we want to view a QFT as a functor $E$ defined on a category of cobordisms, we should
require:
\begin{itemize}
\item Cobordisms should be equipped with complex metrics.
\item The induced maps $U_M \colon E_{M_0} \to E_{M_1}$ should be holomorphic in $M$ 
(viewed as living in some complex analytic moduli space of complex metrics).
\item The operators $U_M$ should be \emph{trace class}: 
there should be bases $\xi_i$ for $E_{M_0}$ and $\eta_i$ for $E_{M_1}$ with 
$\xi_i \mapsto \lambda_i \eta_i$ and $\sum_i \abs{\lambda_i} < \infty$.
\end{itemize}

The main result of \cite{KS} establishes what happens on the boundary of the space of
field theories of this sort. This will follow a familiar pattern from representation
theory.

Recall from \Cref{sec:loop_groups}, that $\Diff^+(S^1)$ can be viewed as the
boundary of a certain holomorphic semigroup of contraction operators. 
We saw that this was an infinite-dimensional example of what we saw with the action of the
circle inside of $\PSL_2 \CC$. 
Another familiar instance of this phenomenon is the unitary group $U_n$ in
$\GL_n\left(\CC\right)$:
Consider the semigroup of contraction operators, i.e. $A \colon \CC^n \to \CC^n$ such that
$\norm{A\xi} < \norm{\xi}$. 
These form a ball in $\GL_n\left(\CC\right)$, and the unitary group is the Shilov
boundary. 

The Shilov boundary is the smallest closed subset in which an 
analog of the maximum modulus principle holds:
It has the property that, 
if I am given any holomorphic function on the domain which extends continuously to the
topological boundary, then it obtains its maximum on the Shilov boundary. 
Note that this implies that any holomorphic function on the domain that extends to
the topological boundary at all are determined by their values on the Shilov boundary. 
The field theoretic analogue of this statement is that a Lorentzian field theory is 
completely determined by its Wick rotation to Riemannian spacetime. 

Recall however that, unlike $\U\left(n\right)$, $\PSL_2(\RR)$ is not compact so there are
other things on the boundary outside of $\PSL_2\left(\RR\right)$. 
The entire Shilov boundary of the semigroup of contraction operators is the solid torus
consisting of $\PSL_2\left(\RR\right)$ (the real M\"obius
transformations) compactified by the 2-torus of degenerate M\"obius transformations. 
The statement is that the contraction representation extends to $\PSL_2 \RR$, but perhaps
not the degenerate points on the boundary. 
It was shown in \cite{KS} that this has a direct analogue in the field theoretic situation.

Recall from \Cref{allowable_real} that real Lorentzian metrics, and no other
nondegenerate metrics, lie on the Shilov boundary of the space of allowable complex
metrics. 
Consider Lorentzian cobordisms $M$ between Riemannian manifolds $M_0$ and $M_1$
as our basic notion of spacetime. There is a subclass of these which are ``globally
hyperbolic,'' meaning every point in 
$M_0$ can be connected to a point in $M_1$ via a timelike curve.
This is analogous to requiring that there are no black holes:
If one does not assume that the metric is globally hyperbolic then
one can imagine finding a timelike curve which falls into a 
closed sub trajectory, i.e. it might leave $M_0$ and never reach $M_1$ while remaining
timelike. 

What was shown in \cite{KS} is that theories which are defined for these 
positive metrics have boundary values which are unitary on the part of the Lorentzian
metrics which are globally hyperbolic.
When one drops the globally hyperbolicity, the associated operators stop being unitary,
and in fact become unbounded. 
This is precisely analogous to when one tries to get a 
degenerate M\"obius transformation (i.e. real $2\times 2$ rank $1$ matrix) 
to act on a discrete series representation. 

\section{6/11 -- Desired Properties of QFT}

In his first lecture, Ibou Bah proposed five properties that a (nice) QFT should satisfy:
\begin{enumerate}[label = (\roman*)]
\item \emph{unitarity}, 
\item \emph{locality}, 
\item \emph{causality}, 
\item \emph{renormalization}, and
\item \emph{cluster decomposition}.
\end{enumerate}
We will discuss these properties in the setting of functorial field theory. 

\subsection{Renormalization}

Quantum field theory started off, historically, by thinking of itself as the process of
quantizing a wavelike system:
One had a scalar field, or a vector field, or a bundle with a connection, and then one
is trying to quantize the space of such field configurations. 
One way to study these theories is by defining a generalized Riemannian metric on the
space of these field configurations.
Then the system, classically at least, should follow a geodesic in this space of
configurations as it develops in time. 

The crucial step in the historical development of QFT was the understanding of
the process of renormalization. 
The idea is that only some Lagrangians make good sense: the ones for which one can carry
out the process of renormalization.
The condition that allows one to do this is precisely the condition one encounters in the
calculus of variations.

Fixing a manifold $N$, there is a field theory in any dimension, called the $\sigma$-model
into $N$, given as follows.
Given some spacetime manifold $M$, the fields on $M$ are maps from $M\to N$.
If $M$ and $N$ have Riemannian metrics, we can define an energy functional on the space of
such maps. For instance when $M = \RR$ or $S^1$ we obtain the usual expression for the
energy of a path in a Riemannian manifold. 
Starting with a random smooth curve between points in a Riemannian manifold, there is a
well-behaved Morse theory: given a path between two points, there is a gradient flow to
a geodesic between them. 
When $\dim M = 2$, things are almost well-behaved, except there are famously problems with
bubbling. For $\dim M > 2$ the Morse theory is totally wrong:
For $3$-dimensional $M$, following the gradient flow of some map $M\to N$ will
not lead to some nice e.g. harmonic map at a critical point, but rather $M$ disintegrates
into shreds. This is trying to move backwards along a heat flow.

\subsection{Unitarity}

% Lecture 3 - 10 minutes 

Let $M$ be an oriented cobordism from $M_0$ to $M_1$.
Writing $\ol{M}_i$ for $M_i$ with its orientation reversed, the cobordism $M$ can equally
well be regarded as an oriented cobordism:

If $M$ is an oriented cobordism from $M_0$ to $M_1$, 
it can always be regarded, without changing the orientations or anything, as a cobordism:
\begin{equation*}
M^{\mathrm{rev}} \colon \overline{M}_1 \to \ol{M}_0 \ .
\end{equation*}
Note that we did not change the orientation on $M$, we are just regarding it as a bordism
in the opposite direction. 
Write the corresponding Hilbert spaces and linear maps between them as:
\begin{align*}
U_M \colon \Hc_{M_0} \to \Hc_{M_1}
&&
U_{M^{\mathrm{rev}}} \colon \Hc_{\overline{M}_1} \to
\Hc_{\overline{M}_0}
\end{align*}
The Hilbert spaces $\Hc_{M_i}$ and $\Hc_{\ol{M}_i}$ are canonically dual:
Directly from the axioms, one can define morphisms 
\begin{equation}
\raisebox{25pt}{
\rotatebox{-90}{
\begin{tikzpicture}[xscale=-1,baseline ={(current bounding box.center)}]
\begin{scope}
\clip (-0.01,-2) rectangle (0.5,0.5);
\node[cylinder,draw=black,aspect=0.7,
minimum height=20pt,minimum width=20pt,
shape border rotate=180,cylinder uses custom fill, cylinder body     
fill=gray!60,cylinder end fill=gray!10] (A) {};
\node[cylinder,draw=black,aspect=0.7,
minimum height=20pt,minimum width=20pt,
shape border rotate=180,cylinder uses custom fill, cylinder body     
fill=gray!60,cylinder end fill=gray!10] at (0,-40pt) {};
\end{scope}
\begin{scope}
\clip (0,-2) rectangle (-2,0.5);
\draw[fill,color=gray!60] (0,-20pt) circle (30pt);
\draw[fill,color=white] (0,-20pt) circle (10pt);
\draw[color=black] (0,-20pt) circle (30pt);
\draw[color=black] (0,-20pt) circle (10pt);
\end{scope}
\end{tikzpicture}
}}
\quad \mapsto \quad 
\begin{tikzcd}
\Hc_{M_i} \otimes \Hc_{\ol{M_i}} \arrow{d} \\
\CC
\end{tikzcd}
\quad \text{ and } \quad
\raisebox{-25pt}{
\rotatebox{90}{
\begin{tikzpicture}[xscale=-1,baseline ={(current bounding box.center)}]
\begin{scope}
\clip (-0.01,-2) rectangle (0.5,0.5);
\node[cylinder,draw=black,aspect=0.7,
minimum height=20pt,minimum width=20pt,
shape border rotate=180,cylinder uses custom fill, cylinder body     
fill=gray!60,cylinder end fill=gray!10] (A) {};
\node[cylinder,draw=black,aspect=0.7,
minimum height=20pt,minimum width=20pt,
shape border rotate=180,cylinder uses custom fill, cylinder body     
fill=gray!60,cylinder end fill=gray!10] at (0,-40pt) {};
\end{scope}
\begin{scope}
\clip (0,-2) rectangle (-2,0.5);
\draw[fill,color=gray!60] (0,-20pt) circle (30pt);
\draw[fill,color=white] (0,-20pt) circle (10pt);
\draw[color=black] (0,-20pt) circle (30pt);
\draw[color=black] (0,-20pt) circle (10pt);
\end{scope}
\end{tikzpicture}
}}
\quad \mapsto \quad 
\begin{tikzcd}
\CC
\arrow{d} \\
\Hc_{\ol{M}_i} \otimes \Hc_{M_i} 
\end{tikzcd}
\label{eq:ev_coev}
\end{equation}
such that the Zorro axiom is satisfied: 
\begin{equation*}
\raisebox{-23pt}{
\begin{tikzpicture}[baseline={(current bounding box.center)}]
\begin{scope}
\clip (-0.01,-2) rectangle (0.5,0.5);
\node[cylinder,draw=black,aspect=0.7,
minimum height=20pt,minimum width=20pt,
shape border rotate=0,cylinder uses custom fill, cylinder body     
fill=gray!60,cylinder end fill=gray!10] (A) {};
\node[cylinder,draw=black,aspect=0.7,
minimum height=20pt,minimum width=20pt,
shape border rotate=0,cylinder uses custom fill, cylinder body     
fill=gray!60,cylinder end fill=gray!10] at (0,-40pt) {};
\end{scope}
\begin{scope}
\clip (0,-2) rectangle (-2,0.5);
\draw[fill,color=gray!60] (0,-20pt) circle (30pt);
\draw[fill,color=white] (0,-20pt) circle (10pt);
\draw[color=black] (0,-20pt) circle (30pt);
\draw[color=black] (0,-20pt) circle (10pt);
\end{scope}
\end{tikzpicture}
}
\quad \circ \quad 
\raisebox{23pt}{
\begin{tikzpicture}[xscale=-1,baseline ={(current bounding box.center)}]
\begin{scope}
\clip (-0.01,-2) rectangle (0.5,0.5);
\node[cylinder,draw=black,aspect=0.7,
minimum height=20pt,minimum width=20pt,
shape border rotate=180,cylinder uses custom fill, cylinder body     
fill=gray!60,cylinder end fill=gray!10] (A) {};
\node[cylinder,draw=black,aspect=0.7,
minimum height=20pt,minimum width=20pt,
shape border rotate=180,cylinder uses custom fill, cylinder body     
fill=gray!60,cylinder end fill=gray!10] at (0,-40pt) {};
\end{scope}
\begin{scope}
\clip (0,-2) rectangle (-2,0.5);
\draw[fill,color=gray!60] (0,-20pt) circle (30pt);
\draw[fill,color=white] (0,-20pt) circle (10pt);
\draw[color=black] (0,-20pt) circle (30pt);
\draw[color=black] (0,-20pt) circle (10pt);
\end{scope}
\end{tikzpicture}
}
\simeq\qquad
\begin{tikzpicture}[baseline={(current bounding box.center)}]
  \node[cylinder,draw=black,aspect=0.7,
  minimum height=60pt,minimum width=20pt,
  shape border rotate=0,cylinder uses custom fill, cylinder body     
  fill=gray!60,cylinder end fill=gray!10] (A) {};
\draw[dashed]
    let \p1 = ($ (A.after bottom) - (A.before bottom) $),
        \n1 = {0.5*veclen(\x1,\y1)-\pgflinewidth},
        \p2 = ($ (A.bottom) - (A.after bottom)!.5!(A.before bottom) $),
        \n2 = {veclen(\x2,\y2)-\pgflinewidth}
  in
    ([xshift=-\pgflinewidth] A.before bottom) arc [start angle=270, delta angle=180,
    x radius=\n2, y radius=\n1];
\end{tikzpicture}
\end{equation*}
It also follows directly from the axioms that $U_M$ is always adjoint to $U_{\ol{M}}$
in the algebraic sense.

Continue to consider the oriented cobordism $M$ from $M_0$ to $M_1$.
Now, rather than simply reversing the direction of $M$ as above, we will reverse the
orientation (and take the complex conjugate metric) on $M$ itself, to obtain a cobordism
\begin{equation*}
\overline{M} \colon \overline{M_0} \to \overline{M_1} \ .
\end{equation*}
In the case of a Riemannian metric this will of course leave the metric fixed, 
but in the case of a Lorentzian metric it will appropriately change the
direction of time.
In order to discuss unitary, one should impose the further condition that 
\begin{equation}
U_{\ol{M}} = \ol{U_M} \ .
\label{eq:conj}
\end{equation}
In addition to this, we would like to assume \emph{reflection-positivity}:
A reflection structure induces a hermitian metric on the vector space of states
and positivity is the condition that these hermitian structures be positive definite.

\subsection{Locality}

Let $M$ be a manifold. We are interested in making observations at various points.
Recall that one can associate, to any point $x\in M$, a topological vector space 
$\Oc_x$ of observables:
Consider a collection of balls around (or neighborhoods of) $x$, getting smaller and
smaller, forming an ordered set under inclusion. 
The observables at $x$ then comprise the inverse limit of the 
values of the given TQFT $E$ on the boundaries of these neighborhoods:
\begin{equation*}
\Oc_x = \varprojlim_{D\downarrow x} E_{\partial D}
\end{equation*}
Now consider the situation where several of these points are contained in some larger 
region in $M$ which is fixed.
Locality is then the fact that, there is an expansion of observables in this larger region
in terms of observables which just come from the $\Oc_x$ for various points in the region.
This is quite similar to the language of operator product expansion used in physics. 
Mathematically this has been formalized in the language of 
factorization algebras (algebras over the little-disk operad of $M$) \cite{CG1,CG2}.

\subsection{Cluster decomposition}

The idea of working with manifolds with complex metrics (\Cref{sec:C_met}) 
was to allow one to freely use the concept of Wick rotation: One might as well work with
Riemannian manifolds for most purposes. 

Imagine there are two clusters of points in spacetime $M$ such that the spacetime $M$ can
be factorized into bordisms such that there is a cylindrical region separating them in
spacetime.
The cluster decomposition has to do with allowing this cylindrical bit to get very long. 
Recall that in the Riemannian setting, the operator associated to this cylinder is just
exponentiating a positive operator, at least for a unitary theory. 
So the output of the first cluster gets scaled down to the vacuum vector by this
exponentiated positive operator.
It is then this vacuum vector which is fed into the  second cluster, rather than the
potentially complicated output of the first cluster, making it clear why there is an
associated factorization. 

\subsection{Causality}
\label{sec:causality}

Consider two spacelike separated points in a 
Lorentzian manifold $M$ i.e. there is no smooth timelike curve between them. 
In order to enforce some sort of causality, one would like to formalize the notion that,
in this situation, observables at these points ought not to affect one another. In other
words, the associated operators ought to commute.
If $M$ is a cobordism from $M_0$ to $M_1$ then observables at a point $x$,
$\Oc_x$, acts:
\begin{equation*}
\Oc_x \otimes E_{M_0} \to E_{M_1} 
\end{equation*}
However it doesn't make sense for this to commute with another such $\Oc_y$, since they
are not defined on the same spaces. 

Such a causality property can be formulated on a spacetime which is globally hyperbolic.
This is assumption that every maximally-extended time-like geodesic in $M$ travels from
$M_0$ to $M_1$.
Here we are only considering compact cobordisms, thought these are not the usual focus in
relativity theory.
In this case, being globally hyperbolic is equivalent to the existence of a smooth 
function $t \colon M \to \left[0,1\right]$ with gradient everywhere in the positive
light-cone. Therefore the fibers of $r$ are all Riemannian, and there is in fact a
diffeomorphism $M\simeq M_0 \times \left[0,1\right]$ by following the orthogonal
trajectories of the time slices.
Therefore this assumption allows one to identify the values of the TQFT $E_{M_0} \simeq
E_{M_t}\simeq E_{M_1}$ for all time slices $M_t$. 

Now rotate to a Riemannian metric on $M$.
Given a time slice $M_t$, it is useful to consider the neighboring slices.
Given a small cobordism around $M_t$, the incoming boundary has dense image in $E_{M_t}$.
This is like the way that heat flow always pushes $L^2$ functions onto smooth functions. 
So inside of $E_{M_t}$, we will always have some dense subspace, which we can think of as
like the smooth functions in a Hilbert space.
The outgoing boundary corresponds to the dual of the smooth functions, the distributions. 
This is a completion of $E_{M_t}$.

So we have seen that the assumption of being globally hyperbolic allowed us to identify
the values of the theory on every time slice, and now we have also seen that they come
equipped with a dense subspace and a nice completion. 
Given distinct points $x_i$ on the same time slice (so the $x_i$ are spacelike separated),
we can now form the operator 
\begin{equation*}
\Oc_{x_1, \dots, x_n} \otimes \check{E}_{M_0} \to \hat{E}_{M_1} \ ,
\end{equation*}
where $\check{\left(-\right)}$ denotes a dense subspace and $\hat{\left(-\right)}$ denotes a completion.
One should think of this as something like an operator with a distributional kernel, i.e.
something which only makes sense as an operator after one has 
smoothed it a bit.
On any particular time slice one can insert any number of operators, and this construction
was clearly independent of the ordering of the points. 
Using global hyperbolicity, we can extend this to any collection of spacelike separated
points.

\begin{rmk}
The notion of ``small disk'' makes complete sense in a Riemannian metric, but it seems a
bit odd in a Lorentzian metric.
To make sense of this, we recall our notion of complex metric $g \colon V \to \CC$ on a
real vector space.
If we pass to the complexification, $V_\CC$, we get a quadratic form $V_\CC \to \CC$.
If this is nondegenerate, all such are the same (up to $\Aut\left(V\right)$).
So one might as well choose some coordinates, and think that $V\subset \CC^d$,
where $d = \dim_\RR V$.

One can now ask oneself, given a complex vector space with the standard quadratic form,
what kind of subspaces $V$ are the ones we want?
In the Grassmannian of $d$-dimensional real subspaces of $\CC^d$, there is a contractible
open subspace formed by these allowable complex metrics.
One way of thinking of the angles $\theta_i$ in \eqref{eqn:C_metric}
is that they are actually the critical values of $\arg$ regarded as a function on the real
projective space underlying $V$.

For each such choice of embedding $V\subset \CC^d$, there will be a version of observables
at $x$, $\Oc_x^{\left(V\right)}$.
One might worry that this depends on the particular embedding $V\subset \CC^d$, however
they all turn out to be canonically isomorphic. 
\end{rmk}

\subsection{Scaling}
\label{sec:scaling}

There are some crucial aspects of quantum field theory which cannot presently be
incorporated in the functorial framework. 
This section will explain how a certain scaling axiom might address this. 

Wick rotation, as we have discussed, is meant to an idea of positive energy into the
picture. On a general spacetime, this is encoded in terms of an energy-momentum tensor.
This is meant to be a local operator, i.e. something of the form $\Oc_x$ in
\Cref{sec:causality}. Unfortunately we cannot prove that the energy momentum is actually
of this form: In ordinary QFT on Lorentzian spacetime, field operators are things which
need to be smeared. More formally, they are operator-valued distributions. 
Although the definition we gave for local operators does indeed yield a locally
trivial vector space $\Oc_x$ over every point $x$ of our spacetime, but we haven't
introduced anything which would suggest that it is interesting to smear things. 

Another way one might have defined field operators is as follows.
Consider a cobordism $M$ from $M_0$ to $M_1$. 
It is natural to say that a field for some QFT $E$, say a scalar field to begin with,
is something which can be integrated against a smooth density on $M$ with compact
support (away from the boundary of $M$). 
From this perspective, a \emph{scalar field operator} or simply \emph{scalar field} might
be defined to be a morphism from the fields to operators between $E_{M_0}$ and $E_{M_1}$:
\begin{equation*}
\Dens_c\left(M\right) \to \Hom\left(E_{M_0} , E_{M_1} \right) 
\end{equation*}
By tampering with the densities (making them tensor densities of various kinds, etc.) one 
can define fields of various kinds, e.g. $1$-form fields, Riemannian metric fields etc..
One example of something defined this way is the energy momentum tensor:
When $M$ has a metric, we can differentiate the associated operator $E_{M}$ with respect
to the metric. The energy momentum tensor is this derivative, regarded as a transformation
of the above type: It associates an operator $E_{M_0}\to E_{M_1}$ to something dual to a
smooth metric on $M$.
General transformations of this type are \emph{Wightman fields}. 

Now the question is if Wightman fields are the same as smooth sections of the bundle with
fibers $\Oc_x$ as in \Cref{sec:causality}.
It is unclear how to relate these two perspectives without making some additional
assumptions. 
So this was the first problem: Defining an energy-momentum tensor, and then seeing how
these observables $\Oc_x$ behave under moving $x$, i.e. seeing 
that these bundles of observables have something to do with operator-valued distributions. 

Another problem is to prove that there is a good relation between Lagrangians 
(the thing that physicists think of as characterizing a QFT)
and functorial QFTs.
One upshot of the functorial QFT perspective is that there is a well-defined moduli space
of theories. 
Rescaling the complex metric induces a $1$-parameter flow, called the 
renormalization group flow, on this moduli space.

When we consider our moduli space of theories, and imagine making a deformation of it, we
would like to be able to 
prove that this deformation corresponds to a field in either of
the two senses (which are of course meant to be the same).
If one is able to do this, then one would have a fairly good link between theories
defined by Lagrangians and functorial theories.

We should be able to fix these with a scaling axiom 
(defined in Riemannian signature) which allows us to smear observables. 
Consider disks $D_r$ of radius $r$ around a point $x$, for $r\in \RR_{>0}$.
For any functorial QFT $E$, we obtain an inverse system of vector spaces
$E_{\left(r\right)} \coloneqq E_{\partial D_r}$. 
We can think that this forms a bundle over $\RR_{>0}$ with fiber over $r$ given by
$E_{\left(r\right)}$.
It seems that the required condition is that this bundle can be extended from $\RR_{>0}$
to $\RR_{\geq 0}$, so that is there is some $E_{\left(0\right)}$ associated to the point
$x$ itself. 

From this perspective a field operator, 
i.e. an element of $\Oc_x$, is a section of this bundle.
This will normally ``tend to infinity'' as $r\to 0$
since outward propagation is given by a contraction operator.
If one can extend the bundle to $0$, then one obtains a filtration on each $\Oc_x$, where 
a section has filtration $k$ if 
when you divide its representative in radius $r$ by $r^k$, 
then it tends to a limit in this eventual space $E_{\left(0\right)}$.
So we obtain a filtration by order of growth on the spaces $\Oc_x$, and 
$E_{\left(0\right)}$ is the associated graded to this filtration. 

An axiom of this kind would allow one to smear observables, which will allow us to pass
between the two definitions of fields.
Theories satisfying such an axiom are sometimes called 
``asymptotically conformal''.

The other thing, which we will focus on in the final lecture, 
is that in order to prove that an infinitesimal deformation is given by a field, then we
have to go from theories as we have defined them to ones which are ``extended''.
We have to add a layer of extra structure to our theory: To manifolds of codimension two
we will associate some kind of linear category. 

\section{6/12 -- More Desired Properties of QFT; Connection to Homotopy Theory}

\subsection{More on scaling}
\label{sec:scaling_2}

If you look at any physics book on QFT, scaling seems to be ubiquitous.
However in more abstract mathematical models of QFT, it very much doesn't play a role. 
For example consider the $2$-dimensional CFT given by the $\sigma$-model into a compact
real torus $T$. T-duality is the statement that this theory is equivalent to the 
$\sigma$-model into the Langlands dual torus $T^*$.
On the other hand, this isomorphism is totally incompatible with scaling: 
it interchanges short-distance and long-distance aspects of the theories.

\subsection{Locality}

Suppose that our theory sends a $(d-1)$-manifold $M$ to a (topological) vector space $E_M$.
To say that our theory is local, we'd like to say that $E_M$ can be constructed ``locally on $M$.''
More precisely, if $M = M_1 \cup_{\Sigma^{d-2}} M_2$, we'd like to construct $E_M$ from
$E_{M_1}$, $E_{M_2}$, and some to be determined $E_{\Sigma^{d-2}}$.
It would be nice if we simply had that $E_{M_1} \otimes E_{M_2}$ is equivalent to  $E_M$,
but this is not the case, as we see in the following low-dimensional example.

\begin{ex}
Let $G$ be a compact Lie group and $\Lc G$ the loop group of $G$.
Set $M = S^1$, so that the associated field theory is studying the representation theory
of $\Lc G$. 
Assume $E_M$ is an irrep of $\Lc G$.
These representations have the following wonderful property. 
If we think of $S^1$ as the quotient of an interval $I$, then we can define the subgroup
$\Lc_I G$ of $\Lc G$ to consist of the maps from $I$ to $G$ which vanish smoothly around
the endpoints. 
The positive energy representation theory of $\Lc_I G$ turns out to be exactly the same as
$\Lc G$. 

Any other interval $I'\subset I$ defines an associated subgroup 
$\Lc_{I'} G\subset \Lc_I G$.
For any $I'$, the orbit of any vector in the original representation under the 
group $\Lc_{I'} G$ will form a dense subset. 
Another feature of positive energy representations of loop groups, is that they admit an
intertwining action of the diffeomorphisms of the circle. 
There is always a diffeomorphism sending any such $I'$ to any other such $I'' \subset I$,
so we see that the positive energy representation theory of $\Lc_{I'} G$ is the same as
any other $\Lc_{I''} G$. 

In terms of locality, the point of this discussion was that if we decompose $M = S^1$
as $M_1 = I'$ and its complement, then there is no difference between 
$E_M$ and $E_{M_1}$, making it unclear how $E_{M_2}$ or $\Sigma$ should be involved in the
expression. This might lead one to believe that having an intertwining action of
diffeomorphisms is incompatible with locality, however this is not the case:
the theory is indeed local. 
\label[ex]{loop}
\end{ex}

In a general QFT, one should imagine that the space $E_M$ is completely
described by the action (on $E_M$) of all of the smeared operators which can be made out
of fields on the whole manifold.
Consider a decomposition $M = M_1 \cup_{\Sigma^{d-2}} M_2$.
Looking just at the smeared operators from $M_1$, one can still obtain
part of $E_M$, e.g. a dense subset by acting on a vacuum vector. 
Write algebras of fields in each of these regions as $\Ac_{M_1}$ and $\Ac_{M_2}$. 
As $M_1$ and $M_2$ are spacelike separated, these algebras commute.
Therefore we have an action of $\Ac_{M_1}\otimes \Ac_{M_2}$ on $E_M$, which is enough to
determine the action of all of the operators, i.e. we obtain the entirety of $E_M$, rather
than a dense subset, when we leave out observables supported on $\Sigma$.

On the other hand, if you just give me the individual actions of $\Ac_{M_1}$ and
$\Ac_{M_2}$ on $E_M$, then it is unclear how to obtain the entire theory for the same
reason as in \Cref{loop}.
If we consider slightly larger regions containing $M_1$ and $M_2$, written:
\begin{align*}
M_1  \subset M_1^+ \subset M 
&&
M_2 \subset M_2^+ \subset M
\end{align*}
then the entire theory can be constructed from keeping track of what happened along the 
intersection $M_1^+ \cap M_2^+$ for the following reason. 

Write the commutant of $\Ac_{M_1}$ acting on $E_M$ as 
$\Ac_{M_1}^{\mathrm{comm}}$. Note that this is a von Neumann algebra and it contains both:
\begin{equation*}
\Ac_{M_1^+}\subset
\Ac_{M_1}^{\mathrm{comm}}
\supset \Ac_{M_2}
\end{equation*}
The Tomita-Takesaki theorem tells us that there is an anti-involution 
\begin{equation*}
J \colon \Ac_{M_1} \to \Ac_{M_1}^{\mathrm{comm}}
\end{equation*}
and makes $\Ac_{M_1}^{\mathrm{comm}}$ the von Neumann closure of $\Ac_{M_1}$. 
First of all this says that knowing the von Neumann algebra is not really knowing 
anything at all: If you just want to know the von Neumann algebra
that acts beyond the horizon, you know it already because it is completely determined by
the operators you have in your hands. 
That in fact doesn't tell you anything about what's really happening outside, but it tells
you enough to reassemble the theory. 

Consider the operators in $M_1^+$ as a representation of the von Neumann algebra of things
which just live in the intersection $M_1^+\cap M_2$. 
Similarly the operators in $M_2^+$ can be thought of as a representation of the von Neumann
algebra of things on $M_2^+ \cap M_1$. Both of these things can be worked out just from
the local data. This gives us enough information to reassemble the actual global
situation:
We can reassamble the full vector space $E_M$ from these two von Neumann algebras in such
a way that $\Ac_{M_1}$ and $\Ac_{M_2}$ act on it. Since the combination of these two
algebras generate the observables everywhere, we have obtained the whole theory. 
This tells you what to associate to codimension $2$ submanifolds in order to extend the
TQFT one dimension downwards. 

\begin{ex}
Returning to \Cref{loop}, suppose we decompose the interval 
$I = I_1 \cup_{\pt} I_2$. Consider all possible ways of extending $I_1$ into an ambient
$2$-dimensional conformal manifold. 
There is an algebra of loops supported on any such extension, 
and invariance under diffeomorphisms of $S^1$ mean that they all have the same positive
energy representation theory. 
Bordisms between the different possible ways of extending $I_1$ induce 
bimodules between these algebras which relate representations of the algebra supported on
one extension to another.

The upshot is that something in codimension $2$ should be sent to the data of 
an algebra for each extension of $I_1$ along with bimodules relating the representations
of these algebras.
This is related to the ``Tomita-Takesaki flow.'' In the functional analysis literature
this is usually considered along the real axis, in which case it is simply rescaling along
any single flap. 
We have discussed a Wick-rotated version: If we simply rescale one extension then we
obtain a unitary, and the Wick rotation of this is a contraction operator. 
But generally, to go from one extension to another, this is only realized by a bimodule.

The category is actually not that difficult: In the case of the loop group, it is
essentially the complexification of the action of $\Diff\left(S^1\right)$ on all of the
little flaps around the point.
\end{ex}

\begin{rmk}
This is well-known to people who do general relativity:
In the context of field theories on Minkowski space this flow is known, when you do it
radially, to be the action of  Poincar\'e boosts in and out along the trajectories of the
Poincar\'e group.
\end{rmk}

\subsection{Non-commutative geometry}

There is a sense in which representations of bordism categories on linear categories can
be thought of as living in the world of non-commutative geometry. 
Before explaining this, consider the commutative ring $A = \Cinfty\left(M\right)$.
The geometry of $M$ can be constructed completely by writing the points of $M$ as
morphisms $A\to \CC$. In other words, the geometry of $M$ can be constructed entirely from
the category of $A$-modules.
Motivated by this, given a general noncommutative algebra $A$, we study the category
of $A$-modules as an avatar of the noncommutative geometry captured by $A$. Note that
therefore Morita equivalent noncommutative algebras describe the same noncommutative
geometry.

The $K$-theory of an algebra $A$ can be defined to be the finitely-generated projective
modules over $A$. In the case $A = \Cinfty\left(M\right)$, we find that the $K$-theory
encodes the cohomology of $M$, e.g. 
\begin{equation*}
K\left(A\right) \otimes \QQ = 
\bigoplus
H^{2i}\left(M , \QQ\right)
\end{equation*}
This decomposition relies directly on $A$ being commutative:
This is the eigenspace decomposition for the Adams operations, which only exist when $A$
is commutative. 

If we were doing a corresponding thing with complex cobordism, a different cohomology
theory, then a similar thing would be true:
Instead of Adams operations, which is essentially just an action of the multiplicative
group of the rationals on the $K$-theory, 
one would get an action of some formal group, which involves the formal group law of the
theory. 
It would not give you a grading, but something more complicated 
which is the subject of chromatic homotopy theory.
This scaling we discussed in \Cref{sec:scaling,sec:scaling_2}
(which essentially says that you are close to a commutative ring)
is meant to provide the possibility of making a decomposition/grading like this.

% Lecture 4 - 36 minutes

An important feature of this noncommutative homotopy category is the following notion of
Bott periodicity.
Consider the Schwartz space $\Cinfty_S(\RR^2)$ of functions decaying sufficiently rapidly
at infinity.
This can be thought of as the operators on a $2$-dimensional dynamical system, e.g. the
position and momentum of something moving in a single dimension. 
This has a canonical quantum/Heisenberg deformation,
$\Cinfty_S\left(\RR^2\right)_{\hslash}$, 
where functions on the two axes no longer commute, but rather the commutator is $i\hslash$. 
This deformation 
$\Cinfty_S\left(\RR^2\right)_{\hslash}$ is no longer commutative, and it is isomorphic
(via Fourier transform) to the algebra of ``smoothing operators'' from
$\Cinfty(\RR)$ to itself.
By ``smoothing operator,'' we mean a smooth function $K\left(x,y\right)$, viewed as an integral transform
\begin{equation*}
\left(Kf\right)\left(x\right) = \int f\left(y\right) K\left(x,y\right) dy \ .
\end{equation*}
These smoothing operators essentially form a full matrix algebra:
It only has one irreducible representation, which is the natural one on
$\Cinfty\left(\RR\right)$, and is therefore Morita equivalent to $\CC$. 

More generally, any object $X$ of the noncommutative homotopy category is 
homotopy equivalent to $X\times \RR^2_{\mathrm{cpt}}$.
In other words any object is equivalent to its two-fold suspension.
This corresponds exactly to the fact that, in the noncommutative homotopy category,
we lose the concept of dimension.

This scaling axiom might put it back, but in general one hits this problem immediately. 
This Bott-periodicity, required by this fundamental fact about quantization, 
is the essential difference between the ordinary homotopy category and the noncommutative
homotopy category:
In the noncommutative case, allowing maps to suspend an even number of dimensions gives
you naturally Floer homotopy types rather than ordinary homotopy types.

\subsection{The smooth homotopy category}

What is the right model of space on which one can do QFT?
The main thing we have is the long-distance behavior, which sort of looks like homotopy
theory: We don't have a notion of distance and things can be shoved around.
On the other hand, we can also look at short distances.
Segal believes that QFT is very much about manifolds, even at short
distances.\footnote{Segal warns: This is something not everyone, e.g. lattice model
practitioners, might not agree with.}

This is saying, in terms of the ring of
functions, that when we look close up we have the formal power series ring in independent
variables. There is indeed category with objects like this, and it has been much studied.
Segal became interested in this long ago when working with Raoul Bott on the so-called
Gelfand-Fuchs cohomology of vector fields on a manifold \cite{BS77}. 
As is shown there, the cohomology of this Lie algebra looks like the cohomology of some
space. 
Once one asks the question of how a Lie algebra can look like a space, one realizes that
we have secretly been doing this all the time:
When we look at the cohomology of a finite-dimensional Lie algebra, say for a compact group, 
it looks like cohomology of the compact group.
E.g. the Heisenberg group doesn't seem to have any topology at all, but the Lie algebra
still has cohomology groups.

Recall that the ordinary homotopy category can be defined as the contravariant functors
from the category of finite CW-complexes (and homotopy classes of maps)
to the category of sets with a certain Mayer-Vietoris property.
Such functors correspond precisely to homotopy types by Brown's representability theorem.
To define the \emph{smooth homotopy category}, one replaces the category of sets with
category of smooth manifolds. 
One can also replace the category of finite CW-complexes (and homotopy classes of maps)
with the (equivalent) category of finite-dimensional smooth open manifolds and smooth maps
between them up to smooth homotopy.

As was pointed out to Segal by Quillen in the 1960s, 
Lie algebras and groups can be seen as objects of the smooth homotopy category as follows.
Suppose we have a Lie algebra $\gfr$. To any object $M$ of the category of 
finite-dimensional smooth open manifolds 
we associate the space of flat $\gfr$-connections on the trivial bundle on $M$ up to
isomorphism.
This is the functor (object of the smooth homotopy category) representing $\gfr$. 
One has to be slightly careful about the notion of equivalence here:
Flat $\gfr$-connections certainly form a manifold
(they are the $1$-forms obeying the curvature condition)
and we need to quotient appropriately. 
Similarly, given a Lie group $G$, we can view the smooth homotopy theory of $G$ as given
by sending $M$ to the moduli space of flat $G$-bundles on $M$.

Given any pointed manifold $p\in X$, one can form the based loop space 
$\Omega_p X$. Work of Kapranov \cite{Kap07} exhibits $\Omega_p X$ as an object of 
the smooth homotopy category. 
For any bundle with connection on $X$ with structure group $G$, 
there is an associated morphism $\Omega_p X \to G$ as objects of the smooth homotopy category.
So this is the long-distance data: 
Homotopy classes of morphisms $\Omega_p X \to G$ classify bundles.
It is shown in \cite{Kap07} that the short-distance/local data, i.e. the induced map of
Lie algebras, is exactly all of the curvature data from the Riemannian structure.

The homotopy category is built from the category of sets in the following way:
Beginning with a set, suppose you allow yourself the concept of a path from one point to
another. Now allows yourself to add the concept of a homotopy between paths, etc..
Said differently, homotopy theory is fundamentally ``perturbative''.
Analysis is also ``perturbative:'' given a function we want to study near a point, first
we study its value, and then we take its derivatives one at a time. 

However, the theories are ``perturbative'' in different directions: homotopy theory gives
the ``long-distance perturbative theory'' while analysis gives ``short-distance
perturbative theory.''
The upshot is that, just as QFT is done perturbatively with a short and long-distance
version, the smooth homotopy category has a sort of homotopic perturbation
theory in addition to a sort of local/infinitesimal one.
