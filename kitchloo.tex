\chapter{Nitu Kitchloo -- Symmetry Breaking, Surface Defects, and Link Homologies}

\section{6/10 -- Overview}

\subsection{Artin braid groups}

Let $G$ be a compact connected Lie group with maximal torus $T$.
Recall that this means $T \subset G$ is a maximal connected compact abelian subgroup.

\begin{ex}
	For $G = \U(n)$, we can take $T$ to be the subgroup $\Delta$ of diagonal matrices.
	We will return to this example throughout.
\end{ex}

Given $(G,T)$, we can define an \emph{Artin braid group} $\Br(G, T)$.

\begin{ex}
	For $(G, T) = (\U(n), \Delta)$, this is the usual braid group
	\[
		\Br(G, T) = \Br_n := \langle \sigma_1, \dots, \sigma_{n-1} \, | \, \sigma_i \sigma_{i+1} \sigma_i = \sigma_{i+1} \sigma_i \sigma_{i+1}, \hspace{1em} \sigma_i \sigma_j = \sigma_j \sigma_i \textrm{ for } \abs{i - j} > 1 \rangle.
	\]
\end{ex}

More generally, note that $T$ acts on $G$ (and its Lie algebra $\gfr$) via the adjoint representation.
Complexify $\gfr$ to get a complex representation $\gfr_\CC = \gfr \otimes_\RR \CC$ of $T$.
Let $R$ be the set of roots of $G$, i.e.\ nonzero characters of $T$ in $\gfr_\CC$.
Let $\hfr = \Lie(T) \otimes_\RR \CC$, and let
\[
	\Hc = \hfr_\CC - \cup_{\alpha \in \RR} \hfr_\alpha \textrm{ for } \hfr_\alpha = \{ h \in \hfr_\CC \, | \, \alpha(h) = 0 \}.
\]
In other words, $\Hc$ is the union of the interiors of the Weyl chambers.
Note that $\Hc$ has a free action of the Weyl group $W(G, T) = N_T(G) / T$.

\begin{dfn}
	The \emph{Artin braid group} $\Br(G, T)$ is $\pi_1(\Hc / W(G, T))$.
\end{dfn}

It is a fact that $\Hc / W(G, T)$ is a $K(\Br(G, T), 1)$.
Let's see why this works for $G = \U(n)$.

\begin{ex}
	For $(G, T) = (\U(n), \Delta)$, we have
	\[
		\cup_\alpha \hfr_\alpha = \{ (x_1, \dots, x_n ) \, | \, x_i = x_j \textrm{ for some } i \neq j \}.
	\]
	Since $W(G, T) = \Sigma_n$ acting on $\hfr_\CC = \CC^n$ by permuting coordinates, we see that $\Hc / W(G, T)$ is the moduli of $n$ points in $\CC$.
	This picture gives us the usual definition of braid groups.
\end{ex}

Note that $\pi_0 \Map(S^1, \Hc / W(G, T))$ is the space of conjugacy classes of elements in $\Br(G, T)$, which we will write as $[\Br(G, T)]$.

\begin{ex}
	For $(G, T) = (\U(n), \Delta)$, elements of $\Br(G, T)$ are represented by braid closures.
	We may view these as links in $\RR^3$ using Markov's theorem.
\end{ex}

\begin{thm}[Markov]
	The map sending a braid to its closure expresses the set of equivalence classes of links in $\RR^3$ as the set of equivalence classes of all braids, $(\sqcup_{n \geq 1} \Br_n) / \sim$, where $\sim$ is the equivalence relation generated by M1 and M2:
	\begin{enumerate}
		\item M1 (Conjugation): $\sigma \sim \rho \sigma \rho\inv$ for $\sigma, \rho \in \Br_n$
		\item M2 (Stabilization): $\sigma \sim \sigma \sigma_n^{\pm 1}$ for $\sigma \in \Br_n \subset \Br_{n+1}$
	\end{enumerate}
\end{thm}

The individual maps from $[\Br_n]$ to the space of links are not faithful.
That is, stabilization does cause us to lose some information.

\subsection{Categorification}

We'd like to categorify $\Br(G, T)$ and $[\Br(G, T)]$.
In particular, we'd like a functor of (``discrete'' / non-$\infty$) 2-categories\footnote{Recent work extends this to a map of $(\infty, 2)$-categories.}
\[
	* \sslash \Br(G, T) \to \Gr(\Alg).
\]
Here $* \sslash \Br(G, T)$ is the 2-category with:
\begin{itemize}
	\item one object,
	\item $\Br(G, T)$ as 1-morphisms, and
	\item trivial 2-morphisms.
\end{itemize}
The right hand side, $\Gr(\Alg)$, is the 2-category with:
\begin{itemize}
	\item objects: graded commutative algebras,
	\item 1-morphisms: chain complexes of graded bimodules, and
	\item 2-morphisms: homotopy classes of chain maps
\end{itemize}
We also want a map $[\Br(G, T)] \to \Ch_{\bullet,\bullet}(\ZZ)$.

This has been worked out by Soergel, Rouquier, Khovanov, and Khovanov-Rozansky.
For $G = \U(n)$, the last map above gives bigraded homological invariants of links.
(This requires us to check that the stabilization move is satisfied.)

Being more ambitious, we can try to realize this using 3d and 4d topological gauge theories.
This splits into subgoals:
\begin{enumerate}
	\item Describe a configuration space of $G$-background fields on a 2-torus.
		(Here ``background'' means that we haven't ``summed over these fields.'')
	\item Quantize by ``summing over the background fields'' to obtain the partition function of the 2-torus in 3d and 4d topological gauge theories with defects.
	\item Identify the partition functions with the categorifications mentioned above.
\end{enumerate}

\subsection{Preliminaries: $\U(n)$}

We'll start out by looking at the case of $G = \U(n)$.
Let $R_n = H_T^\bullet(\pt) = \ZZ[x_1, \dots, x_n]$ with $\abs{x_i} = 2$.
The Weyl group $W$ is $\Sigma_n$, acting on $R_n$ with $\sigma_i$ swapping $x_i$ and $x_{i+1}$.

\begin{dfn}
	The \emph{Bott-Samelson $R_n$-bimodule} is the graded bimodule $\bS_i = R_n \otimes_{R_n^{\sigma_i}} R_n$, where $R_n^{\sigma_i}$ is the ring of $\sigma_i$-invariants in $R_n$.
	The category $\SBimod_n$ \emph{Soergel bimodules} is the subcategory of $R_n$-bimodules generated by the $\bS_i$'s (allowing idempotent completion and grading shifts).
\end{dfn}

\begin{thm}[Soergel]
	The category $\SBimod_n$ categorfies the Hecke algebra $H_q(A_n) = \ZZ[q, q\inv]\langle \Br_n \rangle / I$, where $I$ is generated by the relations
	\begin{align*}
		(\sigma_i - q^2)(\sigma_i + 1) &= 0 \\
		\sigma_i\inv = q^{-2} \sigma_i + (q^2- 1).
	\end{align*}
	That is, there is an equivalence $K_0(\SBimod_n) \cong H_q(A_n)$.
	Under this equivalence:
	\begin{itemize}
		\item $1$ corresponds to $R_n$,
		\item $\bS_i$ corresponds to $\sigma_i + 1$, and
		\item $q$ corresponds to grading shift.
	\end{itemize}
\end{thm}

The chain complex $C_i = (\bS_i \to R_n)$, where the map is given by $\alpha \otimes \beta \mapsto \alpha \beta$, should be viewed as a categorification of $\sigma_i$.

\begin{thm}[Rouquier]
	The assignment $\sigma_I \mapsto c_I := C_{i_1} \otimes_{R_n} C_{i_2} \otimes_{R_n} \dots \otimes_{R_n} C_{i_k}$ gives a well-defined map $* \sslash \Br_n \to \Gr(\Alg)$.
\end{thm}

\begin{ex}
	The braid $\sigma_1 \sigma_2$ is sent to the chain complex $C_{12} = (\bS_1 \to R_3) \otimes (\bS_2 \to R_3)$, which can also be written
	\[
		\begin{tikzcd}
			\bS_1 \otimes_{R_3} \bS_2 \rar & \bS_1 \oplus \bS_2 \rar & R_3.
		\end{tikzcd}
	\]
\end{ex}

Rouquier's theorem is nontrivial for two reasons:
\begin{itemize}
	\item We need to show that the map, which \emph{a priori} is defined only on braid words, is actually defined on the braid group.
	\item We need to make sure that this respects the triviality of the higher morphisms in $* \sslash \Br_n$.
\end{itemize}

\section{6/10 -- Geometric Interpretations for the $\U(n)$ Case}

Last time, we discussed a categorification of $\Br_n$.
We'd like to obtain a corresponding categorification of $[\Br_n]$.

\subsection{Categorification of $[\Br_n]$}

Recall the definition of Hochschild homology.

\begin{dfn}
	If $M$ is a graded $R_n$-bimodule, then the \emph{Hochschild homology} of $M$ is
	\[
		HH(M) = \Tor^{\bullet,\bullet}_{R_n \otimes R_n\op}(M, R_n)
	\]
\end{dfn}

We will consider the Hochschild homology $HH(C_I)$ termwise.

\begin{ex}
	For $\sigma_1 \sigma_2$, we discussed $C_{12}$ in the last lecture.
	With the above convention, $HH(C_{12})$ is
	\[
		\begin{tikzcd}
			HH(\bS_1 \otimes_{R_3} \bS_2) \rar & HH(\bS_1 \oplus \bS_2) \rar & HH(R_3).
		\end{tikzcd}
	\]
\end{ex}

\begin{thm}[Khovanov]
	The assignment $\sigma_I \mapsto HH(C_I)$ extends to $[\Br_n] \to \Ch_{\bullet,\bullet}(\ZZ)$.
	In fact, the output is naturally trigraded.
	Furthermore, this assignment satisfies M1 and M2, so it gives a homological invariant of links.
\end{thm}
	
This invariant can be identified with the trigraded Khovanov-Rozansky link homology.
If $\Lc$ is the link $[\sigma_i]$, we write
\[
	\slfr_\infty(\Lc) = HHH(\Lc) = HH(C_I).
\]
We may view $\slfr_\infty$ as a ``limit'' of bigraded homological invariants $\slfr_N(\Lc)$ which can be obtained from $HH(C_I)$ by adding a ``matrix factorization'' differential.

\subsection{Topological interpretation of Bott-Samelson bimodules}

\begin{dfn}
	For $i < n$, let $G_i \subset \U(n)$ be the semisimple rank one compact parabolic consisting of matrices of the form
	\[
		\begin{bmatrix}
			D_1 & 0 & 0 \\
			0 & U & 0 \\
			0 & 0 & D_2
		\end{bmatrix}
	\]
	with $D_1$ and $D_2$ diagonal and $U \in \U(2)$ (placed in rows / columns $i$ and $i + 1$).
\end{dfn}

Note that $T \subset G_i \subset G = \U(n)$ for all $i$.

\begin{dfn}
	Let $\BS_i$ be the balanced product $\U(n) \times^T (G_i / T)$, considered with its natural $\U(n)$-action.
\end{dfn}

We may obtain $\BS_i$ from a pullback square
\[
	\begin{tikzcd}
		\BS_i \rar["\alpha"] \dar["\beta"] & \U(n) / T \dar \\
		\U(n) / T \rar & \U(n) / G_i
	\end{tikzcd}
\]
where $\alpha: [g, g_i] \mapsto [g g_i]$ and $\beta: [g, g_i] \mapsto [g]$.

In cohomology, we get a map
\[
	H^\bullet_{\U(n)}(\U(n)/T) \otimes_{H^\bullet_{\U(n)}(\U(n)/G_i)} H^\bullet_{\U(n)}(\U(n)/T) \to H^\bullet_{\U(n)}(\BS_i).
\]
A spectral sequence computation shows this is an isomorphism.
Combining this with the standard isomorphisms $H^\bullet_G(G / K) \cong H^\bullet_K(\pt)$ for any $G$ and any $K \subset G$, we get
\[
	H^\bullet_{\U(n)}(\BS_i) \cong H^\bullet_T(\pt) \otimes_{H^\bullet_{G_i}(\pt)} H^\bullet_T(\pt) \cong R_n \otimes_{R_n^{\sigma_i}} R_n = \bS_i.
\]
Thus, we can interpret the Bott-Samelson bimodule $\bS_i$ geometrically as the $\U(n)$-equivariant cohomology of $\BS_i$.

The pullback square defining $\BS_i$ is in fact a homotopy pullback square.
Comparing with the standard definition of the homotopy pullback, we can view $\BS_i$ as the moduli space of $\U(n)$-bundles on $[0, 1]$ with a reduction of structure group to $G_i$ on $[0, 1]$ and to $T$ on $\{ 0, 1 \}$.

\subsection{Interpretation of Rouquier's chain complex}

More generally, for a multi-index $I = (i_1, \dots, i_k)$, define
\[
	\BS_I = \U(n) \times^T (G_{i_1} \times^T G_{i_2} \times^T \dots \times^T G^{i_k} / T).
\]
Then
\[
	H^\bullet_{\U(n)}(\BS_I) \cong \bS_{i_1} \otimes_{R_n} \bS_{i_2} \otimes_{R_n} \dots \otimes_{R_n} \bS_{i_k}.
\]
We may view $\BS_I$ as the moduli space of $\U(n)$-bundles on $[0, 1]$ with $k + 1$ marked points (including $0$ and $1$) with reduction of structure group to $G_{i_j}$ on the $j$th interval and to $T$ at each marked point.

If $J$ is obtained from $I$ by discarding indices, then we get $\BS_J \hookrightarrow \BS_I$ by replacing every factor $G_{i_k}$ (for each removed index $i_k$) by $T$ in the construction of $\BS_I$.
These replaced copies of $T$ are ``transparent defects.''
This inclusion respects composition of inclusions of multi-indices.

Rouquier's chain complex $C_I$ is isomorphic to
\[
	\begin{tikzcd}
		H^\bullet_{\U(n)}(\BS_I) \rar & \oplus_{\abs{J} = k-1, J \subset I} H^\bullet_{\U(n)}(L_J) \rar & \dots \rar & H^\bullet_{\U(n)}(\U(n)/T).
	\end{tikzcd}
\]
All differentials here are obtained from (signed) inclusions $\BS_J \to \BS_{J'}$.

\subsection{Interpretation of $HH(\bS_I)$}

To interpret $HH(H^\bullet_{\U(n)}(\BS_I)$, we introduce the following.

\begin{dfn}
	Let
	\[
		L_i = \U(n) \times^T G_i
	\]
	where $T$ acts on $G_i$ by conjugation.
\end{dfn}

The map $L_i \to \BS_i$ given by $[g, g_i] \mapsto [g, g_i]$ is a principal $T$-bundle.
In fact, we have a pullback square
\[
	\begin{tikzcd}
		E\U(n) \times^{\U(n)} \rar \dar & BT \dar["\Delta"] \\
		E\U(n) \times^{\U(n)} \BS_i \rar & BT \times BT
	\end{tikzcd}
\]
where both vertical arrows are trivial principal bundles.

Collapse of the Eilenberg-Moore spectral sequence at the $E_2$ page gives
\[
	H^\bullet_{\U(n)}(L_i) \cong \Tor_{H^\bullet(BT \times BT)}(H^\bullet_{\U(n)}(\BS_i), H^\bullet(BT)) = HH(H^\bullet_{\U(n)}(\BS_i)) = HH(\bS_i).
\]
Thus we obtain a geometric interpretation of $HH(\bS_i)$.
We may interpret $L_i$ as the moduli space of $\U(n)$-bundles on $S^1$ with one marked point, where the structure group reduces to $G_i$ away from the marked point and to $T$ at the marked point.

More generally, for a multi-index $I$, we let
\[
	L_I = \U(n) \times^T (G_{i_1} \times^T G_{i_2} \times^T \dots \times^T G_{i_k})
\]
where $T$ acts by conjugation on $G_{i_1}$ and $G_{i_k}$.
An argument analogous to the above gives $H^\bullet_{\U(n)}(L_I) \cong HH(\bS_I)$.

For $J \subset I$, we get an inclusion $L_J \to L_I$ that behaves well with respect to composition of inclusions.
These assemble into a complex
\[
	\begin{tikzcd}
		H^\bullet_{\U(n)}(L_I) \rar & \oplus_{\abs{J} = k-1, J \subset I} H^\bullet_{\U(n)}(L_J) \rar & \dots 
	\end{tikzcd}
\]
which is isomorphic to $HH(C_I)$.\footnote{This is a ``cubical complex,'' related to crossing resolutions for nice presentations of knots.}

\section{6/11 -- Local Systems and Twisted Cohomology}

\subsection{Review of bundle theory}

Let $G$ be a topological group.
Principal $G$-bundles are classified by pulling back a ``universal bundle'' $EG \to BG$.
That is, for any principal $G$-bundle $E \to B$, there exists an $f: B \to BG$ (unique up to homotopy) such that $E = f^* EG$.
To construct $EG \to BG$, just take any contractible space $EG$ with free $G$-action, and let $BG = EG / G$.

We will write a $G$-space $X$ as $X \sslash G$.
From such a space, we may construct the \emph{homotopy orbits} $X \sslash^h G = EG \times^G X := (EG \times X) / G$, where $G$ imposes the relation $(e, x) \sim (eg, g\inv x)$.
We define the \emph{$G$-equivariant cohomology} of $X$ to be
\[
	H^\bullet_G(X) := H^\bullet(X \sslash^h G).
\]

If $K \subset G$, then we may take $EG$ as a model for $EK$.
Because $EG \times^G (G \times^K Y) = EG \times^K Y = EK \times^K Y$, we obtain
\[
	H^\bullet_G(G \times^K Y) = H^\bullet_K(Y).
\]

Given a principal $G$-bundle $E \to B$, we may ask whether or not $E$ is extended from a principal $K$-bundle, i.e.\ whether $E \cong E' \times^K G$ for some principal $K$-bundle $E' \to B$.
This is equivalent to asking whether the classifying map $f_E: B \to BG$ lifts to a map $f_{E'}: B \to BK$.
Note that $BK \to BG$ has fiber $G/K$, so in particular reductions of structure group of the trivial $G$-bundle to $K$ correspond to maps $B \to G / K$.

\subsection{Recap}

Last time, we defined subgroups $G_i \subset \U(n)$ for $1 \leq i < n$.
For each multi-index $I = (i_1, \dots, i_k$, we defined spaces $\BS_I = \U(n) \times^T (G_{i_1} \times^T G_{i_2} \times^T \dots \times^T G_{i_k} / T)$.
We showed
\[
	H^\bullet_{\U(n)}(\BS_I) = \bS_{i_1} \otimes_{R_n} \dots \otimes_{R_n} \bS_{i_k}
\]
for $\bS_i = R_n \otimes_{R_n^{\sigma_i}} R_n$.
The spaces $\BS_I$ are functorial with respect to inclusions of multi-indices.

\begin{ex}
	Let $I = \{ i \}$ and $J = \emptyset$, so $\BS_i = \U(n) \times^T G_i / T$ and $\BS_\emptyset = \U(n) \times^T (T / T) = \U(n) / T$.
	Note that $H^\bullet_{\U(n)}(\BS_i) = \bS_i$ and $H^\bullet_{\U(n)}(\U(n) / T) = H^\bullet_T(\pt) = R_n$.
	The inclusion $\BS_\emptyset \to \BS_i$ comes from $T \subset G_i$, and the induced map on cohomology is the multiplication map
	\[
		\bS_i = R_n \otimes_{R_n^{\sigma_i}} R_n \to R_n.
	\]
	This is the chain complex $C_i$ appearing in Rouquier's work.
\end{ex}

We also defined spaces $L_I = \U(n) \times^T (G_{i_1} \times^T G_{i_2} \times^T \dots G_{i_k})$ where $T$ acts on $G_{i_1}$ and $G_{i_k}$ by conjugation.
These spaces may be viewed as moduli of $\U(n)$-bundles on $S^1$ with $k + 1$ marked points with reduction of structure group to $G_{i_j}$ on the $j$th interval and to $T$ at each marked point.
The cohomologies can be used to construct a chain complex which recovers Khovanov's link homology.
Specifically, we have $H^\bullet_{\U(n)}(L_I) = HH(\bS_I)$.

\subsection{Recovering $\slfr_N$-link homology}

There is a forgetful map $L_I \sslash \U(n) \to \U(n) \sslash \U(n)$, where $\U(n)$ acts on itself via the adjoint representation.
This map is given by the formula
\[
	[g, g_1, \dots, g_k] \mapsto g g_1 \dots g_k g\inv.
\]
If one thinks of the bundles appearing in the moduli description of $L_I$ as having connections, this is essentially taking the product of the holonomies.

The functorial maps of $L_I \sslash \U(n)$ are compatible with the forgetful maps to $\U(n) \sslash \U(n)$.
In particular, Khovanov's chain complex
\[
	\begin{tikzcd}
		H^\bullet_{\U(n)}(L_I) \rar & \oplus_{\abs{J} = k-1, J \subset I}(L_J) \rar & \dots 
	\end{tikzcd}
\]
is a complex over the torsion-free $H^\bullet_{\U(n)}(\U(n))$.
Let's think about what happens if we take termwise cohomology with respect to odd elements of $H^\bullet_{\U(n)}(\U(n))$.

There is a non-canonical ring isomorphism 
\[
	H^\bullet_{\U(n)}(\U(n)) \cong H^\bullet_{\U(n)}(\pt) \otimes H^\bullet(\U(n)).
\]
The non-canonicalness of this is related to the fact that $\U(n) \sslash^h \U(n) \not\simeq B\U(n) \times \U(n)$.
Essentially, it's not clear what to do on primitive Hopf algebra elements.

However, if we take $n \to \infty$ (and write $\U = \U(\infty)$), we do get a canonical equivalence
\[
	\U \sslash^h \U \simeq B\U \times \U.
\]
Pulling back the elements of $H^\bullet(\U)$ gives odd elements $d_1, d_3, \dots$, and we can modify Khovanov's complex to get
\[
	\begin{tikzcd}
		H^\bullet(H_{\U(n)}(L_I), d_{2N-1}) \rar & \oplus_{\abs{J} = k-1, J \subset I} H^\bullet(H_{\U(n)}(L_J), d_{2n-1}) \rar & \dots 
	\end{tikzcd}
\]
Here $n$ and $N$ are unrelated.

\begin{thm}[T. Gomez]
	This complex is isomorphic to the link homology $\slfr_N(\Lc)$ for the link $\Lc = [\sigma_I]$.
\end{thm}

\begin{ex}
	Let $\Lc$ be the unknot, so $\Lc = [\sigma_1]$.
	Then $\slfr_\infty(\Lc) = H^\bullet_{\U(1)}(\U(1) \times^T T) = H^\bullet_T(T) = \ZZ[x_2, y_1]$.
	Here $d_N$ corresponds to the odd class $y_1 x^N$.
	We get
	\[
		H^\bullet(H_{\U(1)}(L_I), d_N) \cong (\ZZ[x] / x^N) \langle y_1 \rangle.
	\]
\end{ex}

\subsection{Local systems}

The above story was a bit ad hoc, so we'd like to fit it into a more topological context.
We can think of $\U(n)$ as $B \Omega \U(n)$, so $\Omega U(n)$ is really acting everywhere.
In fact, we'd like the second Markov move (stabilization) to play well with our theory, so we should pass to an $\Omega \U$-action.
We can implement this by using a \emph{twisted cohomology theory}
\[
	\Hc = \Omega \U_+ \wedge H_\ZZ = (\ZZ \times B\U)_+ \wedge H_\ZZ.
\]
One can compute
\[
	\pi_\bullet \Hc = \pi_\bullet H[b_0^\pm, b_1, b_2, \dots]
\]
with $\abs{b_i} = 2i$.

Setting $b_0 = b_N = 1$ and $b_i = 0$ otherwise (and indicating this with the subscript $N$), one can calculate
\[
	\tilde{\Hc}_{\U(n)}^\bullet(L_I)_N \simeq H^\bullet(H_{\U(n)}(L_I), d_n).
\]

\section{6/12 -- Spectral Sequence Invariants; Relationship to TQFT}

\subsection{Remarks on questions}

Previously, we constructed (for $G = \U(n)$) a $G$-space $L_I$ such that $H_G(L_I) = HH(\bS_I)$.
There is a $G$-equivariant map $L_I \to G$ (where $G$ acts on itself by convolution), and we can view this as a derived local system.
For these purposes, we should view $G = B(\Omega G)$.

\begin{ex}
	Consider $\Omega G$-twisted cohomology $\Hc = (\Omega \U)_+ \wedge H_\ZZ$.
	Then $\pi_\bullet \Hc = (H_\ZZ)_\bullet [b_0^{\pm 1}, b_1, \dots]$.
	Here the $b_i$ control the possible twistings.
	Setting $b_0 = b_N = 1$ and $b_i = 0$ otherwise gives $\slfr_N$-link homology.
	We can consider the power series $\sum_i b_i x^i$, which can be viewed in a matrix factorization context as the derivative of our superpotential.
\end{ex}

\begin{ex}
	We can also consider twisted $K$-theory.
	Note that $K_G$ has twistings indexed by $\NN$.
	Here (for $N \in \NN$)
	\[
		{}^N K_{\U(1)}(L_\emptyset) = \ZZ[x^{\pm 1}] / (x^N - 1)
	\]
	gives the corresponding invariant for the unknot.
	Freed-Hopkins-Teleman allows us to identify this with the ring controlling level $N$ representations of the loop group $L\U(1)$.
	There is a ``level-rank duality'' going on here relating this to $\ZZ[x] / (x^N)$.
\end{ex}

Twisted $K$-theory is a functor defined on $G$-spaces $X$ over $G$.
This produces modules over ${}^0 K_G(X)$.

\subsection{Spectral sequence interpretation of Khovanov homology}

Recall that an inclusion of multi-indices $J \subset I$ induces $L_J \subset L_I$.
We used this to construct a complex producing Khovanov homology.
We'd like to interpret this topologically.
To this end, we construct a filtered $G$-equivariant space $F_\bullet(sL_I)$ with 
\begin{align*}
	F_0(sL_I) &= \cone(\emptyset \to L_I) = L_I \\
	F_1(sL_I) &= \cone\left(\cup_{|J|=k-1} L_J \to L_I\right) \\
	F_2(sL_I) &= \cone\left(\cup_{|J|=k-1 \textrm{ or } k-2} L_J \to L_I\right) \\
		  &\dots
\end{align*}
We can compute $F_k(sL_I) / F_{k-1}(sL_I) = \vee_{\abs{J}=n-r} \Sigma^r (L_{J})_+$.
This yields a spectral sequence with $E_1$ term $HH(C_I^\bullet)$ converging to $H^\bullet_{\U(n)}(F_k(sL_I))$.
This limit space is not particularly interesting -- it's just a Thom space, and it counts the number of components of the link.
However, the terms of the spectral sequence are more interesting.

\begin{thm}
	The pages $E_i$ for $i \geq 2$ are invariants of the link $\sigma_I$.
\end{thm}

We'd like to view the filtered $G$-space $F_\bullet(sL_I)$ as a link invariant itself.
First, we need a notion of equivalence for such objects.
We will consider filtered $G$-spaces as giving key examples of $G$-spectra.

\begin{dfn}
	Let $F_\bullet$, $G_\bullet$ be $G$-spectra.
	We say that a map $f_\bullet: F_\bullet \to G_\bullet$ is an \emph{elementary quasi-equivalence} if its (co)fiber $Z_\bullet$ is acyclic, i.e.\ the identity on $Z_\bullet$ is homotopic to $0$.
	More generally, we say $F_\bullet$ and $G_\bullet$ are \emph{quasi-equivalent} if they are connected by a zigzag of elementary quasi-equivalences.
\end{dfn}

\begin{thm}
	The filtered $G$-space $F_\bullet(sL_I)$ is an invariant of the link $\sigma_I$ up to quasi-equivalence.
\end{thm}

\subsection{Towards a TQFT interpretation}

Suppose that $M$ is a manifold with a $G$-bundle (thought of as a background field).
This is classified by a map $f_M: M \to BG$.
For $M = N \times S^1$, we may regard the classifying map as equivalent to a map $\hat{f}_N: N \to LBG = \Map(S^1, BG)$.

\begin{prop}
	The free loop space $LBG$ is equivalent to $G \sslash^h G = EG \times^G G$ (where $G$ acts on itself by conjugation).
\end{prop}

\begin{proof}
	Decomposing $S^1 = D^+ \cup_{S^0} D^-$, we may view the data of a map $S^1 \to BG$ as consisting of a bundle on $D^+$, a bundle on $D^-$, and an equivalence of the two on $S^0$.
	Written another way, this is a homotopy fiber product
	\[
		\begin{tikzcd}
			LBG \rar \dar & BG \dar["\Delta"] \\
			BG \rar["\Delta"] & BG \times BG.
		\end{tikzcd}
	\]
	Writing
	\begin{align*}
		   BG &\simeq E(G \times G) / \Delta \\
		   &\simeq E(G \times G) \times^{G \times G} (G \times G) / \Delta \\
		   &\simeq E(G \times G) \times^{G \times G} G,
	\end{align*}
	we can write $LBG = BG \times_{BG \times BG}^h (E(G \times G) \times^{G \times G} G)$, which gives the desired result.
\end{proof}

The upshot is that $G$-background fields on $N \times S^1$ are equivalent to $G \sslash^h G$-background fields on $N$.
It follows that we should study $G \sslash G$ and $G \sslash^h G$.
If we let $G / G$ be the space of bona-fide $G$-orbits, we get a map $G \sslash G \to G / G$.

Recall that every element of $G$ is conjugate to an element of $T$, so we can write $G / G = T / W$, where $W = N_G(T) / T$.
For $\hfr = \Lie(T)$, we have an exponential map $\exp: \hfr / W \to T / W$, where $\hfr / W$ is equivalent to a Weyl chamber.
We can interpret $G \sslash G$ or ($G \sslash^h G$) as stacks / spaces over $\hfr / W$.
In the open parts of the Weyl chambers, the fibers look like $(G / T) \sslash G$.
Along the walls of the Weyl chambers, we get jumps / defects where the fibers become $(G / G_i) \sslash G$.

We can also interpret the spaces $L_I$.
Given $I$, inscribe $S^1$ in $\hfr / W$ so that the circle hits the wall $\alpha_{i_j} = 0$ between the points $x_j$ and $x_{j+1}$.
Then $L_I$ consists of sections
\[
	\begin{tikzcd}
		 & {G \sslash G} \dar \\
		S^1 \rar \ar[ur, dashed] & \hfr / W
	\end{tikzcd}
\]

\subsection{Homotopy TQFT interpretation}

We may view $L_I$ as $G \sslash G$-background fields on $S^1$, or equivalently as $G$-background fields on $S^1 \times S^1$.
Let $\Cob_n(\pt \sslash^h G)$ be the $(\infty, n)$-category of manifolds with $G$-background.
Let $\Hc$ be a commutative ``global'' cohomology theory -- see the book of Stefan Schwede for details of what this means.
Write $\Cc_n$ for the category of $\EE_{n-1}$-algebras in $\Hc$-modules.

There is a functor $\Cob_n(\pt \sslash^h G) \to \Cc_n$ given by $\pt \sslash G \mapsto \Hc$.
Compactifying this along $S^1$ (i.e.\ considering values on manifolds of the form $M = N \times S^1$) gives a functor $\Cob_{n-1}(G \sslash^h G) \to \Cc_{n-1}$, defined by $G \sslash^h G \mapsto (\Hc \times G) \sslash G$.
On $S^1$ with $L_I$ background, we get $(L_I \times \Hc \to L_I) \sslash G$.
This is an invertible theory, and we can gauge / take $G$-equivariant global sections to get
\[
	\Hc_G(L_I) \simeq HH(\bS_I).
\]
The inclusions $L_J \hookrightarrow L_I$ give Khovanov's complex.

Twisted theories give modules over $\Hc$, which doesn't quite make sense in the above story.
Instead, we want to view our theory as being the boundary of a 4d theory.
That is, Khovanov-Rozansky $\slfr_N$-link homology comes from a 3d theory on the boundary of a 4d theory.
