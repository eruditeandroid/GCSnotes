\chapter{Ibou Bah -- Lagrangian Methods for Categorical Symmetries}

\section{6/11 -- What is QFT?}

This will be a physics-y, intuitive series of lectures.
We will start by giving some idea of QFT in general.
In the second lecture, we will move on to finite symmetry groups and use these to construct topological operators.
In the third lecture, we will discuss the action of finite symmetry groups on fields (with an emphasis on the example of $\SU(N)$ Yang-Mills theory).
Finally, we will discuss the general construction of topological operators using Lagrangians.

\subsection{What should a QFT do?}

We don't actually know what a QFT \emph{is}.
However, we have an excellent physical idea of what a QFT should \emph{do}.
QFTs should have:
\begin{itemize}
	\item A list of operators: point operators $\{ \Oc_i(x) \}$, line operators $\{ U_\alpha(\gamma_p) \}$, and other operators.
	\item A Hilbert space $\Hc$ of states on spacetime $M = \RR_t \times Y$.
	\item A partition function sending combinations / products of operators to complex numbers, e.g.\ $\langle \Oc_i \Oc_j \rangle$.
\end{itemize}
These should satisfy three key properties: \emph{unitarity}, \emph{locality}, and \emph{causality}. 
Let's discuss each of these in more detail.

\subsection{Unitarity, locality, and causality}

Unitarity can be interpreted as ``preservation of probability.''
For example, for many QFTs, we can construct an ``$S$-matrix'' governing scattering amplitudes of particles, and unitarity forces $S^\dagger S = 1$.
This specific interpretation doesn't work for CFTs / TFTs.
Instead, we use a notion of \emph{reflection positivity}.

To explain this, recall that time evolution in quantum mechanics is governed by
\[
	\ket{\psi} \mapsto e^{-i H t} \ket{\psi(t)}
\]
and
\[
	\Oc(x, 0) \ket{\psi} \mapsto \Oc(x, t) \ket{\psi}.
\]
We should have
\[
	\bra{\psi} \Oc^\dagger \Oc \ket{\psi} = \norm{\Oc(x, t) \ket{\psi}} \geq 0.
\]
To make sense of $\Oc^\dagger$, we allow our time variable to take complex values.
Say that $\Oc(x, t + i \tau)^\dagger = \Oc(x, t - i \tau)$.
Then the above condition forces $\langle \Oc(x, t - i \tau) \Oc(x, t + i \tau) \rangle \geq 0$.
Taking $t = 0$, we see that reflection positivity requires $\langle \Oc(x, -i \tau) \Oc(x, i \tau) \rangle \geq 0$.
More generally, reflection positivity forces
\[
	\langle \Oc_i(x, -i \tau) \Oc_j(x, -i \tau) \dots \Oc_j(x, i \tau) \Oc_i(x, i \tau) \rangle \geq 0.
\]

Locality forces point operators to have an \emph{operator product expansion}
\[
	\Oc_i(x_1) \Oc_j(x_2) = \sum_k c_{ij}^k(x_1, x_2) \Oc(x_1)
\]
governing the behavior as $x_1 \to x_2$.
(For conformal field theories, the $c_{ij}^k$ should be meromorphic.)
More generally, the interaction of operators should depend only on their behavior in a common tubular neighborhood.
If two point operators are outside the same light cone, then we can find a reference frame where the two operators take place at the same time but at different points in space.

Finally, causality requires that any operators which are spacelike separated commute with each other.
If two point operators are in the interior of the same light cone, then we can find a reference frame where the operators take place at the same point in space but at different points in time.
Things are more complicated for extended operators.

\subsection{Extra features}

In many QFTs, we can expect other nice properties to hold:\footnote{There's also a last property of ``analyticity'' which is hard to formulate.}
\begin{itemize}
	\item Cluster decomposition:
	\[
	\lim_{b \to \infty} \langle \Oc_i(x_1) \Oc_j(x_2) \dots \Oc_{i'}(x_1 + b) \Oc_{j'}(x_2 + b) \dots \rangle = \langle \Oc_i(x_1) \Oc_j(x_2) \dots \rangle \langle \Oc_{i'}(x_1 + b) \Oc_{j'}(x_2 + b) \dots \rangle.
	\]
	\item Renormalization: high energy / short distance / UV physics and low energy / long distance / IR physics decouple, at least up to finitely many parameters.
\end{itemize}
Axiomatic treatments can take these as additional axioms.

Gravity fails both of these conditions and locality.
Black holes are UV phenomena which are still visible at low energy.
Furthermore, black holes also break cluster decomposition.
Whatever quantum gravity is, unitarity ``must'' be retained, since it's fundamental to quantum physics.

The goal of the Simons collaboration is to understand the kinematics of operators in QFT.
This is governed by symmetries!
We can essentially think of this as understanding the \emph{labels} of operators $\Oc_i(x)$, $U_\alpha(\gamma)$, etc.

\subsection{Where do QFTs come from?}

There are many sources of QFTs.

\begin{ex}[Lattices]
	Consider a lattice at which each point behaves according to quantum mechanics.
	Taking a continuum limit of the lattice gives a quantum field theory.
	This limit leads to the appearance of infinities and other strange phenomena.
	Nevertheless, it's the most physically ``rigorous'' way of constructing QFTs.
\end{ex}

\begin{ex}[Geometric engineering]
	Start with a string theory, and ``decouple gravity'' by freezing some dimensions.
	This produces a QFT involving geometric objects.
	Class $\Sc$ theories arise in this way.
\end{ex}

\begin{ex}[Lagrangians]
	We can construct many QFTs from Lagrangians.
	We'll spend most of our time focusing on this perspective.
\end{ex}

Each perspective has advantages over the others:
\begin{itemize}
	\item Lattices make renormalization straightforward.
	\item Geometric engineering makes the geometric nature of objects more apparent.
	\item Lagrangians make locality explicit.
\end{itemize}
	
\subsection{Lagrangians}

A Lagrangian is a functional depending on the fields $\phi_\alpha(x)$:
\[
	L(x) = L(\phi_\alpha(x), \partial \phi_\alpha(x), \dots)
\]
This is a classical object.
Locality enforces that $L$ only depends on the behavior of the fields at the given input point (unlike, say, Green's functions $G(x_1, x_2)$).
Locality also ensures that there are only finitely many derivative terms appearing in $L$.

Lorentz symmetry (which we will assume implicitly) tells us that our fields can be:
\begin{itemize}
	\item Scalar fields $\phi(x)$
	\item Vector fields / gauge fields $A_\mu(x)$, either:
		\begin{itemize}
			\item Massless: living in the representation $(1, -1)$, or
			\item Massive: living in $(1, 0, 1)$
		\end{itemize}
	\item Fermions $\psi_a$ (representations of a Clifford algebra)
	\item Metrics $g_{\mu \nu}$
\end{itemize}
We won't consider gravitational theories, so we'll ignore $g_{\mu \nu}$.

\section{6/11 -- Lagrangians in QFT}

Unitarity is true for closed quantum systems.
For open quantum systems, where we ``trace away'' some information, we can drop this assumption.
We must restore unitarity as we restore data to our system.
This explains e.g.\ where the $W$-boson and Higgs boson come from: they're needed to restore unitarity in the Standard Model.

\subsection{Lagrangians}

Consider a Lagrangian
\[
	\Lc(x) = \Lc(\phi_\alpha(x), \partial \phi_\alpha(x), \dots).
\]
From this (classical) data, we can define a (quantum) partition function
\[
	Z = \int D[\phi_\alpha] e^{i \int_M \sqrt{g} \Lc(\phi_\alpha)}
\]
This can be thought of as a formal expression: heuristically, it is a sum over all possible field configurations.

We can obtain correlation functions, etc.\ by inserting local operators
\[
	\Oc_i(\phi_\alpha(x), \partial \phi_\alpha(x), \dots)
\]
or extended operators
\[
	U_a(\gamma) = U\left(\int_\gamma F(\phi_\alpha(x)), \dots\right).
\]
Namely, we declare
\[
	\langle \Oc_i \dots \rangle = \frac{1}{Z(0)} \int D[\phi_\alpha] e^{i \int \Lc} \Oc_i \dots.
\]

\subsection{Parameters}

In general, it is possible that our Lagrangian depends on some additional parameters, say $\tau_a$, $\tau_b$.
The partition function then also depends on the $\tau$'s.
We will assume that these parameters are \emph{perturbative}: that small changes in our theory are governed by Taylor expansions in terms of the $\tau$'s.
Often, there is a large parameter space for a given QFT, and there are many perturbative regions within this space corresponding to different weakly coupled versions of the theory.
These \emph{duality frames} can have different fields, different Lagrangians, and different gauge groups.
(Gauge symmetry is a redundancy in parametrization -- it is not a physical symmetry, but rather a ``trick'' enabling the use of local Lagrangians.)

However, the global symmetry, Hilbert space, and local operators are the same in every duality frame.\footnote{The term ``global'' is not related to this, however.}
This may not always be visible (at least at the classical level) but it is true.
Global symmetry permutes the operators, while gauge symmetry does nothing to the operators.

\begin{ex}
	We can view 4d $\Nc = 4$ super-Yang-Mills as part of the same parameter space as type IIB supergravity on $\mathrm{AdS}_5 \times S^5$.
	Thus these are versions of the same theory.
\end{ex}

In addition to varying our parameters, we can also turn on background fields $B_i(x)$, i.e.\ add in new fields which do not fluctuate.

\subsection{Examples}

Consider a theory with one free massless scalar field $\phi$.
The Lagrangian is
\[
	\Lc = \frac{1}{e^2} d \phi \wedge \star d \phi,
\]
a single kinetic term.
The partition function is
\[
	Z = \int D[\phi] \exp\left( \frac{i}{e^2} \int d\phi \wedge \star d\phi \right).
\]
Taking the coupling parameter $e \to 0$ yields $d\phi = 0$.

We can also add matter, incorporated via a term $m^2 \phi^2 \Vol(M)$ added to the Lagrangian.
Taking $m \to \infty$ gives $\phi = 0$.

The stress tensor of the theory is
\[
	T_{\mu\nu} = \frac{\delta \Lc}{\delta g^{\mu\nu}}.
\]
In our above setup, there are kinetic terms, so $T_{\mu\nu} \neq 0$ and the theory is not topological.
However, the theory does become topological in certain limits.

Suppose we add a gauge field $A_\mu$ transforming (under local gauge transformations) according to $A_\mu \mapsto A_\mu + d\lambda$.
Taking $\phi$ to be a complex scalar, setting $D\phi = d\phi + i A \phi$, and letting $F = dA$, we can define a Lagrangian
\[
	\Lc = D \phi \wedge \star D\phi + \frac{1}{e^2} F \wedge \star F.
\]
This produces a non-topological gauge theory.

We can produce a topological theory by adding topological coupling terms: $\theta F \wedge F$ or $A \wedge d\phi \wedge F$.
These terms would have coefficients determined by the gauge theory.
Alternatively, we could add boundary terms.

\subsection{More remarks on Lagrangians}

The constraints depend on the choice of gauge group.
For example, having gauge group $\RR$ would allow for only local transformations (corresponding to exact 1-forms $d \lambda$).
Gauge group $\U(1)$ would allow more interesting global transformations (e.g.\ $\omega \in H^1(-; \ZZ)$ allowing $A_\mu \mapsto A_\mu + \omega$).

We can split a typical Lagrangian into a \emph{kinetic term}, an \emph{interaction term}, and a \emph{topological term}.
The kinetic term reflects the behavior of free fields.
The interaction term consists of a polynomial in the $\phi_\alpha$, and the signs of the coefficients are constrained by locality and causality.
The topological term is (broadly) independent of the metric.
