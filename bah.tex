\chapter{Ibou Bah -- Lagrangian Methods for Categorical Symmetries}

\section{6/11 -- What is QFT?}

This will be a physics-y, intuitive series of lectures.
We will start by giving some idea of QFT in general.
In the second lecture, we will move on to finite symmetry groups and use these to construct topological operators.
In the third lecture, we will discuss the action of finite symmetry groups on fields (with an emphasis on the example of $\SU(N)$ Yang-Mills theory).
Finally, we will discuss the general construction of topological operators using Lagrangians.

\subsection{What should a QFT do?}

We don't actually know what a QFT \emph{is}.
However, we have an excellent physical idea of what a QFT should \emph{do}.
QFTs should have:
\begin{itemize}
	\item A list of operators: point (local) operators $\{ \Oc_i(x) \}$, line operators $\{ U_\alpha(\gamma_p) \}$, and other operators.
	\item A Hilbert space $\Hc$ of states on spacetime $M = \RR_t \times Y$.
	\item A partition function sending combinations / products of operators to complex numbers, e.g.\ $\langle \Oc_i \Oc_j \rangle$.
\end{itemize}
These should satisfy three key properties: \emph{unitarity}, \emph{locality}, and \emph{causality}. 
Let's discuss each of these in more detail.

\subsection{Unitarity, locality, and causality}

Unitarity can be interpreted as ``preservation of probability.''
For example, for many QFTs, we can construct an ``$S$-matrix'' governing scattering amplitudes of particles, and unitarity forces $S^\dagger S = 1$.
This specific interpretation doesn't work for CFTs / TFTs.
Instead, we use a notion of \emph{reflection positivity}.

To explain this, recall that time evolution in quantum mechanics is governed by
\[
	\ket{\psi} \mapsto e^{-i H t} \ket{\psi(t)}
\]
and
\[
	\Oc(x, 0) \ket{\psi} \mapsto \Oc(x, t) \ket{\psi}.
\]
We should have
\[
	\bra{\psi} \Oc^\dagger \Oc \ket{\psi} = \norm{\Oc(x, t) \ket{\psi}} \geq 0 \forall \Oc.
\]
To make sense of $\Oc^\dagger$, we allow our time variable to take complex values.
Say that $\Oc(x, t + i \tau)^\dagger = \Oc(x, t - i \tau)$.
Then the above condition forces $\langle \Oc(x, t - i \tau) \Oc(x, t + i \tau) \rangle \geq 0$.
Taking $t = 0$, we see that reflection positivity requires $\langle \Oc(x, -i \tau) \Oc(x, i \tau) \rangle \geq 0$.
More generally, reflection positivity forces
\[
	\langle \Oc_i(x, -i \tau) \Oc_j(x, -i \tau) \dots \Oc_j(x, i \tau) \Oc_i(x, i \tau) \rangle \geq 0.
\]

Locality forces point operators to have an \emph{operator product expansion}
\[
	\Oc_i(x_1) \Oc_j(x_2) = \sum_k c_{ij}^k(x_1, x_2) \Oc(x_1)
\]
governing the behavior as $x_1 \to x_2$.
(For conformal field theories, this sum is well-defined and the $c_{ij}^k$ should be meromorphic. For a generic QFT, this should be taken to be an asymptotic series.)
More generally, the interaction of operators should depend only on their behavior in a common tubular neighborhood. This enforces that ``physics happens'' only near the insertions.
If two point operators are outside the same light cone, then we can find a reference frame where the two operators take place at the same time but at different points in space.

Finally, causality requires that any operators which are spacelike separated commute with each other.
If two point operators are in the interior of the same light cone, then we can find a reference frame where the operators take place at the same point in space but at different points in time.
Things are more complicated for extended operators.

\subsection{Other features}

In many QFTs, we can expect other nice properties to hold:\footnote{There's also a last property of ``analyticity'' which is hard to formulate.}
\begin{itemize}
	\item Cluster decomposition:
	\[
	\lim_{b \to \infty} \langle \Oc_i(x_1) \Oc_j(x_2) \dots \Oc_{i'}(x_1 + b) \Oc_{j'}(x_2 + b) \dots \rangle = \langle \Oc_i(x_1) \Oc_j(x_2) \dots \rangle \langle \Oc_{i'}(x_1 + b) \Oc_{j'}(x_2 + b) \dots \rangle.
	\]
	\item Renormalization: high energy / short distance / UV physics and low energy / long distance / IR physics decouple, at least up to finitely many parameters.
\end{itemize}
Axiomatic treatments can take these as additional axioms. We can also show that cluster decomposition follows from locality assuming an additional Lorentz structure. 

We may think of the renormalisation property as cluster decomposition in the momentum space.

Gravity fails both of these conditions and locality.
Black holes are made of high energy states (UV phenomena) which are manifest at low energy. Furthermore, black holes also break cluster decomposition.
Whatever quantum gravity is, unitarity ``must'' be retained, since it's fundamental to quantum physics.

The goal of the Simons collaboration is to understand the kinematics of operators in QFT.
This is governed by symmetries!
We can essentially think of this as understanding the \emph{labels} of operators $\Oc_i(x)$, $U_\alpha(\gamma)$, etc.

\subsection{Where do QFTs come from?}

There are many sources of QFTs.

\begin{ex}[Lattices]
	Consider a lattice at which each point behaves according to quantum mechanics.
	Taking a continuum limit of the lattice gives a quantum field theory.
	This limit leads to the appearance of infinities and other strange phenomena.
	Nevertheless, it's physically the most ``rigorous'' way of constructing QFTs.
\end{ex}

\begin{ex}[Geometric engineering]
	Start with a string theory, and ``decouple gravity'' by freezing some dimensions.
	This produces a QFT involving geometric objects.
	Class $\Sc$ theories arise in this way.
\end{ex}

\begin{ex}[Lagrangians]
	We can construct many QFTs from Lagrangians.
	We'll spend most of our time focusing on this perspective.
\end{ex}

Each perspective has advantages over the others:
\begin{itemize}
	\item Lattices make renormalization straightforward.
	\item Geometric engineering makes the geometric nature of objects more apparent.
	\item Lagrangians make locality explicit.
\end{itemize}
	
\subsection{Lagrangians}

A Lagrangian is a functional on the fields $\phi_\alpha(x)$ and their derivatives:
\[
	L(x) = L(\phi_\alpha(x), \partial \phi_\alpha(x), \dots)
\]
This is a classical object.
Locality enforces that $L$ only depends on the behavior of the fields at the given input point (unlike, say, Green's functions $G(x_1, x_2)$).
Locality also ensures that there are only finitely many derivative terms appearing in $L$.

Lorentz symmetry (which we will assume implicitly) tells us that our fields can be:
\begin{itemize}
	\item Scalar fields $\phi(x)$ (spin $0$)
	\item Vector fields / gauge fields $A_\mu(x)$ (spin $1$), either:
		\begin{itemize}
			\item Massless: living in the representation $(1, -1)$, or
			\item Massive: living in $(1, 0, 1)$
		\end{itemize}
	\item Fermions $\psi_a$ (representations of a Clifford algebra)
	\item Metrics $g_{\mu \nu}$ (spin $2$)
\end{itemize}
We won't consider gravitational theories, so we'll ignore $g_{\mu \nu}$.

\section{6/11 -- Lagrangians in QFT}

Unitarity is true for closed quantum systems.
For open quantum systems, where we ``trace away'' some information, we can drop this assumption.
We must restore unitarity as we restore data to our system.
This explains e.g.\ where the $W$-boson and Higgs boson come from: they're needed to restore unitarity in the Standard Model.

\subsection{Lagrangians}

Consider a Lagrangian
\[
	\Lc(x) = \Lc(\phi_\alpha(x), \partial \phi_\alpha(x), \dots).
\]
From this (classical) data, we can define a (quantum) partition function
\[
	Z = \int D[\phi_\alpha] e^{i \int_M \sqrt{g} \Lc(\phi_\alpha)}
\]
This can be thought of as a formal expression: heuristically, it is a sum over all possible field configurations.

We can obtain correlation functions, etc.\ by inserting local operators
\[
	\Oc_i(\phi_\alpha(x), \partial \phi_\alpha(x), \dots)
\]
or extended operators
\[
	U_a(\gamma) = U\left(\int_\gamma F(\phi_\alpha(x)), \dots\right).
\]
Namely, we declare
\[
	\langle \Oc_i \dots \rangle = \frac{1}{Z(0)} \int D[\phi_\alpha] e^{i \int \Lc} \Oc_i \dots.
\]

\subsection{Parameters}

In general, it is possible that our Lagrangian depends on some additional parameters, say $\tau_a$, $\tau_b$.
The partition function then also depends on the $\tau$'s.
We will assume that these parameters are \emph{perturbative}: that small changes in our theory are governed by Taylor expansions in terms of the $\tau$'s.
Often, there is a large parameter space for a given QFT, and there are many perturbative regions within this space corresponding to different weakly coupled versions of the theory.
These \emph{duality frames} can have different fields, different Lagrangians, and different gauge groups.
(Gauge symmetry is a redundancy in parametrization -- it is not a physical symmetry, but rather a ``trick'' enabling the use of local Lagrangians.)

However, the global symmetry, Hilbert space, and local operators are the same in every duality frame.\footnote{The term ``global'' is not related to this, however.}
This may not always be visible (at least at the classical level) but it is true.
Global symmetry permutes the operators, while gauge symmetry does nothing to the operators.

\begin{ex}
	We can view 4d $\Nc = 4$ super-Yang-Mills as part of the same parameter space as type IIB supergravity on $\mathrm{AdS}_5 \times S^5$.
	Thus these are versions of the same theory.
\end{ex}

In addition to varying our parameters, we can also turn on background fields $B_i(x)$, i.e.\ add in new fields which do not fluctuate.

\subsection{Examples}

Consider a theory with one free massless scalar field $\phi$.
The Lagrangian is
\[
	\Lc = \frac{1}{e^2} d \phi \wedge \star d \phi,
\]
a single kinetic term.
The partition function is
\[
	Z = \int D[\phi] \exp\left( \frac{i}{e^2} \int d\phi \wedge \star d\phi \right).
\]
Taking the coupling parameter $e \to 0$ yields $d\phi = 0$.

We can also add matter, incorporated via a term $m^2 \phi^2 \Vol(M)$ added to the Lagrangian.
Taking $m \to \infty$ gives $\phi = 0$.

The stress tensor of the theory is
\[
	T_{\mu\nu} = \frac{\delta \Lc}{\delta g^{\mu\nu}}.
\]
In our above setup, there are kinetic terms, so $T_{\mu\nu} \neq 0$ and the theory is not topological.
However, the theory does become topological in certain limits.

Suppose we add a gauge field $A_\mu$ transforming (under local gauge transformations) according to $A_\mu \mapsto A_\mu + d\lambda$.
Taking $\phi$ to be a complex scalar, setting $D\phi = d\phi + i A \phi$, and letting $F = dA$, we can define a Lagrangian
\[
	\Lc = D \phi \wedge \star D\phi + \frac{1}{e^2} F \wedge \star F.
\]
This produces a non-topological gauge theory.

We can produce a topological theory by adding topological coupling terms: $\theta F \wedge F$ or $A \wedge d\phi \wedge F$.
These terms would have coefficients determined by the gauge theory.
Alternatively, we could add boundary terms.

\subsection{More remarks on Lagrangians}

The constraints depend on the choice of gauge group.
For example, having gauge group $\RR$ would allow for only local transformations (corresponding to exact 1-forms $d \lambda$).
Gauge group $\U(1)$ would allow more interesting global transformations (e.g.\ $\omega \in H^1(-; \ZZ)$ allowing $A_\mu \mapsto A_\mu + \omega$).

We can split a typical Lagrangian into a \emph{kinetic term}, an \emph{interaction term}, and a \emph{topological term}.
The kinetic term reflects the behavior of free fields.
The interaction term consists of a polynomial in the $\phi_\alpha$, and the signs of the coefficients are constrained by locality and causality.
The topological term is (broadly) independent of the metric.

\section{6/12 -- Introduction to BF Theory}

We start by clarifying something from last time.

\subsection{Jumping global symmetry}

The global symmetry can jump as we move throughout the moduli space of theories.

\begin{ex}
	Suppose we have a theory with two scalar fields $\phi_1$ and $\phi_2$ and Lagrangian
	\[
		\Lc = \frac{1}{2} (\partial_\mu \phi_i)^2 + m_1^2 \phi_1^2 + m_2^2 \phi_2^2.
	\]
	When $m_1 = m_2$, the symmetry of this theory increases to $\SO(2)$.
	This leads to new and interesting physics.
\end{ex}

\subsection{Definition of BF Theory}

BF theory is the TFT with fields $b$ (a $p$-form) and $c$ (a codimension $p+1$ form) and action $S = N \int_M b \wedge dc$ (for some fixed $N$).
The partition function is then
\[
	Z = \int D[b] D[c] \exp\left(-2\pi i N \int_M b \wedge dc \right)
\]
For specificity, we will take $d = 5$, letting $b$ and $c$ be 2-forms.
However, our discussion will make sense in any dimension.

We can allow different classes of gauge symmetries:
\begin{itemize}
	\item $R$-gauge: $b \mapsto b + d \lambda_1$
	\item $\U(1)$ gauge: $b \mapsto b + \omega$ for $\omega \in H^2(M; \ZZ)$
	\item Trivial gauge: $b \mapsto b + \omega$ for $\omega \in H^2(M; \RR)$.
\end{itemize}
We may consider independent transformations for $c$.

BF theory has observables (where $\gamma'$ has codimension $p+1$ and $\gamma$ has dimension $p$):
\begin{align*}
	W_c(\gamma) &= \exp\left(2\pi i \oint_\gamma c\right) \\
	W_b(\gamma') &= \exp\left(2\pi i \oint_{\gamma'} b\right).
\end{align*}
These operators have different behaviors with respect to different gauge symmetries:
\begin{itemize}
	\item $R$-gauge: operators are gauge invariant for all $b$ (or $c$).
	\item $\U(1)$ gauge: $b$ (or $c$) must have holonomy in $\U(1)$.
	\item Trivial gauge: none of these observables are gauge invariant.
\end{itemize}
We will focus on the case of $\U(1)$ gauge symmetry.

Note that BF theory doesn't have any ``iterated gauge symmetries'' or ``St\"uckelberg fields.''
Thus we do not need to worry about some of the more complicated aspects of gauge theories.

\subsection{Linking numbers from correlation functions}

We'd like to compute
\begin{align*}
	\langle W_c(\gamma) W_b(\gamma) \rangle &= \int D[b] D[c] \exp\left(-S + 2\pi i \int_\gamma c + 2\pi i \int_\gamma b\right) \\
						&= \int D[b] D[c] \exp\left(-i N 2\pi (b \wedge dc + b \wedge dc_0) + 2\pi i \int_\gamma (c + c_0) + 2\pi i \int_{\gamma'} b\right) \\
						&= \int D[b] D[c] \exp\left(-i N 2\pi b \wedge dc + 2\pi i \int_\gamma c\right) \exp\left( \frac{2\pi i}{N} \int_\gamma c_0 \right) \\
						&= Z \exp\left(\frac{2\pi i}{N} L(\gamma, \gamma')\right)
\end{align*}
where $L(\gamma, \gamma')$ is the linking number of $\gamma$ and $\gamma'$.

Under $\gamma \mapsto \gamma + N \beta$, we have $W_c(\gamma) \sim W_c(\gamma + N \beta)$, and thus $W_c^N(\gamma) = 1$.
The analogous statement holds for $W_b(\gamma')$.
The upshot is that the theory actually has a $\ZZ_N \times \ZZ_N$ gauge symmetry, coming from the maps
\[
	\begin{tikzcd}
		H^1(-; \ZZ_N) \rar & H^1(-; \U(1)) \rar & H^1(-; \U(1))
	\end{tikzcd}
\]
where the last map is induced by $a \mapsto a^N$.

We can write
\[
	W_c(\gamma) W_b(\gamma') = \exp\left(\frac{2\pi i}{N} L(\gamma, \gamma')\right) W_b(\gamma') W_c(\gamma).
\]
This can be read as saying that the operators $W_c(\gamma)$ are topological operators generating a $\ZZ_N$ global symmetry, where the charged objects are the $W_b(\gamma')$.
The corresponding statement with the roles of $b$ and $c$ reversed is also true.

\subsection{Quantizing BF theory}

Classically, BF theory seems trivial: the equations of motion for $b$ give $N dc = 0$, and the equations of motion for $c$ give $N db = 0$.
Thus classical solutions are just flat connections.
Interesting behavior only appears at the quantum level.

Canonical quantization for $b$ and $c$ gives
\[
	[b(x), c(y)] = \frac{2\pi i}{N} \delta(x - y) \Vol(M).
\]
We can write
\[
	e^{2\pi i \int_\gamma b} e^{2\pi i \int_{\gamma'} c} = e^{\left[2\pi i \int_\gamma b, 2\pi i \int_{\gamma'} c\right]} e^{2\pi i \int_{\gamma'} c} e^{2\pi i \int_\gamma b}
\]
Combining this with the above construction of commutators recovers our previous results.

\subsection{Cheeger-Simons maps for more general gauge theories}

In general, consider a $p$-form $A$ with gauge symmetry $A \mapsto A + \omega$ for $\omega \in H^p(M; \ZZ)$.
We have a natural operator given by the Cheeger-Simons map
\[
	\chi(\Sigma_p) = \exp\left(2\pi i \int_{\Sigma_p} A\right).
\]
Field configurations for which $\chi$ is gauge-invariant are classified by the differential cohomology $\check{H}^{p+2}(M)$, which fits into the hexagon
\[
	\begin{tikzcd}
		& & & & 0 \\
		& \Omega^p / \Omega^p_{\ZZ} \ar[rr, "d"] \ar[dr] & & \Omega^{p+1} \ar[ur] \ar[dr] & \\
		H^p(M; \RR) \ar[ur] \ar[dr] & & \check{H} \ar[ur] \ar[dr] & & H^{p+1}(M; \RR) \\
		& H^p(M; \RR / \ZZ) \ar[rr, "\beta"] \ar[ur] & & H^{p+1}(M; \ZZ) \ar[ur] \ar[dr] & \\
		0 \ar[ur] & & & & 0 \\
	\end{tikzcd}
\]
This is a lot larger than what one would na\"ively expect.
Things are much simpler in BF theory where we only consider flat connections.
One can see this written up in a paper of Freed, Moore, and Segal.

\section{6/12 -- Global symmetries in variants of BF theory}

\subsection{Simultaneous BF theories}

Let's see what happens when we combine two versions of BF theory at once.
Consider the 5d theory with fields $b_2$, $c_2$, $A_1$, and $c_3$ and action
\[
	S = 2 \pi i \int N b_2 \wedge dc_1 + 2N A_1 \wedge dc_3.
\]
The first term gives a $\ZZ_N^{(1)} \times \ZZ_N^{(1)}$ gauge symmetry, and the second term gives a $\ZZ_{2N}^{(0)} \times \ZZ_{2N}^{(2)}$ gauge symmetry.
From the operators $W_c$ and $W_b$ corresponding to the first term, we get $\ZZ_N^{(2)}$ and $\ZZ_N^{(2)}$ global symmetries.
From the (similarly defined) operators $V_A$ and $V_{c_3}$, we get $\ZZ_N^{(3)}$ and $\ZZ_N^{(1)}$ global symmetries.

\subsection{Interacting terms}

Can we get non-invertible global symmetries from something like this
Yes: add in non-dynamical fields $\phi$ and $\beta$, and add an interaction term to the action:
\[
	S = 2 \pi i \int N b_2 \wedge dc_1 + 2N A_1 \wedge dc_3 + \frac{1}{2N} (d\phi - 2 N A) \wedge (d\beta - N b)^2.
\]
This forces the gauge symmetry groups to interact with each other.

The new equations of motion include
\[
	2N dc_3 - (d\beta - Nb)^2 = 0.
\]
This equation implies that our original $V_{c_3}$ is no longer a topological operator.
We seek to correct this somehow.
If $\Sigma_3$ is a boundary, say $\Sigma_3 = \partial Y$, then we can require
\[
	V_{c_3}(\Sigma_3) = \exp\left(\frac{1}{2} N \int_Y b^2\right).
\]
In general, we can define a topological operator
\[
	V_{c_3}(\Sigma_3) = \int d[a] \exp\left(2\pi i \int_{\Sigma_3} c_3 + 2\pi i \int_{\Sigma_3} N a \wedge da + a \wedge b\right)
\]
Essentially, we have an anomaly on the bulk (breaking the $\ZZ_N^{(2)}$ symmetry), and we define a new TQFT on the defect to cancel this out.
This is ``stacking.''
There are other theories we could use to cancel the anomaly, but the above method is ``minimal.''

\subsection{Fusion rules}

We can compute some fusion rules of this new $V$ operator:
\begin{align*}
	V(\Sigma_3) \otimes V^\dagger(\Sigma_3) &= \int d[a] d[a'] \exp\left(2\pi i \int_{\Sigma_3} N (a da - a' da') + (a - a') \wedge b\right) \\
						&= \int d[\alpha] d[\alpha'] \exp\left(2\pi i N \int_{\Sigma_3} \alpha' \wedge d\alpha + \alpha d\alpha + \alpha \wedge b\right)
\end{align*}
or something similar.
Ultimately, this localizes to 
\[
	V \otimes V^\dagger = A(\Sigma_3) \sum_{b \in H^2(\Sigma_3; \ZZ_N)} e^{2\pi i \int_{\Sigma_3} b}.
\]
for some TQFT coefficient $A(\Sigma_3)$ (depending only on $\Sigma_3$, not on the bulk).
The sum here is a ``condensation defect.''
Working this out precisely is left to the reader as an exercise (though if anyone reading this has all the details, feel free to add them).

The upshot is that the Lagrangian approach allows us to understand much of what's going on with categorical symmetries.
We can even get the TQFT coefficients out of this.
Adding more interaction terms can give even more complicated fusion rules (which are still accessible via the Lagrangian presentation).

\subsection{Super Yang-Mills}

In general, we can consider the action of a 5d symmetry TFT on 4d $\SU(N)$ or $\PSU(N)$ super-Yang-Mills.
This 4d theory has interesting topological operators: Wilson lines and 't Hooft lines.
Fully understanding the bulk TFT allows us to work with the symmetries of the boundary theory.
