\chapter{Ryan Thorngren -- Symmetry Breaking, Higher Berry Phase, and the Long Exact Sequence}

These talks are based on joint work with A.\ Debray, S.\ Devalapurkar, C.\ Krulewski, Y.\ L.\ Liu, N.\ Pacheco-Tallaj, available at \url{https://arxiv.org/abs/2309.16749} and \url{https://arxiv.org/abs/2405.04649}.
A sketch of what we'd like to discuss: 
\begin{enumerate}
	\item Symmetry breaking defects including $\rho$-defects, domain walls, and vortices
	\item Gapless modes and defect / bulk anomaly matching
	\item Family anomalies, higher Berry phase, phase diagrams
	\item Incorporating symmetries, index theory
	\item Long exact sequence connecting everything
\end{enumerate}

\section{6/13 -- Symmetry Breaking and Defects}

\subsection{Symmetry breaking}

There are multiple types of ``symmetry breaking'' in physics.
We'll consider spontaneous symmetry breaking and explicit symmetry breaking.

In \emph{spontaneous symmetry breaking}, there is a global symmetry (e.g.\ a Lie group $G$), but the ground states are not symmetric (or perhaps their symmetries lie in a subgroup $H \subset G$).

\begin{ex}
	Consider four vertices which lie in a square.
	We'd like to connect the vertices by a collection of strings of shortest total length.
	There are two solutions, both somewhat between an H shape an X shape.
	Each solution has $D_4$ symmetry rather than the $D_8$ symmetry of the point configuration.
\end{ex}

\begin{ex}
	We'd like to minimize the potential $V = -x^2 + x^4$ on $\RR$.
	The potential is preserved by the $\ZZ_2$ reflection symmetry of $\RR$.
	However, this swaps the minima, so the symmetry is broken to the trivial group.
\end{ex}

In \emph{explicit symmetry breaking}, a system which has a symmetry $G$ is subjected to small perturbations breaking the $G$-symmetry.

\begin{ex}
	In the last example, we can change our potential to $V = \epsilon x - x^2 + x^4$.
	This breaks the symmetry between the minima.
\end{ex}

\subsection{Spatial locality}

\begin{ex}
	Consider the problem of minimizing
	\[
		E = \int [(\nabla \phi)^2 - \phi^2 + \phi^4] dx.
	\]
	This is just given by $\phi$ taking a constant value at a minimum of $V(x) = -x^2 + x^4$.
	Let's make things more interesting and assume we're doing thermodynamics or quantum physics, so that we want to understand ``nearly minimal'' configurations as well.
	This allows $\phi$ to flip between different minima along \emph{domain walls}.
\end{ex}

Define the \emph{correlation function}
\[
	\langle \phi(0) \phi(x) \rangle_c = \langle \phi(0) \phi(x) \rangle - \langle \phi(0) \rangle \langle \phi(x) \rangle.
\]
The value of this is determined by the presence or absence of domain walls.

\begin{ex}
	Let $G = \U(1)$ and $H = 1$.
	Consider a complex scalar field $\phi: X \to \CC$ and the potential $V = -\abs{\phi}^2 + \abs{\phi}^4$.
	For $\phi = re^{ikx}$, we get energy scaling with $k$.
	These configurations are known as \emph{Goldstone modes}.

	For $\phi = z$, we get logarithmic energy, with $\partial_\theta \phi$ and $(\partial_\theta \phi)^2$.
	This spoils the order in 2d, and we get ``quasi-long-range order'' where the correlation function decays according to a power law.
\end{ex}

\subsection{Symmetry defects}

Suppose that our symmetry group $G$ acts transitively on the space of ground states (which can then be identified with $G / H$).
For spontaneous symmetry breaking, codimension $k$ defects correspond to $\pi_{k-1}(G / H)$.

Explicit symmetry breaking is given by a linear perturbation living in a representation $\rho$ of $G$.

\begin{ex}
	For $V = \epsilon x - x^2 + x^4$, the perturbation transforms according to the sign representation of $\ZZ_2$.
\end{ex}

More generally, we can write $\Delta V = \int d^d x \epsilon_a \Oc^a(x)$.

\begin{ex}
	For
	\[
		E = \int d^d x [(\nabla \vec{\phi})^2 - \norm{\vec{\phi}}^2 + \norm{\vec{\phi}}^4 + \vec{\epsilon} \cdot \vec{\phi},
	\]
	the perturbation transforms with the vector representation of $\Orm(N)$.
\end{ex}

We can construct a ``$\rho$-defect'' (for $\dim \rho = k$) as
\[
	\epsilon_a(y_1, \dots, y_k, x_{k+1}, \dots, x_d) = f\left(\sum_{i=1}^k y_i^2\right) \cdot \frac{y_a}{\abs{y_a}}
\]
for some scalar-valued $f$, typically growing linearly near zero and with or without a cutoff at infinity.
This $\epsilon$ ``winds around a $(k-1)$-sphere $S(\rho) \subset \rho$.''
The defect has codimension $k$.

The explicit symmetry breaking defects we've seen so far are examples of $\rho$-defects.

\subsection{Jackiw-Rebbe zero modes}

Consider 1+1d QFT with (real or complex) two-component fermions.
Say $\gamma^0 = i Y$ and $\gamma^1 = X$.
We take the Lagrangian
\[
	\Lc = i \psi^\dagger \gamma^0 \gamma^\mu \psi + i m \psi^\dagger \gamma^0 \psi.
\]
The equations of motion are
\[
	(\gamma^\mu \partial_\mu + m) \psi = 0,
\]
so we get solutions 
\[
	\psi^i(x, t) = u^i(E, k) e^{i E t + i k x}
\]
where $E = \sqrt{k^2 + m^2}$.

The lowest energy state ($k = 0$) has energy $m$ -- we call this a \emph{mass gap}.
Letting our mass vary with space according to an odd function, we get gapped systems both left and right of the origin.
We get a 0-dimensional $\rho$-defect (with $\rho$ the sign representation) for time-reversal symmetry $\psi \mapsto \gamma^0 \psi$ and $m \mapsto -m$.

If $m(x) = x$, the equations of motion
\[
	(\gamma^\mu \partial_\mu + x) \psi = 0
\]
have a time-independent solution $\psi_{\pm} = u_{\pm} e^{-\mp x^2/ 2}$ where $\gamma^1 u_{\pm} = \pm u \pm$.
With the $+$ sign, this is normalizable; with the $-$ sign, this is not.
Somehow this ties in to anomalies.

\begin{exer}
	For a 2+1d theory with real 2-component fermions and $\gamma^2 = Z$, show that for
	\[
		\psi^i(t, x_1, y_2) = f(x_1) \chi^i(t, y_2),
	\]
	we can choose $f(x_1)$ such that $\chi^i$ is chiral and satisfies the massless Majorana equation.
\end{exer}

\begin{exer}
	For a 3+1d theory with complex 2-component Weyl fermions, considering
	\[
		x_1 \bar{\psi} \psi + x_2 i \bar{\psi} \gamma^3 \psi,
	\]
	there is a real chiral massless fermion along the vortex line.
\end{exer}

We'd like to understand how these behave as we add deformations.

\section{6/13 -- ('t Hooft) Anomalies}

\subsection{General theory}

Suppose we have a quantum system with $D$-dimensional spacetime $X$ and background $G$-gauge field / symmetry $A$.
In general, $Z(X, A)$ transforms via
\[
	Z(X, A^g) = e^{i \Omega(X, A, g)} Z(X, A).
\]
One can think of $Z$ as a \emph{section of a line bundle} on the space of connections on $X$ (rather than as a \emph{function}) on said space).
Stated better, one may view $\ket{Z(X, A)}$ as an element of a Hilbert space of an invertible $(D+1)$-dimensional TQFT on $X \times [0, 1]$.
Here $X \times [0, 1]$ (with $e^{i \Omega(\dots)}$) connects $X \times 0$ (with $Z(X, A)$) to $X \times 1$ (with $Z(X, A^g)$).

``Cobordism theory'' tells us that these form a group $\Omega^{D+1}_G$.
The anomaly TQFTs are determined by their partition functions $\omega(Z^{D+1}, A)$.
Furthermore, $e^{i \omega(\dots)}$ behaves like the holonomy of a connection.
The actual topological quantity here is the Chern number of this connection.
We get a universal coefficient sequence
\[
	\begin{tikzcd}
		1 \rar & \Omega_{D+1}^{G,\mathrm{tors}} \rar & \Omega^{D+1}_G \rar & \Hom(\Omega_{D+2}^G, \ZZ) \rar & 1
	\end{tikzcd}
\]
Here $\Omega_r^G$ is the bordism group of $r$-manifolds with $G$-connection.
As with the usual universal coefficient sequence, the above sequence splits, but not naturally.

\subsection{Adding structure}

We'd like to consider manifolds with more structure.
For bosonic theories, we want an orientation.
For fermionic theories, we want a spin structure.
We'll refer to these and similar topological structures as \emph{$s$-structures} on the tangent bundle $TX$.
In general, we'll ``twist'' these as follows.

Take a \emph{twisting vector bundle} $\eta$ on $BG$, and view our gauge field as giving $A: X \to BG$.
We say that an \emph{$\eta$-twisted $s$-structure} is an $s$-structure on $TX \oplus A^* \eta$.

The class $w_1(\eta): G \to \ZZ_2$ controls which elements reverse the orientation of spacetime.

\begin{ex}
	Let $G = \U(1)$, let $s = \SO$ be orientation structure, and let $\eta$ be trivial.
	Then
	\[
		\Omega^3_{\U(1),\SO} = \Hom(\Omega+4^{\U(1),\SO}, \ZZ) = \ZZ^{c_1^2} \oplus \ZZ^{p_1}.
	\]
\end{ex}

In general, we have a useful map $H^{D+2}(BG; \ZZ) \to \Omega^{D+1}_{G,s}$.

\begin{ex}
	Majorana zero modes are associated with the Arf invariant in $\Omega^2_{\Spin} = \ZZ_2$.
\end{ex}

\begin{ex}
	Consider time reversal symmetry with $T^2 = 1$.
	Let $w_2(\eta) = 0$ and $w_1(\eta) \neq 0$ -- say $\eta$ is the sign representation of $\ZZ_2$.
	Note
	\[
		0 = w_1(TX \oplus A^* \eta) = w_1(TX) + w_1(A^* \eta) = w_1(TX) + A
	\]
	and
	\[
		0 = w_2(TX \oplus A^* \eta) = w_2(TX) + w_1^2(TX).
	\]
	We end up with $\Pin^-$ appearing somehow.
	Note $\Omega^2_{\Pin^-} = \Omega^2_{\ZZ_2,\Spin,\sgn} = \ZZ_8$.
\end{ex}

\subsection{Defect anomaly matching}

Consider a gauge group $G$ with a representation $\rho$.
Assume our theory has operators $\Oc^a$ transforming with $\rho$ such that, for large $\epsilon_a \in \rho$, the theory perturbed by $\epsilon_a \Oc^a$ is non-degenerately gapped.
We call such a theory \emph{$\rho$-gappable}.
This produces a $\rho$-defect local $(D-k)$ dimensional system.
How can we relate the anomalies with the bulk?

On $X^{D+1}$ with background field $A$, we want to choose a section $\phi$ of $A^*\rho$.
Let $Y = \phi^*(0)$; for generic $\phi$, this is a codimension $k$ submanifold.
The theory is trivialized away from $Y$, so we can write
\[
	\omega(X^{D+1}, A, \xi) = \alpha(Y^{D-k+1}, A|_Y, \xi|_Y).
\]
This lets us reconstruct the bulk anomaly $\omega$ from the defect $\alpha$.
Here $\xi$ is an $\eta$-twisted $s$-structure on $TX$.

Note that $TX|_Y = TY + NY$, where $NY$ is the vector bundle associated to $A^*\rho|_Y = A|Y^* \rho$.
Thus $\xi|_Y$ is actually an $(\eta + \rho)$-twisted $s$-structure.

We end up with a linear map
\[
	\Def_\rho: \Omega^{D-k+1}_{G,s,\eta+\rho} \to \Omega^{D+1}_{G,s,\eta}.
\]
This satisfies $\Def_\rho \alpha = \omega$.
Note that, if this map is not injective, then $\alpha$ is not uniquely defined by $\omega$.
Similarly, if the map is not surjective, there may not be an $\alpha$ corresponding to $\omega$.
(This latter case is covered by the possibility that the theory is not $\rho$-gappable.)

The extra twisting here arises from the fact that, in the presence of a $\rho$-defect, we don't really have $G$-symmetry.
One can understand this in examples involving CRT symmetry.

\begin{ex}
	Consider a 3+1d complex 4-component Dirac fermion $\psi_L, \psi_R$ with $\U(1)_L$ background field.
	Take $\eta$ to be trivial.
	We get an anomaly
	\[
		\frac{1}{6} c_{1,L}^3 - \frac{1}{24} c_{1,L} p_1(TX^6).
	\]
	Let $\rho$ be the charge 1 representation.
	Take a section $\phi$ of the bundle associated with $A^* \rho$; then $Y = \phi\inv(0)$ is Poincar\'e dual to $c_{1,L}$.
	We can write
	\[
		\int_{X^6} \frac{1}{6} c_{1,L}^3 - \frac{1}{24} c_{1,L} p_1(TX^6) = \int_Y \frac{1}{6} c_{1,L}^2 - \frac{1}{24} p_1(TX).
	\]
	We can write $p_1(TX)|_Y = p_1(TY \oplus A^* \rho) = p_1(TY) + p_1(A^* \rho)$.
	As $\CC \otimes A^* \rho = A^* \rho \oplus \ol{A^* \rho}$, we get $p_1(TX)|_Y = p_1(TY) + c_{1,L}^2$.
	It follows that the defect anomaly is
	\[
		\frac{1}{8} c_1^2 - \frac{1}{24} p_1(TY) \in \Omega^3_{\Spin^c}.
	\]
	This matches the anomaly of the 1+1d Weyl fermion with charge 1 under $\U(1)$.
\end{ex}

\section{6/14 -- Defect Anomalies and Families}

\subsection{Review}

Last time, we constructed a defect anomaly map
\[
	\Def_\rho: \Omega^{D-k+1}_{G,s,\eta+\rho} \to \Omega^{D+1}_{G,s,\eta}.
\]
This sends the anomaly of the $\rho$-defect to the bulk anomaly.

\begin{ex}
	For a 3+1d $\U(1)$-symmetric Dirac fermion, where the map is
	\[
		\Def_\rho: \Omega^3_{\Spin^c} \to \Omega^5_{\U(1),\Spin}.
	\]
	We can show that this is an isomorphism
	\[
		\ZZ^{c_1^2} \oplus \ZZ^{c_1^2 / 8 - p_1 / 24} \to \ZZ^{c_1^3 / 6 - c_1 p_1 / 24} \oplus \ZZ^{c_1^3}.
	\]
\end{ex}

More generally, we want to understand the kernel and image of $\Def_\rho$.

\subsection{Families of QFTs}

\begin{ex}
	Consider a 1+1d Dirac fermion with
	\[
		a \ol{\psi} \psi + i b \ol{\psi} \gamma^c \psi.
	\]
	Let's understand the behavior of the theory as a function of $(a, b)$.
	At $a = b = 0$, this is gapless.
	Away from $(0, 0)$, this theory has trivial phase.

	We can ask whether the gapless point is protected, or whether it will disappear under small perturbations.
	With $\U(1)$ symmetry, this point is protected: as we move (``adiabatically slowly'') along a circle around the gapless point, we get a current $j = \partial_t \theta / 2\pi$.
	Moving around a full circle, one unit of charge gets ``pumped'' via the ``Thouless pump.''
	This charge is quantized, so we can't get rid of the gapless point.
\end{ex}

This is analogous to Berry phase.

\begin{ex}
	Consider a two-state system with Hamiltonian
	\[
		H = a \cdot \sigma^x + b \cdot \sigma^y + c \cdot \sigma^z.
	\]
	At $a = b = c = 0$, we have a twofold degenerate state.
	This is topologically protected: moving around it corresponds to a line bundle on $S^2$ with Chern number 1.
	In optics, such a point is called a ``diabolical point.''
\end{ex}

In QFT, we can also find nontrivial families without symmetry.

\begin{ex}
	Consider the 1+1d CFT $\SU(2)_1$ (where 1 is the ``level'').
	Add an $\SO(4)$ vector to break the symmetries and get a trivial theory.
	We get an interesting $S^3 \subset \RR^4$.
	We may take an effective field theory $\phi: X \to S^3$ which contains a Wess-Zumino--Witten theory at level 1.
	This theory has a quantized term protecting the behavior of our original theory at the origin.
	We may call this a ``higher Berry phase.''
\end{ex}

We want to generalize our cobordism theory to include such $\phi$.
Let $\Omega^D_{G,s,\eta}(M)$ classify cobordism invariants of $D$-manifolds $X$ with $G$-background field, $\eta$-twistes $s$-structure, and $\phi: X \to M$.
This can also be viewed as parametrizing $G$-symmetric theories of invertible theories.

\begin{ex}
	We can write $\Omega^2_{\U(1),\SO}(S^1) = \Omega^2_{\U(1),\SO} \oplus \Omega^1_{\U(1),\SO}$.
	Here $\Omega^1_{\U(1),\SO} = \ZZ$ is what gets pumped.
	We can write
	\[
		e^{i S_{\textrm{eff}}} = e^{i \int_{X^2} A \frac{d\phi}{2\pi}}
	\]
	where $j_x = \partial_t \phi / 2\pi$ is our current.
	Note that this is bosonic: we get similar phenomena in both bosonic and fermionic cases.
\end{ex}

Family anomalies happen at boundaries of nontrivial invertible families.

\begin{ex}
	Consider a Thouless pump with boundary (looking something like an interval).
	The current flows in one direction, accumulating charge at the endpoint.

	Suppose in particular that we have a level crossing along the ray $\theta = \pi$ in parameter space.
	At the origin, we have a bulk diabolical point.
	Away from the origin and $\theta = \pi$, we have a unique ground state.
	There are also two rays on which we have doubly degenerate ground states.
\end{ex}

The above is largely academic; the following is more exciting.

\begin{ex}
	Consider 1+1d QED with $N$ charge 1 bosons (giving $\CC\PP^{n-1}$), a $\theta$-angle, and $\PSU(N)$ flavor symmetry.
	For $N$ even, there is an anomaly at $\theta = \pi$.
	This corresponds to $u_3 \in H^3(B\PSU(N), \U(1)) = \ZZ_N$.

	For $N$ odd, we may choose $k$ such that $2k \equiv 1 \mod N$.
	We end up with $u_3 = d u_2 / N$, giving a counterterm $k u_2 / N$ canceling the anomaly.
	There's still a family anomaly 
	\[
		\frac{1}{N} \int \frac{d\phi}{2\pi} u_2(\PSU(N))
	\]
	coming from the fact that there's no choice of counterterm over the whole family.
\end{ex}

Similar stories can be told in other contexts.

\subsection{Explicit symmetry breaking in families}

In general, we want to study the explicit symmetry breaking family we get on a big sphere $S(\rho)$ in the representation $\rho$.
However, $G$ acts on $S(\rho)$ because we broke the symmetry.
Thus we don't have a family of symmetric theories on $S(\rho)$ but rather a ``$G$-equivariant family on $S(\rho)$.''
These can also be studied using bordism.

Say we have an anomaly-free $G$-equivariant $M$-family of invertible theories where $G \curvearrowright M$.
Via the cobordism hypothesis, this can be viewed as an invertible TQFT for manifolds equipped with a $G$-gauge field $A$, a (possibly $\eta$-twisted) $s$-structure $\xi$, and a section $\phi$ of $P_A \times^G M$.
This latter map is equivalent to a map $X \to M /_h G$, as can be seen from the pullback square
\[
	\begin{tikzcd}
		P_A \times_G M \rar \dar & M /_h G \dar \\
		X \rar & BG.
	\end{tikzcd}
\]
Such TQFTs are classified by $\Omega^D_{G,s,\eta}(M)$.

\section{6/14 -- The long exact sequence and applications}

\subsection{The long exact sequence}

Recall that we wanted to compute the kernel and image of the defect map $\Def_\rho: \Omega_{G,s,\eta+\rho}^{D-k+1} \to \Omega_{G,s,\eta}^{D+1}$.
This can be done using the \emph{symmetry breaking long exact sequence}
\[
	\begin{tikzcd}
		\dots \rar & \Omega_{G,s,\eta}^D(S(\rho)) \rar["\Ind_\rho"] & \Omega_{G,s,\eta+\rho}^{D-k+1} \rar["\Def_\rho"] & \Omega_{G,s,\eta}^{D+1} \rar["\Res_\rho"] & \Omega_{G,s,\eta}^{D+1}(S(\rho)) \rar & \dots
	\end{tikzcd}
\]
Here $\Ind_\rho$ is the \emph{index map} and $\Res_\rho$ is the \emph{residual family map}.

\subsection{Examples: Majorana fermions}

\begin{ex}
	Consider a 2+1d Majorana fermion $\psi$.
	We have a time reversal symmetry $T: \psi \mapsto \gamma^0 \psi$ with $T^2 = (-1)^F$.
	This has an anomaly in $\Omega^4_{\Pin^+} = \ZZ_{16}$.
	Taking $\rho$ to be the sign representation and considering $i m \ol{\psi} \psi$ with $T$ odd, we have two solutions $\psi_+$ (right-moving) and $\psi_-$ (left-moving) corresponding to eigenvectors of $\gamma^1$.
	Only one of these is normalizable, but the choice depends on the signs involved.
	We end up with a gravitational anomaly $\pm 1 \in \ZZ = \Omega^3_{\Spin}$, where the sign agrees with $\psi_{\pm}$.

	Since we are dealing with Majorana fermions, charge conjugation acts trivially, so $CRT$ sends
	\[
		\psi(x_1, y_2, t) \mapsto \gamma^1 \gamma^0 \psi(-x_1, y_2, -t).
	\]
	The symmetry on the defect is $U := RT^2: \psi \mapsto -\gamma^1 \psi$.
	We can write $\Omega^3_{\Spin,\ZZ_2} = \ZZ \oplus \ZZ_8$ where $\ZZ_8$ controls the number of right movers minus the number of left movers.
	The right-moving chiral mode has anomaly $(1, 1)$, and the left-moving mode has anomaly $(-1, 0)$.

	The defect map here is $\Def_\rho: \Omega^3_{\Spin,\ZZ_2} \to \Omega^4_{\Pin^+}$, and we can write this numerically as
	\begin{align*}
		\ZZ \oplus \ZZ_8 &\to \ZZ_{16} \\
		(a, b) &\mapsto -a + 2b.
	\end{align*}
	The kernel is $\ZZ \cdot (2, 1)$, looking like two right movers, one of which is charged.
\end{ex}

\begin{ex}
	Let's modify the preceding example to have $\psi_1$ and $\psi_2$ with $T: \psi_k \mapsto (-1)^k \gamma^0 \psi_k$.
	This theory is anomaly free.
	Here $i \ol{\psi}_1 \psi_1 - i \ol{\psi}_2 \psi_2$ lives in the sign representation.
	
	Solving this theory, we get two right movers, one of which is charged.
	For $m_1 > 0$ and $m_2 < 0$, we get $1 \in \ZZ = \Omega^3_{\Spin}$, corresponding to $p + ip$.
	For $m_1 < 0$ and $m_2 > 0$, we get $-1 \in \ZZ = \Omega^3_{\Spin}$, corresponding to $p - ip$.
	The symmetry exchanges the movers, so the eigenvalues must be $\pm 1$.
\end{ex}

\subsection{Index maps}

Let's consider $\Omega^D_{G,s,\eta}(S(\rho))$.
This classifies anomaly-free $G$-equivariant invertible families on $S(\rho)$.
The index map $\Ind_\rho: \Omega^D_{G,s,\eta}(S(\rho)) \to \Omega^{D-k+1}_{G,s,\eta+\rho}$ corresponds to compactification on $S(\rho)$.
We can compute
\[
	\Ind_\rho(\beta)(Y^{D-k+1},A,\xi) = \beta(Y^{D-k+1} \times_A S(\rho), \pi^* A, \pi^*\xi, \phi)
\]
where $\phi: Y^{D-k+1} \times_A S(\rho) \to S(\rho)$.

\begin{ex}
	Returning to our example of two Majorana fermions, the domain of $\Ind_\rho$ is $\Omega^3_{\Pin^+}(S^0)$.
	Here $S^0$ corresponds to the two points / theories $p - ip$ and $p + ip$.
	The $T$-symmetry exchanges these two, so we have $\Omega^3_{\Pin^+}(S^0) = \ZZ$.
	The index map corresponds to stacking the theories.
\end{ex}

\subsection{Residual family maps}

Recall that, for the $\rho$-defect to be local, the theory must be non-degenerately gapped on $S(\rho)$.
The map $\Res_\rho: \Omega^{D+1}_{G,s,\eta} \to \Omega^{D+1}_{G,s,\rho}(S(\rho))$ breaks the symmetry of the $(D+1)$-dimensional anomaly theory and tracks the $S(\rho)$-family.
Mathematically,
\[
	\Res_\rho(\omega)(X^{D+1}, A, \xi, \phi) = \omega(X^{D+1}, A, \xi).
\]
By ignoring $\phi$, we are allowing more flexibility in constructing counterterms.

This map is often zero -- for example, it is zero in the Majorana fermion example we considered.

\begin{ex}
	Consider a 2+1d Majorana fermion with $T^2 = (-1)^F$, taking $\rho$ to be two times the sign defect.
	Can we find $T$-odd operators $\Oc_1$, $\Oc_2$ such that, for $r \gg 0$ and all $\theta$, the theory plus $r \cos \theta \cdot \Oc_1 + r \sin \theta \Oc_2$ is trivially gapped?
	
	No!
	To compute the residual family anomaly, we can use any choice of $\Oc_1$ and $\Oc_2$, e.g.\
	\[
		\Oc_1 = \Oc_2 = i \ol{\psi} \psi.
	\]
	The theory is not $\rho$-gapped: any circle around the origin contains two gapless points (on the line $m = 0$).
	One of these points adds a $p + ip$, while the other point adds a $p - ip$.
	This corresponds to a 3+1d family on $S(\rho)$ where one point pumps $p+ip$ to 2 and the other pumps $p-ip$.
	Time reversal symmetry acts as a $\pi$ rotation.
	There's a $\ZZ_2$ invariant, and the residual family map $\Omega^4_{\Pin^+} \to \Omega^4_{\Pin^+}(S(\rho))$, equivalently $\ZZ_{16} \to \ZZ_2$, is reduction mod 2.
\end{ex}

\subsection{LES computations for nice groups}

It is easy to compute the symmetry breaking LES when the groups involved are uncomplicated.

\begin{ex}
	Consider $\Pin^+$ with $\rho$ being twice the sign representation.
	We can compute the cobordism groups as follows (with homomorphisms being computed using standard homological algebra):

	\begin{center}
		\begin{tabular}{|c|c|c|c|} \hline
			$D$ & $\Omega^{D-2}_{\Pin^-}$ & $\Omega^D_{\Pin^+}$ & $\Omega^D_{\Pin^+}(S(\rho)) = \tilde{\Omega}^{D+1}_{\Spin}(\RR\PP^2)$ \\ \hline
			$0$ & $0$ & $\ZZ_2$ & $\ZZ_2$ \\ \hline
			$1$ & $0$ & $0$ & $\ZZ_2$ \\ \hline
			$2$ & $\ZZ_2$ & $\ZZ_2$ & $\ZZ_4$ \\ \hline
			$3$ & $\ZZ_2$ & $\ZZ_2$ & $\ZZ_2$ \\ \hline
			$4$ & $\ZZ_8$ & $\ZZ_{16}$ & $\ZZ_2$ \\ \hline
		\end{tabular}
	\end{center}
\end{ex}

\begin{ex}
	For $\Pin^+$ with $\rho$ being the sign representation, we can also compute the cobordism groups.
	Standard homological algebra again gives a (partially complete) description of the maps.
	
	\begin{center}
		\begin{tabular}{|c|c|c|c|} \hline
			$D$ & $\Omega^{D-1}_{\ZZ_2,\Spin}$ & $\Omega^D_{\Pin^+}$ & $ \Omega^{D-1}_{\ZZ_2,\Spin} = \Omega^D_{\Spin}(S^0)$ \\ \hline
			$-1$ & $0$ & $0$ & $\ZZ$ \\ \hline
			$0$ & $\ZZ$ & $\ZZ_2$ & $0$ \\ \hline
			$1$ & $0$ & $0$ & $\ZZ_2$ \\ \hline
			$2$ & $\ZZ_2^2$ & $\ZZ_2$ & $\ZZ_2$ \\ \hline
			$3$ & $\ZZ_2^2$ & $\ZZ_2$ & $\ZZ$ \\ \hline
			$4$ & $\ZZ_8$ & $\ZZ_{16}$ & $0$ \\ \hline
		\end{tabular}
	\end{center}
\end{ex}
