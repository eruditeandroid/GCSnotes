\chapter{Claudia Scheimbauer -- Higher Categories for TQFT}

\section{6/13 -- Motivation, Examples, and First Definition}

There are many examples that shape how we might want to think about higher categories.

\subsection{Example 1: Categories}

The collection of all categories naturally forms a 2-category.
Instead of demanding isomorphism of categories, we typically only want to ask for ``equivalence.''
Formalizing this requires a notion of ``natural isomorphism.''

To encapsulate this information, we need a structure with three layers:
\begin{itemize}
	\item 0-morphisms / objects: categories,
	\item 1-morphisms: functors, and
	\item 2-morphisms: natural transformations.
\end{itemize}
We also need notions of composition of 1-morphisms and 2-morphisms.

\subsection{Example 2: Bordisms}

In the Segal-Atiyah approach to QFT, we assemble the possible ``spacetimes'' into a ``category'' $\Cob_{n,n-1}$ controlling cutting and pasting along codimension 1 slices.
The category $\Cob_{n,n-1}$ has:
\begin{itemize}
	\item 0-morphisms / objects: closed $(n-1)$-manifolds with boundary, and
	\item 1-morphisms $M_0 \to M_1$: $n$-dimensional smooth bordisms (i.e.\ manifolds $\Sigma$ with boundary together with a diffeomorphism $\partial \Sigma \xrightarrow{\sim} M_0 \sqcup M_1$).
\end{itemize}
Here composition is via gluing along the common boundary (but see below!).
We are often interested in adding additional structure to the objects of $\Cob_{n,n-1}$: Riemannian metrics, flat $G$-bundles, etc.

A (topological) quantum field theory is a symmetric monoidal functor $Z: \Cob_{n,n-1} \to \Vect$.
In the topological case, we require that all structures appearing in $\Cob$ are topological.
In more general cases, we may consider modifications of the target, e.g.\ replacing $\Vect$ by the category of topological vector spaces.
We'll focus on the case of TQFT.

As defined above, composition of bordisms does not necessarily produce a smooth bordism.
There are a few ways to fix this.
For example, we can consider bordisms only up to diffeomorphism, in which case we can adjust things so that composites inherit smooth structures.
Alternatively, we can enhance bordisms by adding the data of collars around the boundary.

The former approach reflects the fact that bordisms really live in a higher category, with:
\begin{itemize}
	\item 0-morphisms: closed $(n-1)$-manifolds,
	\item 1-morphisms: $n$-dimensional bordisms, and
	\item 2-morphisms: diffeomorphisms between bordisms up to isotopy.
\end{itemize}
This allows us access to mapping class groups, which are interesting!
From a physics perspective, we often want the values of our field theory to vary smoothly in the moduli space of bordisms.
Really, we want to consider the space of diffeomorphisms as well, so we should allow 3-morphisms corresponding to isotopies, 4-morphisms corresponding to isotopies of isotopies, etc.

We often take our bordisms to be pointing ``forward in time'' -- however, there's no mathematical reason to prefer a specific direction.
In particular, we should be able to compute $Z$ by slicing up our manifolds along different directions.
Note that, if we compare slices taken in two different directions, we typically get codimension 2 submanifolds along the intersections of the slices.
Iterating this process produces closed submanifolds of higher codimension, and we should explain what $Z$ assigns to such manifolds.
To this end, we can expand our bordism category to a $k$-category, allowing
\begin{itemize}
	\item 0-morphisms: closed $(n-k)$-manifolds,
	\item 1-morphisms: $(n-k+1)$-dimensional bordisms,
	\item \dots
	\item $(k - 1)$-morphisms: $(n-1)$-dimensional bordisms of bordisms of \dots,
	\item $k$-morphisms: $n$-dimensional bordisms of bordisms of \dots
\end{itemize}
Including the diffeomorphism groups discussed earlier, we should really consider an $(\infty, k)$-category $\Bord_{n,n-k}$.\footnote{For non-$\infty$ categories, we will write $\Cob$.
For $(\infty, -)$-categories, we will write $\Bord$.}

\subsection{Example 3: Targets}

If we want to define a TQFT as a functor out of $\Bord_{n,n-k}$, the codomain should also be an $(\infty, k)$-category.
What $(\infty, k)$-categories should we consider?

\begin{ex}
	For a first example, consider functors from $\Cob_{3,2,1}$.
	Following Segal's lectures, we may send $S^1$ to an algebra, 1-dimensional bordisms to bimodules, and 2-dimensional bordisms of bordisms to bimodule homomorphisms.
	That is, we take functors valued in the Morita bicategory $\Alg$, with:
	\begin{itemize}
		\item 0-morphisms: algebras (over our favorite field),
		\item 1-morphisms $A \to B$: $(A, B)$-bimodules, and
		\item 2-morphisms: bimodule homomorphisms.
	\end{itemize}
	In particular, the TQFT sends 2-manifolds or 2-dimensional bordisms to scalars or more generally linear maps.
\end{ex}

We can define other targets similarly.

\subsection{What is a (weak) 2-category?}

Now that we've completed our motivation, we should find an actual mathematical definition of higher categories.

Let's first recall how we define monoidal categories:

\begin{dfn}
	A \emph{monoidal category} is:
	\begin{itemize}
		\item A category $\Mc$, with
		\item A tensor product functor $\otimes: \Mc \times \Mc \to \Mc$,
		\item A unit object $\onebb$,
		\item An associator natural transformation $a_{A,B,C}:(A \otimes B) \otimes C \xrightarrow{\sim} A \otimes (B \otimes C)$, and
		\item Left and right unitor natural transformations $\lambda_A:\onebb \otimes A \xrightarrow{\sim} A \xleftarrow{\sim} A \otimes \onebb:\rho_A$.
	\end{itemize}
	These are required to satisfy:
	\begin{itemize}
		\item The ``pentagon axiom'', i.e., for all objects $A,B,C,D$ in $Mc$, the following diagram commutes:
		\begin{equation*}
            \begin{tikzcd}[column sep=0ex, row sep = large]
                &&(A\otimes B)\otimes C)\otimes D \arrow[dll, "a_{A\otimes B,C,D}"'] \arrow[drr, "a_{A B,C}\otimes Id_D"] \\
                (A\otimes B)\otimes(C\otimes D)\arrow[dr, "a_{A,B,C\otimes D}"']
                &&&&(A\otimes(B\otimes C))\otimes D \arrow[dl, "a_{A,B\otimes C,D}"] \\
                &A\otimes (B\otimes(C\otimes D))
                &&A\otimes ((B\otimes C)\otimes D)\arrow[ll, "Id_A\otimes a_{B,C,D}"],
            \end{tikzcd}
        \end{equation*}		
		and
		\item The ``triangle axiom'', i.e., for all objects $A,B$ in $Mc$, the following diagram commutes:
		\begin{equation*}
			\begin{tikzcd}
				(A\otimes \onebb)\otimes B \arrow[rr, "a_{A,\onebb,B}"] \arrow[dr, "\rho_A\otimes Id_B"']
				&&A\otimes (\onebb\otimes B) \arrow[dl, "Id_A\otimes\lambda_B"]\\
				&A\otimes B
			\end{tikzcd}
		\end{equation*}
	\end{itemize}
	We say that the monoidal category $\Mc$ is \emph{strict} if the associators and unitors are the identity maps.
\end{dfn}

The definition of weak 2-categories is quite similar!

\begin{dfn}
	A \emph{weak 2-category} or \emph{bicategory} $\Bc$ consists of:
	\begin{itemize}
		\item A collection of objects $\ob \Bc$,
		\item For all $X, Y \in \ob \Bc$, a category of morphisms $\Hom_{\Bc}(X, Y)$,
		\item For all $X, Y, Z \in \ob \Bc$, a composition functor 
		\begin{equation*}
			\circ: \Hom_{\Bc}(Y, Z) \times \Hom_{\Bc}(X, Z) \to \Hom_{\Bc}(X,Z),
		\end{equation*}
		\item For all $X \in \ob \Bc$, a functor 
		$$* \to \Hom_{\Bc}(X, X),$$
		corresponding to $\id_X\in Hom_{\Bc}(X,X)$ and $\id_{\id_X}:id_X\Rightarrow id_X$,
		
		\item An associator, i.e., for all $X,Y,Z,W \in \ob \Bc$,
		\begin{equation*}
			\begin{tikzcd}
				Hom_{\Bc}(Z,W)\times Hom_{\Bc}(Y,Z) \times Hom_{\Bc}(X,Y) \arrow[dd, shift left=-3ex, out=-135,in=135,"(-\circ(-\circ -)"']\arrow[dd, shift left=3ex, out=-45,in=45, "(-\circ -)\circ -"]\\
				\xRightarrow[a_{X,Y,Z,W}]{\cong}\\
				Hom_{\Bc}(X,W),
			\end{tikzcd}
		\end{equation*}
		and
		\item Left and right unitors , i.e., for all $X,Y \in \ob \Bc$,
		\begin{equation*}
			\begin{aligned}
			\begin{tikzcd}
				Hom_{\Bc}(X,Y)\arrow[dd, shift left=-2ex, out=-135,in=135,"Id_Y \circ -"']\arrow[dd, shift left=2ex, out=-45,in=45, "Id"]\\
				\xRightarrow[\lambda_{X,Y}]{\cong}\\
				Hom_{\Bc}(X,Y),
			\end{tikzcd}
			&&\begin{tikzcd}
				Hom_{\Bc}(X,Y)\arrow[dd, shift left=-2ex, out=-135,in=135,"- \circ Id_X"']\arrow[dd, shift left=2ex, out=-45,in=45, "Id"]\\
				\xRightarrow[\rho_{X,Y}]{\cong}\\
				Hom_{\Bc}(X,Y),
			\end{tikzcd}
			\end{aligned}
		\end{equation*}
	\end{itemize}
	satisfying the pentagon and triangle axioms.
	We say that a 2-category $\Bc$ is \emph{strict} if the associators and unitors are the identities.
\end{dfn}

In general, we need the flexibility of ``weak'' / ``non-strict'' constructions.
Most examples that arise in practice are not strict!

\section{6/13 -- Formal Definition}

\subsection{Examples of weak 2-categories}

Some examples we have considered are (possibly weak) 2-categories.

\begin{ex}
	Categories, functors, and natural transformations form a strict 2-category.
\end{ex}

\begin{ex}
	The collection of cobordisms down to codimension 2, $\Cob_{n,n-1,n-2}$, forms a weak 2-category.
\end{ex}

\begin{ex}
	Algebras, bimodules, and bimodule homomorphisms form a weak 2-category.
\end{ex}

Checking the examples can be difficult, especially when we move into higher-dimensional cases.
Furthermore, while weak 2-categories and strict 2-categories are (in some suitable sense) equivalent, this breaks down in higher dimensions.
We'd like to find a suitable framework for dealing with higher category theory.

\subsection{Simplicial sets and redefining categories}

There are many definitions of $(\infty, 1)$-categories in the literature, but all are equivalent.
The most common model is that of \emph{quasicategory}, called $\infty$-category by Lurie.
We'll use a somewhat different definition that is more suited to our needs.
Let's first consider the case of ordinary categories.

Let $\Delta$ be the skeletal category of nonempty linearly ordered finite sets and order preserving maps, i.e., the category with:
\begin{itemize}
	\item Objects: $[0]$, $[1]$, $[2]$, etc.\ where $[n] = \{ 0 < 1 < \dots < n \}$
	\item Morphisms: (non-strictly) order-preserving maps.
\end{itemize}

\begin{ex}
	From $[1]$ to $[2]$, there are six maps:
	\begin{itemize}
		\item $0 \mapsto 0, 1 \mapsto 1$,
		\item $0 \mapsto 0, 1 \mapsto 2$,
		\item $0 \mapsto 1, 1 \mapsto 2$, and
		\item $0 \mapsto k, 1 \mapsto k$ for $k = 0$, $1$, or $2$.
	\end{itemize}
\end{ex}

\begin{ex}
	From $[2]$ to $[1]$, there are four maps:
	\begin{itemize}
		\item $0 \mapsto 0, 1 \mapsto 1, 2 \mapsto 1$,
		\item $0 \mapsto 0, 1 \mapsto 0, 2 \mapsto 1$, and
		\item $0 \mapsto k, 1 \mapsto k, 2 \mapsto k$ for $k = 0$ or $1$.
	\end{itemize}
\end{ex}
Graphically,
\begin{equation*}
	\begin{tikzcd}
	{[}0{]} \arrow[r, shift left=2, blue] \arrow[r, shift right=2, blue]
	&{[}1{]} \arrow[l, purple] \arrow[r, shift left=4, blue] \arrow[r, shift right=4, blue] \arrow[r, blue]
	&{[}2{]}  \arrow[l, shift left = 2, purple]  \arrow[l, purple, shift right=2] \arrow[r, shift left=2, blue] \arrow[r, shift right=2, blue] \arrow[r, shift left=6, blue] \arrow[r, shift right=6, blue]
	&{[}3{]}... \arrow[l, shift left=4, purple] \arrow[l, shift right=4, purple] \arrow[l,purple]
	\end{tikzcd}
\end{equation*}
We can then get all maps in $\Delta$ by composing them.
\begin{dfn}
	A \emph{simplicial set} is a functor $X_\bullet: \Delta\op \to \Set$.
	More generally, a \emph{simplicial object} in a category $\Cc$ is a functor $X_\bullet: \Delta\op \to \Cc$.
	We write $X_n$ for the image of $[n]$ under $X_\bullet$.
\end{dfn}

\begin{ex}
	Let $\Cc$ be a small category.
	The \emph{nerve} of $\Cc$ is the simplicial set $N\Cc: \Delta\op \to \Set$ with $N\Cc_0 = \ob \Cc$ and $(N \Cc)_i$ the set of composable sequences of $i$ morphisms, i.e.,
	\begin{equation*}
		\begin{aligned}
			&N\Cc_0 = \ob \Cc \\
			&N\Cc_1 = mor \Cc \\
			&(N\Cc)_2 = \sqcup_{X,Y,Z \in \ob \Cc} \Hom_{\Cc}(Y, Z) \times \Hom_{\Cc}(X, Y) = \mor \Cc \times_{s,\ob \Cc,t} \mor \Cc \\
			&...\\
			&(N\Cc)_n = \sqcup_{X_0,...X_n \in \ob \Cc} \prod_{i=0}^{n-1} \Hom_{\Cc}(X_i, X_{i+1})
		\end{aligned}
	\end{equation*}
	The structure maps are given by source maps, target maps, composition, and diagonals.
\end{ex}

For every $i \in \{0, \dots, n-1\}$, there is a morphism $[1] \to [n]$ in $\Delta$ given by $0 \mapsto i$ and $1 \mapsto i+1$.
In a simplicial set $X$ these induce maps $\gamma_i: X_n \to X_1$.

\begin{ex}
	In $N \Cc$, these $\gamma_i$ are given by $(f_1, \dots, f_n) \mapsto f_i$.
\end{ex}

We may use the above discussion to arrive at a new definition of a category.

\begin{dfn}
	A \emph{category} is a simplicial set $X_\bullet$ such that the natural maps $X_n \to X_1 \times_{X_0} X_1 \times_{X_0} \dots \times_{X_0} X_1$ (induced by the $\gamma_i$) are bijections.
	A \emph{monoid} is a category with $X_0 = *$.
\end{dfn}

\subsection{Category objects}

The advantage of the above definition is that it allows us to generalize the notion of category.

\begin{dfn}
	Let $\Sc$ be a category with finite limits and monoidal structure given by the cartesian product. 
	A \emph{category object} in $\Sc$ is a simplicial object $X_\bullet: \Delta\op \to \Sc$ satisfying the \emph{Segal condition}: the natural maps $X_n \to X_1 \times_{X_0} \dots \times X_1$ are isomorphisms.
	If $\Sc$ has a notion of weak equivalence, then we take homotopy fiber products and require that the natural maps are weak equivalences.
	The simplicial structure is given by
	\begin{equation*}
		\begin{tikzcd}
			X_0 \arrow[r, purple] 
			&X_1 \arrow[l, shift left=2, blue] \arrow[l, shift right=2, blue] \arrow[r, shift left = 2, purple]  \arrow[r, purple, shift right=2] 
			&X_2 \arrow[l, shift left=4, blue] \arrow[l, shift right=4, blue] \arrow[l, blue] \arrow[r, shift left=4, purple] \arrow[r, shift right=4, purple] \arrow[r,purple] 
			&X_3... \arrow[l, shift left=2, blue] \arrow[l, shift right=2, blue] \arrow[l, shift left=6, blue] \arrow[l, shift right=6, blue]
		\end{tikzcd}
	\end{equation*}
	with the
	\begin{itemize} \color{blue}
		\item faces given by forgetting some interval (if it was $I_0$ or $I_n$, cut off the manifold) \color{purple}
		\item degeneracies given by doubling an interval, i.e.,
		$$I_0\leq I_1 \leq I_3 \mapsto I_0\leq I_1 \leq I_1 \leq I_2 $$
	\end{itemize}
\end{dfn}

\begin{ex}
	In $\Cat$ with weak equivalences given by isomorphisms, the category objects are strict 2-categories.
	We can use this to produce an inductive definition of strict $n$-category.
	Namely, strict $n$-categories are category objects in $\Cat_{n-1}$ (the category of strict $(n-1)$-categories) with weak equivalences given by isomorphisms.
\end{ex}

\begin{ex}
	In $\Cat$ with weak equivalences given by the usual equivalences, the category objects are weak 2-categories.
\end{ex}

We can also consider category objects in the $(\infty, 1)$-category $\Spaces$ of spaces (i.e.\ $\infty$-groupoids).
If we are careful enough, we can avoid circular reasoning.

\begin{dfn}
	A \emph{Segal space} (or \emph{pre-$(\infty, 1)$-category} or \emph{flagged $(\infty, 1)$-category}) is a category object in $\Spaces$, i.e.\ a simplicial space $X_\bullet: \Delta\op \to \Spaces$ satisfying the Segal condition $X_n \xrightarrow{\sim} X_1 \times_{X_0} \dots \times_{X_0} X_1$.
	An \emph{$(\infty, 1)$-category} is a flagged $(\infty, 1)$-category satisfying a ``univalence equals completeness axiom'' $X_0 \xrightarrow{sim} X_1^{\textrm{inv}}$.\footnote{This is essentially saying that we can reconstruct the objects from the equivalences.}
\end{dfn}

\subsection{Bordisms}

We would like to understand bordisms from this perspective.
Specifically, we seek to construct a simplicial object $X_\bullet = (\Bord_{d,d+1})_\bullet$ where the $X_n$ are composable sequences of $n$ bordisms.

From another perspective, we can see $X_n$ as the space of large bordisms with certain ``allowed'' cuts.

\begin{dfn}
	Let $X_n$ be the ``space'' with each point labeled by
	\begin{itemize}
		\item An interval $(a, b) \subset \RR$,
		\item An $n$-dimensional submanifold $M \subset (a, b) \times \RR^{\infty}$, and
		\item Closed subintervals $I_0, \dots, I_n \subset (a, b)$ with endpoints $a = a_0$, $b_0$, \dots, $a_n$, $b_n = b$ (giving collars of the cut values)
	\end{itemize}
	such that
	\begin{itemize}
		\item $a_0 \leq \dots \leq a_n$ and $b_0 \leq \dots \leq b_n$,
		\item The projection $\pi: M \to (a, b)$ is submersive over each $I_i$, and
		\item $\pi$ is proper.
	\end{itemize}
\end{dfn}

The upshot here is that we avoid thinking about singularities by focusing entirely on $n$-dimensional objects (with $(n-1)$-dimensional objects viewed as cylinders),
The restriction that $\pi$ is submersive over each $I_i$ avoids cutting at a Morse critical point (where the slice may be singular, e.g.\ a figure-eight).
The properness of $\pi$ ensures that all fibers are compact, so we don't have to worry about anything going off to infinity.

\begin{ex}
	For $X_0$, we have $I_0 = (a, b)$, so $\pi: M \to (a, b)$ is a proper submersion.
	Ehresmann's fibration theorem implies that $\pi$ is a fiber bundle, so contractibility of $(a, b)$ tells us that $M \cong M_0 \times (a, b)$ for some $M_0$.
	Thus $X_0$, the space of ``objects,'' can be identified with the space of $(n-1)$-manifolds.
\end{ex}

\section{6/14 -- More Bordisms; Factorization Algebras}

\subsection{Bordisms, continued}

Last time, we sketched the construction of a Segal space $X_\bullet = (\Bord_{d,d+1})_\bullet$.\footnote{This is not complete in high dimensions due to the existence of $H$-cobordisms.
Details can be found in Lurie's paper on the cobordism hypothesis.}
More precisely, we described the points of $X_n$ as bordisms with $n$ allowed cut loci.
The topology is hard to describe precisely but comes from allowing ``wiggling'' of both the bordism and the intervals.

We should also describe the structure maps of the simplicial space.
The face maps $X_i \to X_{i-1}$ are given by forgetting cuts (for inner faces) or discarding the outer parts of the manifold (for outer faces).
Degeneracy maps $X_i \to X_{i+1}$ correspond to repeating intervals.
These maps (and the intervals themselves) are essentially bookkeeping.

\subsection{Extended bordisms and the cobordism hypothesis}

Recall that $(\infty, 1)$-categories are defined as simplicial spaces $\Delta\op \to \Spaces$ satisfying the Segal condition and a completeness condition.
Similarly, we define $(\infty, k)$-categories to be $k$-simplicial spaces $(\Delta\op)^{\times k} \to \Spaces$ satisfying a Segal condition, a completeness condition, and ``globularity.''
From this perspective, defining $(\infty, k)$-categories is not too much harder than defining $(\infty, 1)$-categories.

To define an $(\infty, k)$-category of bordisms $\Bord_{d,d-k}$, we:
\begin{itemize}
	\item Embed $M$ in $\RR^k \times \RR^\infty$,
	\item Take the ``intervals'' labeling our cuts to be rectangles in $\RR^k$, and
	\item Modify the submersivity condition on $\pi$.
\end{itemize}

For $k = d$, we get an $(\infty, d)$-category $\Bord_d := \Bord_{d,0}$.
We say that a topological field theory is \emph{fully extended} if it is defined on $\Bord_d$.

The \emph{cobordism hypothesis} (due to Baez-Dolan, Lurie, and others) is a way of formalizing the notion that (fully extended) TFTs are fully local.
This can be stated as follows.
Let $\Cc$ be a symmetric monoidal $(\infty, d)$-category.
Write $\Bord_d^{\mathrm{fr}}$ for the \emph{framed} bordism category.
Then evaluation at $\pt$ gives an equivalence
\[
	\Fun^\otimes(\Bord_d^{\mathrm{fr}}, \Cc) \xrightarrow{\sim} (\Cc^{d\mathrm{-dualizable}})^{\simeq}
\]
between the $(\infty, d)$-category of symmetric monoidal functors from $\Bord_d^{\mathrm{fr}}$ to $\Cc$ and the maximal $\infty$-groupoid of $d$-dualizable objects in $\Cc$.
For more details, see the lectures of David Jordan at the GCS summer school last year.

There are also variants allowing for different tangential structures on the TFTs.
The above discussion doesn't account for anomalous theories; corrections are needed to deal with these.

\subsection{TFT targets}

What should we take as the codomain $\Cc$ of a TFT?

\begin{ex}
	One weak $2$-category that shows up frequently in practice is the category $\Alg_1(\Vect)$ (denoted $\Alg$ earlier) with:
	\begin{itemize}
		\item Objects: algebras,
		\item 1-morphisms: bimodules, and
		\item 2-morphisms: bimodule homomorphisms.
	\end{itemize}
	Here composition is given by the relative tensor product: ${}_B N_C \circ {}_A M_B = {}_A M \otimes_B N_C$.
	Note that this is only weakly associative.
\end{ex}

\begin{ex}
	Trying to define a 3-categorical example, we can let $\Alg_1(\Cat^{\textrm{lin}})$ be the weak 3-category with:
	\begin{itemize}
		\item Objects: tensor categories,
		\item 1-morphisms: bimodule categories,
		\item 2-morphisms: functors,
		\item 3-morphisms: natural transformations.
	\end{itemize}
	This is tricky -- we first have to figure out what we mean by ``weak 3-category.''
	However, this example is relevant to practical examples such as Witten-Reshetikhin-Turaev theories and Turaev-Viro theories.
\end{ex}

\begin{ex}
	Recent work has made progress in defining $\Alg_2(\Cat^{\textrm{lin}})$ and $\Alg_3(\Cat^{\textrm{lin}})$.
	This gets progressively harder as the subscript increases!
\end{ex}

It seems difficult (but worthwhile) to understand what's going on here.
We'll instead think about a different approach.

\subsection{(Pointless) factorization algebras}

Factorization algebras show up frequently in studies of TFT:
\begin{itemize}
	\item They give a variation of algebraic QFT (which assigns an algebra $\Ac(U)$ to every open subset $U$ of spacetime).
	\item They axiomatize observables in (perturbative) QFT -- the work of Costello-Gwilliam has explained BV quantization from this perspective.
	\item They encapsulate $A_\infty$-algebras, bimodules, $\EE_n$-algebras, etc.
\end{itemize}
We'll try to use these to understand TFT.

\begin{dfn}
	Let $X$ be a topological space.
	The operad $\Open(X)^{\otimes}$ of opens in $X$ has open sets of $X$ as colors and one operation $(U_1, \dots, U_n; V)$ if the $U_i$ are disjoint subsets of $V$.
	A \emph{prefactorization algebra} on $X$ with values in $(\Sc, \otimes)$ (often $\Sc = \Ch$) is an algebra $\Ac$ for this operad.
	In more detail, for every $U_1 \sqcup \dots \sqcup U_n \subset V$, we have $\Ac(U_1) \otimes \dots \Ac(U_n) \to \Ac(V)$ ``coherently.''
	That is, for $O_1 \sqcup \dots \sqcup O_n \subset U_1$, $O_1' \sqcup \dots \subset U_2$, \dots, and $U_1 \sqcup U_2 \sqcup \dots \subset V$, the triangle
	\[
		\begin{tikzcd}
			\Ac(O_1) \otimes \dots \otimes \Ac(O_n) \otimes \Ac(O_1') \otimes \dots \ar[rr] \ar[dr] & & \Ac(U_1) \otimes \Ac(U_2) \otimes \dots \ar[dl] \\
														& \Ac(V) &
		\end{tikzcd}
	\]
	commutes (up to coherent homotopy).
	This last condition is encapsulated nicely by the language of operads and $(\infty, 1)$-categories.
\end{dfn}

\begin{rmk}
	This makes sense with $\Open(X)$ replaced by any poset of opens in $X$.
\end{rmk}

Let's see how this captures the examples we've seen so far.

\begin{ex}
	Let $X = \RR$ and let $A$ be an associative algebra over a field $k$.
	We can define $\Ac$ on the poset $\Intsf$ of intervals in $X$ via the assignment $\Ac(I) = A$, with $\Ac(\emptyset) = k$.
	For nonempty $I$, the map $\Ac(\emptyset) \to \Ac(I)$, equivalently $k \to A$, gives the unit of $A$.
	For nonempty $I_1, I_2 \subset I_3$, the map $\Ac(I_1) \otimes \Ac(I_2) \to \Ac(I_3)$, equivalently $A \otimes A \to A$, gives the multiplication on $A$.
	Coherence of $\Ac$ corresponds to associativity of $A$.
\end{ex}

\begin{ex}
	Let $p \in X = \RR$, let $M$ be an $(A, B)$-bimodule, and let $m \in M$.
	Define a prefactorization algebra on $\Intsf$ by $\Ac(\emptyset) = k$
	\[
		\Ac((x, y)) = \begin{cases}
			A & y \leq p \\
			M & x < p < y \\
			B & p < x.
		\end{cases}
	\]
	This obtains a natural structure of prefactorization algebra.
	(The maps $k \to M$ are just specified by the choice of $m$.)
\end{ex}

\begin{ex}
	Let $X = \RR^2$ and consider the poset of disks within the unit disk.
	The corresponding operad is the $\EE_2$-operad.
	The corresponding prefactorization algebras in $\Vect$ are commutative algebras, while the corresponding prefactorization algebras in $\Cat$ are braided monoidal categories.
\end{ex}

\section{6/14 -- Factorization algebras and the Morita category}

\subsection{Factorization algebras}

We've talked about prefactorization algebras -- what does it mean to drop the ``pre?''

\begin{dfn}
	A \emph{factorization algebra} on a topological space $X$ with values in $(\Sc, \otimes)$ is a prefactorization algebra $\Ac$ such that:
	\begin{enumerate}
		\item $\Ac$ is multiplicative, i.e.\ $\Ac(U_1) \otimes \dots \otimes \Ac(U_n) \xrightarrow{\sim} \Ac(U_1 \sqcup \dots \sqcup U_n)$ and $\Ac(\emptyset) \simeq \onebb$.
		\item $\Ac$ satisfies descent for Weiss covers.
	\end{enumerate}
	Here a \emph{Weiss cover} is a collection $\Uc$ of open sets of $X$ such that for all finite subsets $S \subset X$, there exists $U_S \in \Uc$ such that $S \subset U_S$.
	Descent means that, if $\Uc$ is a Weiss cover of $V$, then
	\[
		\colim_{\{ U_1, \dots, U_n \} \in \Uc} \Ac(U_1 \cap \dots U_n) \xrightarrow{\sim} \Ac(V).
	\]
\end{dfn}

The Weiss condition is a bit strange and hard to think about.
However, it gives us a way to extend from a prefactorization algebra on a basis to a factorization algebra on the whole space.
This can be implemented using Kan extensions -- see work of Karlsson-Scheimbauer-Walde.

\begin{ex}
	Given an associative algebra $A$, we get a prefactorization algebra $\Ac$ on $\Intsf$ as before.
	Multiplicativity tells us how to extend this to $\Intsf^{\sqcup}$ (finite disjoint unions of intervals).
	The Weiss condition tells us how to extend to a factorization algebra on all of $\Open(\RR)$.
\end{ex}

\subsection{Constructibility}

We are particularly interested in prefactorization algebras that interact well with stratified manifold structures.

\begin{dfn}
	Let $X$ be a conically smooth manifold.\footnote{This is a ``nice'' stratified space, e.g.\ a stratification of an $n$-dimensional manifold by smooth submanifolds.}
	A prefactorization algebra $\Ac$ on $X$ is \emph{constructible} if, for every inclusion $U \subset V$ of abstractly isomorphic disks (accounting for the stratification), we get an equivalence $\Ac(U) \xrightarrow{\sim} \Ac(V)$.
\end{dfn}

\begin{thm}[Lurie]
	Equip $\RR^n$ with the trivial stratification $\emptyset \subset \RR^n$.
	There is an equivalence
	\[
		\Fact_{\RR^n}\cstr \simeq \EE_n\dAlg.
	\]
\end{thm}

\begin{thm}[Ayala-Francis-Tanaka]
	Equip $\RR$ with the stratification $\emptyset \subset \{ p \} \subset \RR$.
	There is an equivalence between $\Fact_{\RR}\cstr$ and the $(\infty, 1)$-category of pointed bimodules. 
\end{thm}

Constructible factorization algebras glue, so it suffices to define them locally.
That is:

\begin{thm}[Karlsson-Scheimbauer-Walde]
	Given an open cover $\Uc$ of $X$, there is an equivalence
	\[
		\colim_{\{U_1, \dots, U_n\} \subset \Uc} \Fact_{U_1 \cap \dots U_n}\cstr \xrightarrow{\sim} \Fact_X\cstr.
	\]
\end{thm}

\begin{ex}
	We can glue our ``pointed bimodule'' example to get a picture of smashing defects on the real line.
\end{ex}

The preceding theorem is surprisingly difficult to prove -- it had been claimed for years, but a proof is only appearing now.
It is not known whether the result extends to all factorization algebras.
The ultimate reason is that Weiss covers are hard to work with!

One can also prove other useful results in the constructible context.

\begin{thm}[KSW]
	The category $\Fact_X\cstr$ has a symmetric monoidal structure given (at a na\"ive level) by $(\Fc \otimes \Gc)(U) = \Fc(U) \otimes \Gc(U)$.
\end{thm}

It's also possible to understand constructible factorization algebras on a cone.

\begin{thm}[Brav-Rozenblyum, KSW]
	Let $C(Z) = Z \times [0, 1) / (Z \times \{0\})$.
	There is a pullback square
	\[
		\begin{tikzcd}
			\Fact_{C(Z)}\cstr \rar \dar & \Fact_{Z \times (0,1)}\cstr \dar \\
			\Fact_{[0, 1)}\cstr \rar & \Fact_{(0,1)}\cstr.
		\end{tikzcd}
	\]
\end{thm}

One can understand the terms in this pullback individually:
\begin{itemize}
	\item $\Fact_{[0,1)}\cstr$ consists of algebras together with a pointed left module.
	\item $\Fact_{(0,1)}\cstr \simeq \EE_1\dAlg$ by Lurie's result above.
	\item $\Fact_{Z \times (0,1)}\cstr \simeq \Fact_Z\cstr(\EE_1\dAlg)$ by work in progress of \v{S}vraka.
\end{itemize}
A few more comments on this last equivalence: Ayala-Berry proved that
\[
	\Fact_{X \times Y} \simeq \Fact_X(\Fact_Y)
\]
in the non-constructible case.
The work of \v{S}vraka extends this to the constructible case.

\subsection{Pointless factorization algebras}

We'd like to use the above results to understand the Morita bicategory and other important target categories in TFT.
However, what we've done so far requires \emph{pointed} bimodules, which aren't quite what we want!
Work in progress of Karlsson fixes this.

Let $X$ be a stratified space, and let $X_0$ be a collection of chosen points (forming a 0-dimensional stratum).
Replace $\Open(X)$ by the poset $\Open^m(X)$ consisting of all open sets of $X$ but with only inclusions $U \subset V$ such that $U \cup X_0 \xrightarrow{\sim} V \cap X_0$ abstractly.
Karlsson's work redoes the above theorems in this context.

\subsection{The Morita category}

From the above perspective, we can construct the Morita category as the Segal space $X_\bullet: \Delta\op \to \Spaces$ with:
\begin{itemize}
	\item $X_0$ consisting of constructible factorization algebras on $\RR$ (equivalently, $\EE_1$-algebras),
	\item $X_1$ consisting of constructible factorization algebras on $\RR$ with one marked point (almost...), i.e.\ bimodules,
	\item $X_2$ consisting of constructible factorization algebras on $\RR$ with two marked points (almost...), i.e.\ pairs of compatible bimodules,
	\item \dots
\end{itemize}
The face maps come from either forgetting the leftmost or rightmost algebra and bimodule or tensoring adjacent pairs of bimodules.
From the factorization algebra viewpoint, this corresponds to a pushforward.
Degeneracy maps require some cleverness -- for example, we can use an approach like that used for bordisms above.
One can also modify the definitions slightly to get an easier definition.
There's some hard bookkeeping so that we can allow the points to vary.

We can also replace $\Spaces$ by $\Cat_{(\infty,1)}$ or other interesting targets.
This recovers other notions of Morita categories.

This construction is due to Calaque-Scheimbauer and is in Scheimbauer's thesis.
Work of Karlsson will extend this to the pointless case.
Haugseng accomplishes a similar goal without factorization algebras.

\subsection{Outlook}

From this perspective, we can construct a pointed version of $\Alg_n$ based on a stratification of $\RR^n$.
To understand dualizability, work of Karlsson uses the geometry of factorization algebras (as opposed to the operadic approach of Haugseng).
Future work will compare Haugseng's approach to the pointless version.

There is also much work on understanding unitarity in this context.
Many people involved in this task will be at the conference.
