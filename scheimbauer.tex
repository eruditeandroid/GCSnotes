\chapter{Claudia Scheimbauer -- Higher Categories and TQFT}

\section{6/13 -- Motivation, Examples, and First Definition}

There are many examples that shape how we might want to think about higher categories.

\subsection{Example 1: Categories}

The collection of all categories naturally forms a 2-category.
Instead of demanding isomorphism of categories, we typically only want to ask for ``equivalence.''
Formalizing this requires a notion of ``natural isomorphism.''

To encapsulate this information, we need a structure with three layers:
\begin{itemize}
	\item 0-morphisms / objects: categories,
	\item 1-morphisms: functors, and
	\item 2-morphisms: natural transformations.
\end{itemize}
We also need notions of composition of 1-morphisms and 2-morphisms.

\subsection{Example 2: Bordisms}

In the Segal-Atiyah approach to QFT, we assemble the possible ``spacetimes'' into a ``category'' $\Cob_{n,n-1}$ controlling cutting and pasting along codimension 1 slices.
The category $\Cob_{n,n-1}$ has:
\begin{itemize}
	\item 0-morphisms / objects: closed $(n-1)$-manifolds with boundary, and
	\item 1-morphisms $M_0 \to M_1$: $n$-dimensional smooth bordisms (i.e.\ manifolds $\Sigma$ with boundary together with a diffeomorphism $\partial \Sigma \xrightarrow{\sim} M_0 \sqcup M_1$).
\end{itemize}
Here composition is via gluing along the common boundary (but see below!).
We are often interested in adding additional structure to the objects of $\Cob_{n,n-1}$: Riemannian metrics, flat $G$-bundles, etc.

A (topological) quantum field theory is a symmetric monoidal functor $Z: \Cob_{n,n-1} \to \Vect$.
In the topological case, we require that all structures appearing in $\Cob$ are topological.
In more general cases, we may consider modifications of the target, e.g.\ replacing $\Vect$ by the category of topological vector spaces.
We'll focus on the case of TQFT.

As defined above, composition of bordisms does not necessarily produce a smooth bordism.
There are a few ways to fix this.
For example, we can consider bordisms only up to diffeomorphism, in which case we can adjust things so that composites inherit smooth structures.
Alternatively, we can enhance bordisms by adding the data of collars around the boundary.

The former approach reflects the fact that bordisms really live in a higher category, with:
\begin{itemize}
	\item 0-morphisms: closed $(n-1)$-manifolds,
	\item 1-morphisms: $n$-dimensional bordisms, and
	\item 2-morphisms: diffeomorphisms between bordisms up to isotopy.
\end{itemize}
This allows us access to mapping class groups, which are interesting!
From a physics perspective, we often want the values of our field theory to vary smoothly in the moduli space of bordisms.
Really, we want to consider the space of diffeomorphisms as well, so we should allow 3-morphisms corresponding to isotopies, 4-morphisms corresponding to isotopies of isotopies, etc.

We often take our bordisms to be pointing ``forward in time'' -- however, there's no mathematical reason to prefer a specific direction.
In particular, we should be able to compute $Z$ by slicing up our manifolds along different directions.
Note that, if we compare slices taken in two different directions, we typically get codimension 2 submanifolds along the intersections of the slices.
Iterating this process produces closed submanifolds of higher codimension, and we should explain what $Z$ assigns to such manifolds.
To this end, we can expand our bordism category to a $k$-category, allowing
\begin{itemize}
	\item 0-morphisms: closed $(n-k)$-manifolds,
	\item 1-morphisms: $(n-k+1)$-dimensional bordisms,
	\item \dots
	\item $(n - 1)$-morphisms: $(n-1)$-dimensional bordisms of bordisms of \dots,
	\item $n$-morphisms: $n$-dimensional bordisms of bordisms of \dots
\end{itemize}
Including the diffeomorphism groups discussed earlier, we should really consider an $(\infty, k)$-category $\Bord_{n,n-k}$.\footnote{For non-$\infty$ categories, we will write $\Cob$.
For $(\infty, -)$-categories, we will write $\Bord$.}

\subsection{Example 3: Targets}

If we want to define a TQFT as a functor out of $\Bord_{n,n-k}$, the codomain should also be an $(\infty, k)$-category.
What $(\infty, k)$-categories should we consider?

\begin{ex}
	For a first example, consider functors from $\Cob_{3,2,1}$.
	Following Segal's lectures, we may send $S^1$ to an algebra, 1-dimensional bordisms to bimodules, and 2-dimensional bordisms of bordisms to bimodule homomorphisms.
	That is, we take functors valued in the Morita bicategory $\Alg$, with:
	\begin{itemize}
		\item 0-morphisms: algebras (over our favorite field),
		\item 1-morphisms $A \to B$: $(A, B)$-bimodules, and
		\item 2-morphisms: bimodule homomorphisms.
	\end{itemize}
	In particular, the TQFT sends 2-manifolds or 2-dimensional bordisms to scalars or more generally linear maps.
\end{ex}

We can define other targets similarly.

\subsection{What is a (weak) 2-category?}

Now that we've completed our motivation, we should find an actual mathematical definition of higher categories.

Let's first recall how we define monoidal categories:

\begin{dfn}
	A \emph{monoidal category} is:
	\begin{itemize}
		\item A category $\Mc$, with
		\item A tensor product functor $\otimes: \Mc \times \Mc \to \Mc$,
		\item A unit object $\onebb$,
		\item An associator natural transformation $(A \otimes B) \otimes C \xrightarrow{\sim} A \otimes (B \otimes C)$, and
		\item Left and right unitor natural transformations $\onebb \otimes A \xrightarrow{\sim} A \xleftarrow{\sim} A \otimes \onebb$.
	\end{itemize}
	These are required to satisfy:
	\begin{itemize}
		\item A ``pentagon axiom'' ensuring nice behavior of the associator, and
		\item A ``triangle axiom'' ensuring nice behavior of the unitors.
	\end{itemize}
	We say that the monoidal category $\Mc$ is \emph{strict} if the associators and unitors are the identity maps.
\end{dfn}

The definition of weak 2-categories is quite similar!

\begin{dfn}
	A \emph{weak 2-category} or \emph{bicategory} $\Bc$ consists of:
	\begin{itemize}
		\item A collection of objects $\ob \Bc$,
		\item For all $X, Y \in \ob \Bc$, a category of morphisms $\Hom_{\Bc}(X, Y)$,
		\item For all $X, Y, Z \in \ob \Bc$, a composition functor $\circ: \Hom_{\Bc}(Y, Z) \times \Hom_{\Bc}(X, Z)$,
		\item For all $X \in \ob \Bc$, a functor $* \to \Hom_{\Bc}(X, X)$ (corresponding to $\id_X$ and $\id_{\id_X}$),
		\item An associator (natural isomorphism between threefold composites), and
		\item Unitors (natural isomorphisms between morphisms and their composites with the identity)
	\end{itemize}
	satisfying the pentagon and triangle axioms.
	We say that a 2-category $\Bc$ is \emph{strict} if the associators and unitors are the identities.
\end{dfn}

In general, we need the flexibility of ``weak'' / ``non-strict'' constructions.
Most examples that arise in practice are not strict!

\section{6/13 -- Formal Definition}

\subsection{Examples of weak 2-categories}

Some examples we've considered are (possibly weak) 2-categories.

\begin{ex}
	Categories, functors, and natural transformations form a strict 2-category.
\end{ex}

\begin{ex}
	The collection of cobordisms down to codimension 2, $\Cob_{n,n-1,n-2}$, forms a weak 2-category.
\end{ex}

\begin{ex}
	Algebras, bimodules, and bimodule homomorphisms form a weak 2-category.
\end{ex}

Checking the examples can be difficult, especially when we move into higher-dimensional cases.
Furthermore, while weak 2-categories and strict 2-categories are (in some suitable sense) equivalent, this breaks down in higher dimensions.
We'd like to find a suitable framework for dealing with higher category theory.

\subsection{Simplicial sets and redefining categories}

There are many definitions of $(\infty, 1)$-categories in the literature, but all are equivalent.
The most common model is that of \emph{quasicategory}, called $\infty$-category by Lurie.
We'll use a somewhat different definition that is more suited to our needs.
Let's first consider the case of ordinary categories.

Let $\Delta$ be the skeletal category of nonempty linearly ordered finite sets, i.e.\ the category with:
\begin{itemize}
	\item Objects: $[0]$, $[1]$, $[2]$, etc.\ where $[n] = \{ 0 < 1 < \dots < n \}$
	\item Morphisms: (non-strictly) order-preserving maps.
\end{itemize}

\begin{ex}
	From $[1]$ to $[2]$, there are six maps:
	\begin{itemize}
		\item $0 \mapsto 0, 1 \mapsto 1$,
		\item $0 \mapsto 0, 1 \mapsto 2$,
		\item $0 \mapsto 1, 1 \mapsto 2$, and
		\item $0 \mapsto k, 1 \mapsto k$ for $k = 0$, $1$, or $2$.
	\end{itemize}
\end{ex}

\begin{ex}
	From $[2]$ to $[1]$, there are four maps:
	\begin{itemize}
		\item $0 \mapsto 0, 1 \mapsto 1, 2 \mapsto 1$,
		\item $0 \mapsto 0, 1 \mapsto 0, 2 \mapsto 1$, and
		\item $0 \mapsto k, 1 \mapsto k, 2 \mapsto k$ for $k = 0$ or $1$.
	\end{itemize}
\end{ex}

\begin{dfn}
	A \emph{simplicial set} is a functor $X_\bullet: \Delta\op \to \Set$.
	More generally, a \emph{simplicial object} in a category $\Cc$ is a functor $X_\bullet: \Delta\op \to \Cc$.
	We write $X_n$ for the image of $[n]$ under $X_\bullet$.
\end{dfn}

\begin{ex}
	Let $\Cc$ be a small category.
	The \emph{nerve} of $\Cc$ is the simplicial set $N\Cc: \Delta\op \to \Set$ with $N\Cc_0 = \ob \Cc$ and $(N \Cc)_i$ the set of composable sequences of $i$ morphisms.
	For example,
	\[
		(N\Cc)_2 = \sqcup_{X,Y,Z \in \ob \Cc} \Hom_{\Cc}(Y, Z) \times \Hom_{\Cc}(X, Y) = \mor \Cc \times_{s,\ob \Cc,t} \mor \Cc
	\]
	The structure maps are given by source maps, target maps, composition, and diagonals.
\end{ex}

For every $i \in \{0, \dots, n-1\}$, there is a morphism $[1] \to [n]$ in $\Delta$ given by $0 \mapsto i$ and $1 \mapsto i+1$.
In a simplicial set $X$ these induce maps $\gamma_i: X_n \to X_1$.

\begin{ex}
	In $N \Cc$, these $\gamma_i$ are given by $(f_1, \dots, f_n) \mapsto f_i$.
\end{ex}

We may use the above discussion to arrive at a new definition of a category.

\begin{dfn}
	A \emph{category} is a simplicial set $X_\bullet$ such that the natural maps $X_n \to X_1 \times_{X_0} X_1 \times_{X_0} \dots \times_{X_0} X_1$ (induced by the $\gamma_i$) are bijections.
	A \emph{monoid} is a category with $X_0 = *$.
\end{dfn}

\subsection{Category objects}

The advantage of the above definition is that it allows us to generalize the notion of category.

\begin{dfn}
	Let $\Sc$ be a category with finite limits. 
	A \emph{category object} in $\Sc$ is a simplicial object $X_\bullet: \Delta\op \to \Sc$ satisfying the \emph{Segal condition}: the natural maps $X_n \to X_1 \times_{X_0} \dots \times X_1$ are isomorphisms.
	If $\Sc$ has a notion of weak equivalence, then we take homotopy fiber products and require that the natural maps are weak equivalences.
\end{dfn}

\begin{ex}
	In $\Cat$ with weak equivalences given by isomorphisms, the category objects are strict 2-categories.
	We can use this to produce an inductive definition of strict $n$-category.
	Namely, strict $n$-categories are category objects in $\Cat_{n-1}$ (the category of strict $(n-1)$-categories) with weak equivalences given by isomorphisms.
\end{ex}

\begin{ex}
	In $\Cat$ with weak equivalences given by the usual equivalences, the category objects are weak 2-categories.
\end{ex}

We can also consider category objects in the $(\infty, 1)$-category $\Spaces$ of spaces (i.e.\ $\infty$-groupoids).
If we are careful enough, we can avoid circular reasoning.

\begin{dfn}
	A \emph{Segal space} (or \emph{pre-$(\infty, 1)$-category} or \emph{flagged $(\infty, 1)$-category}) is a category object in $\Spaces$, i.e.\ a simplicial space $X_\bullet: \Delta\op \to \Spaces$ satisfying the Segal condition $X_n \xrightarrow{\sim} X_1 \times_{X_0} \dots \times_{X_0} X_1$.
	An \emph{$(\infty, 1)$-category} is a flagged $(\infty, 1)$-category satisfying a ``univalence equals completeness axiom'' $X_0 \xrightarrow{sim} X_1^{\textrm{inv}}$.\footnote{This is essentially saying that we can reconstruct the objects from the equivalences.}
\end{dfn}

\subsection{Bordisms}

We would like to understand bordisms from this perspective.
Specifically, we seek to construct a simplicial object $X_\bullet = (\Bord_{d,d+1})_\bullet$ where the $X_n$ are composable sequences of $n$ bordisms.

From another perspective, we can see $X_n$ as the space of large bordisms with certain ``allowed'' cuts.

\begin{dfn}
	Let $X_n$ be the ``space'' with each point labeled by
	\begin{itemize}
		\item An interval $(a, b) \subset \RR$,
		\item An $n$-dimensional submanifold $M \subset (a, b) \times \RR^{\infty}$, and
		\item Closed subintervals $I_0, \dots, I_n \subset (a, b)$ with endpoints $a = a_0$, $b_0$, \dots, $a_n$, $b_n = b$ (giving collars of the cut values)
	\end{itemize}
	such that
	\begin{itemize}
		\item $a_0 \leq \dots \leq a_n$ and $b_0 \leq \dots \leq b_n$,
		\item The projection $\pi: M \to (a, b)$ is submersive over each $I_i$, and
		\item $\pi$ is proper.
	\end{itemize}
\end{dfn}

The upshot here is that we avoid thinking about singularities by focusing entirely on $n$-dimensional objects (with $(n-1)$-dimensional objects viewed as cylinders),
The restriction that $\pi$ is submersive over each $I_i$ avoids cutting at a Morse critical point (where the slice may be singular, e.g.\ a figure-eight).
The properness of $\pi$ ensures that all fibers are compact, so we don't have to worry about anything going off to infinity.

\begin{ex}
	For $X_0$, we have $I_0 = (a, b)$, so $\pi: M \to (a, b)$ is a proper submersion.
	Ehresmann's fibration theorem implies that $\pi$ is a fiber bundle, so contractibility of $(a, b)$ tells us that $M \cong M_0 \times (a, b)$ for some $M_0$.
	Thus $X_0$, the space of ``objects,'' can be identified with the space of $(n-1)$-manifolds.
\end{ex}

